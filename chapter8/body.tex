\chapter{APPLICATIONS TO NICKEL EMBEDDED ATOM POTENTIALS}

The aim of this chapter is to apply the earlier described methods and tools involved in developing potentials for metallic systems.  The defining difference between metals and nonmetals is the lack of an energy gap between ground and excited states.  Metallic systems are normally characterized as a system existing within a sea of delocalized electrons.

\section{Insights from Quantum Mechancial Techniques}

In the development of the Finnis-Sinclair method, the second-moment tight-binding(3)

\section{Embedded Atom Model}

For metallic systems, the most common potential forms are those of the Finnis-Sinclar method\cite{finnis1984_fs} and the embedded atom model (EAM) \cite{daw1983_eam,daw1984_eam}.
Both of these approaches have similar formalisms which combine a pair-potential function and an energy functional dependent upon the electron density contributions from an atoms neighbors with the total energy of the system $V$ being described as

\begin{equation}
  V=\sum_{i<j}\phi_{s_i s_j}(r_{ij})
	+\sum_{i}F_{s_i}(\bar{\rho}_{i})
\end{equation}

The first term is the summation over all neighbors of a pair potential of a pair potential energy($\phi_{ij}$) between two atoms, $i$ and $j$, with the chemical species, $s_i$ and $s_j$.
The second term is the embedding energy ($F_i$) necessary to place the atom $i$ at its position in an electron gas density ($\bar{\phi}_i$) influenced by all the neighbors of the atom.
In EAM, the host electron density is given by the sum of the contributions $\rho_i$ is represented by the sum of the contributions $\rho_{s_j}(r)$ from all neighboring atoms $j$.
\begin{equation}
	\bar{\rho}_i = \sum_{j \neq i} \rho_{s_j}(r_{ij})
\end{equation}

In the Finnis-Sinclair potential, $F_{s_i}(\bar{\rho})=-\sqrt{\bar{\rho}_i}$

\section{Approaches to EAM Potential Development}

The original formulation of the EAM represented the reepulsion between the atomic cores (nuclei and inner electron shells), represented by a power law or an Born-Mayer type exponential function.  In these cases, the pair potential, $\phi$, is purely repulsive, while the embedding energy $F$ was attractive.
Later EAM pair potentials used Morse Functions[10,19,20], which has a distinct energy well, while the embedding function in these cases serves as a corrective term.

In either case, the electron density function and the pair potential are given as analytic functions of the radial separation of two atoms $r_{ij}$ with fitted parameters.  The embedding function is sometimes specified as having a specific functional form, or determined by fitting the embedding function to an equation of state such as Rose \emph{et al}[21].

For Nickel, Johnson and Oh\cite{johnson1989_eam_embedded_bjs}, Voter-Chen[19,23], Angelo \emph{et al}[20] have used parametric functional forms, where Ercolessi and Adams[24] and Mishin \emph{25} implemented cubic-spline approaches, which do not specify an analytical functional form.

In early potential development, the parameters are adjusted to match properties such cohesive energy, lattice parameters, elastic properties, and vacancy formation energies.  With additional computational power, stacking fault energies, phase order differences, and surface energies to fit to important relevant environments thought to be important to produce transferrable potentials.

\section{Development of an EAM potential}

Even when a potential has an analytical functional form. EAM potentials are specified in a functional form which is dependent a specific code format which precalculates evaluation of the $\phi(r_{ij})$, $\rho(r_{ij})$, and $F(\bar{\rho})$.  Specifics of this calculation are provided in Appendex \ref{appendix:setfl}.

\subsection{Potential Formalisms}

A variety of potential formalisms have been explored.

\subsubsection{Density Function}


Mishin\cite{mishin2004_eam_NiAl} and Purja Pun and Mishin\cite{purjapun2009_eam_NiAl} use the following density function (\verb|density_mishin2004|):

\begin{equation}
\label{eq:density_mishin2004}
	\rho(r) = A_0 (r-r_0)^y e^{-\gamma (r-r_0)}(1+B_0 e^{-\gamma (r-r_0} + C0,
\end{equation}
Mishin sets $A_$ to be zero, but can arbitrarily set to any number since

which is mollified by the function $\psi(x) = x^4 / (1+x^4), such that the mollified density function is 

\begin{equation}
    \bar{\rho}(r) = \psi \left(\frac{r-r_c}{h}\right)\rho(r)
\end{equation}

\subsection{Pair Potential Functions}


The Ni pair potential takes the form of a generalized LJ potential as in Mishin\cite{mishin2004_eam_NiAl}

\begin{equation}
\label{eq:pair_mishin2004}
\end{equation}

\section{Generalized Stacking Fault in FCC}
The generalized stacking fault in FCC metals describe the the slip of \{111\} planes of the face centered cubic cell in the <112> direction, which represents the energy
In the early works of Frenkel and Mackenzie describe this motion as a function of the macroscopically measured shear modulus, the Burgers vector and the interplanar spacing of the \{111\} planes.

Rice[3] unstable stacking faults and stable stacking faults

Vitek\cite{vitek1966_gsf,vitek1968_gsf} develops the notion of the generalized stacking fault, which cannot be measured experimental except at a single point knows as the intrinsic stacking fault $\gamma_{ISF}$

Calculation of unstable stacking faults\cite{sun1990_eam_esf,sun1993_eam_esf,farkas1997_eam_usf} was done with EAM potentials, while DFT has been used for the calcualtion of the GSF curve [14,15,16]

Zimmerman takes the approach of creating a simulation of the FCC crystal oriented in the <111> direction, with the basal plane formed by the <112> and the <111> direction.  For the ease of creating the stacking fault, the <112> direction is chosen on the x-axis, and the <110> is chosen for the y-axis, and the z-axis on the <111>.  To create the stacking fault, the lower half remains fixed, while the upperhalf is displaced in the <112> direction in small increments.  Zimmerman uses 6000 atoms consisting of 30 {111} planes.  After lateral displacement, the atoms are allowed to relax laterally (in the <111> direction) except for three {111} planes at the top and bottom.  This method is used in [11-13]

To create a surrogate model, we use the atomic simulation enviornment (ASE)

\subsection{Density Functional Theory}



\subsection{Molecular Dynamics}

First generate a simulation cell of a representative fcc bulk with \hkl[1 1 2] in the x direction, \hkl[-1 1 0] in the y direction, and \hkl[-1 -1 1] in the z-direction, which was adapted from the script of Spear\cite{spear2012_lammps_gsf}.
