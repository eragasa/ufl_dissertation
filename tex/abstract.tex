% Write in only the text of your abstract, all the extra heading jargon is automatically taken care of
\begin{abstract}
  In this work, we analyze the limitations to the current approaches to the parameterization of empirical potentials.  Typically, a single objective function of the weighted sum of squared errors between the predictions of empirical potentials and reference targets is minimized.  The choice of weights is subjective and an appropriate choice cannot be realistically determined \emph{a priori}.  Scalar optimization identifies local concave solutions which are dependent upon the choice of weights and initial conditions.  This process requires the constant intervention of a skilled potential developer and is not amenable to an automated, algorithmic approach to potential development.

  These concerns are addressed by the presentation of a novel methodology.  The problem is recast as a multi-objective optimization problem where the fidelity of prediction with respect to each material property is treated independently.  This removes the need to communicate performance requirements \emph{a priori}.  In this way, the solution to parameterization expands from a single local solution to an ensemble of potentials through the use of the concept of Pareto optimality.  Since each potential is optimal in some sense, a final potential can then be selected \emph{a posteriori}, using the full knowledge of the performance tradeoffs between the candidate potentials.

  A solution scheme for this methodology is devised, combining Monte Carlo sampling with an iterative scheme to evolve a probability distribution function representing the epistemic uncertainty of Pareto optimal parameterizations.  A Bayesian-like inference process is used to periodically update the distribution function.

  A software library \emph{pypospack}, is developed to automate the simulation tasks, calculate material properties, and execute sampling based optimization routines.

  To demonstrate the flexibility and potency of this approach, applications to specific materials systems are presented.
\end{abstract}
