\chapter{PROBABILITY METHODS} \label{materials}

\section{Probability}
A random variable $X$ is a variable whose possible values are outcomes of a random phenemon.  As a function, a random variable is required to be measurable, which rules out pathological issues.

The underlying foundation of any probability  distribution is the sample space, which is the set of all probable outcomes denoted as $\Omega$.  The realization of an outcome is denoted $\omega \in \Omega$.

The events for the measure space are defined in such a way that a probability measure can be assigned.  This allows to assign probability measures on complex events to characterize groups of outcomes.  The collection of all such events is a $\sigma$-algebra $\mathcal{F}$ of subsets of $\Omega$.  Not every subset of the sample space $\Omega$ must be considered an event, the $\sigma$-algebra restricts $\mathcal{F}$ to subsets of $\Omega$ for which $\mathbb{P}$ can be asssigned.

The probability measure, $\mathbb{P}$, a function which maps $\mathbb{P}:\mathcal{F}\rightarrow[0,1]$.  A probability is a real number between zero and one.  Within this work we do not disguish the difference between impossible events which have probability zero, and probability-zero events which are not necessarily impossible.  Events of probability one is an event that happens almost surely, with almost total certainty.

The triplet $(\Omega,\mathcal{F},\mathbb{P})$ is the probability measure space.

\section{Random Variable}
A random variable has a probability distribution, which specificies the probability falls in.  Specifically, $X:\Omega\rightarrow\mathbb{R}$.  If a random variable $X:\Omega \rightarrow \mathbb{R}$ is defined on the probability space $(\Omega,\mathcal{F},\mathbb{P})$.  Then the probability of an event $A$ occuring is $\{\omega:X(\omega)=A\}$ which is denoted as $\mathbb{P}(X=A)$.

A probability density function for a random variable $X$ has a density $f_X$, where $f_X$ is a non-negative function:
\begin{equation}
  \mathbb{P}[a \leq X \leq b]=\int_{a}^{b}f_{X}(x)dx
\end{equation}

\subsection{Expectation}
If $X$ is a random variable with a finite number of outcomes, $\{x_1,x_2,...,x_k\}$ occurs with the probabilities $\{p_1,p_2,...p_k\}$.  Then the expectation of $X$ is defined as
\begin{equation}
  \mathbb{E}[X]=\sum_{i=1}^{k}x_i p_i
\end{equation}
If $X$ is a continuous random variable, then
\begin{equation}
  \mathbb{E}[X]=\int_{\Omega} X(\omega) d\mathbb{P}(\omega)
\end{equation}
If $X$ admits a density $f(x)$, then the expected value is defined as
\begin{equation}
  \mathbb{E}[X]=\int_{\mathbb{R}}x f_{X}(x) dx
\end{equation}

\section{Estimation of a Distribution}

\subsection{Non-parametric Estimation}
 Fusce eget tempus lectus, non porttitor tellus. Aliquam molestie sed urna quis convallis. Aenean nibh eros, aliquam non eros in, tempus lacinia justo. In magna sapien, blandit a faucibus ac, scelerisque nec purus. Praesent fermentum felis nec massa interdum, vel dapibus mi luctus. Cras id fringilla mauris. Ut molestie eros mi, ut hendrerit nulla tempor et. Pellentesque tortor quam, mattis a scelerisque nec, euismod et odio. Mauris rhoncus metus sit amet risus mattis, eu mattis sem interdum.

\subsection{Kernel Density Estimation}
Let $(x_1,x_2,...x_n)$ be a univariate and identically distributed sample drawn from some distribution with an unknown density $f$.  The goal is to estimate the shape of this function $f$.  The kernel density estimator is
\begin{equation}
  \hat{f}_h(x)=\frac{1}{n}\sum_{i=1}^{n}K_h(x-x_i)
    =\frac{1}{nh}\sum_{i=1}^{n}K\left(\frac{x-x_i}{h}\right)
\end{equation}
where $K$ is the kernel and $h>0$ is a smoothing parameter called the badwidth.  The kernel function satisfies the condition
\begin{equation}
  \int_{-\infty}^{+\infty}K(x)dx=1
\end{equation}
\subsubsection{Choice of kernels}
Popular kernels: Epanachnikov, Bi-weight, Triangular, Gaussian, Rectangular.

For the kernel method, we adopt the gaussian kernel $\phi$,
\begin{equation}
  \phi(x)=\frac{1}{\sqrt{2\pi\sigma}}\exp{\left(\frac{t^2}{2\sigma^2}\right)}
\end{equation}
\subsubsection{Selection of bandwidth parameters}
The chosen bandwith is important because it has a strong influenjce on the boundary of the density curve.  The curve boundary has poor smoothness quality when bandwidth takes small value; while as the increasing bandwidth, the smoothness improves, but the fitness of the curves becomes poor.  The accuracy of kernel estimation is dependent on suitable bandwidth.

\emph{Scott' Method}

\emph{Silverman Method}

\emph{Chiu Method}
\section{Bayesian Estimation}

\section{Sampling}

Requirements for kernel
\begin{table}[htbp]
   \caption{Sample size required to ensure relative mean square error at zero is
       less than $0.1$, when estimating a standard normal density using a normal
       kernel and the window width that minimize the mean square error loss at
       zero\cite{silverman1986_density_estimation}}
   \label{tab:kde_sample_req}
   \begin{tabularx}{6.5in}{XX}
     \hline
     Dimenstionality & Required Sample Size \\
     \hline
     1 & 4 \\
     2 & 19 \\
     3 & 67 \\
     4 & 223 \\
     5 & 768 \\
     6 & 2790 \\
     7 & 10700 \\
     8 & 43700 \\
     9 & 187000 \\
     10 & 842000 \\
     \hline
   \end{tabularx}
\end{table}

\section{Monte Carlo Methods}

Monte Carlo methods are a broad class of computational algorithms which are dependent upon random sampling to obtain numerical results.
The essential aspect of these approaches is to use randomness to solve problems which might be deterministic in principle.
Monte Carlo simulations sample from a probability distribution for each variable to produce an arbitrarily large number of possible outcomes, and  the results are analyzed to get probabilities of different outcomes occuring.
