\chapter{APPLICATIONS TO IONIC SYSTEMS}

Lewis Catlow potential\cite{lewis1985_buckingham}

Henkelman \emph{et al}\cite{henkelman2005_buckingham_MgO}

Lewis and Catlow\cite{lewis1985_buckingham} parameterized a wide range of oxide systems with a pairwise potential, between two atoms $i$ and $j$.
\begin{equation}
  V_{ij}(r_{ij})=\frac{Z_i Z_j}{r_{ij}}+A_{ij}\exp(-r_{ij}/\rho_{ij})-\frac{C_{ij}}{r_{ij}^6}
\end{equation}
The physics of ionic systems is modelled with three pairwise terms.
The first term is a Coulombic interaction dependent on the the point charges, $Z_i$ and $Z_j$.
The other terms is a Buckingham potential the repulsion between the two ions due to the Pauli exclusion principle with the exponential function combined with a term for a weak van der Waals interaction.
In the Lewis and Catlow parameterization (LC), the charges are assumed to have the formal charges of $\pm 2e$ on the magnesium and oxygen ions,respectively.
Additionally, the cations only interact with each other though the Coulomb interaction.

Two potentials unpublished Buckingham potentials were developed by Ball and Grimes (BG1 and BG2), but were used in the work of Henkelman \emph{et al}\cite{henkelman2005_buckingham_MgO}.
