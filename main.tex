%&latex
% UF Sample ETD Main Document Fall 2014
% Documenting the dvipdfmx/dvipdfm "File not Found" error
% Improved method of handling the single/multiple appendices issue
% Updated font calls to meet latest LaTeX standards

% Define Document Class to be used and options - Choose the option that meets your OS%
\documentclass[12pt,final,CPage]{ufthesis} %Use this line for Windows OS
%\documentclass[12pt,dvipdfmx,final,CPage]{ufthesis} %Use this line for Macintosh/Linux OS

% the command for this on OSX
% latexmk -pvc -pdf main
% latexmk -pvc -pdfps main

% Macintosh and Linux users - If you get a dvipdfm file not found error
% change dvipdfm to dvipdfmx here and in the packages.tex file graphicx and hyperref packages and
% compile using Latex, Latex, Bibtex, Latex, Latex, XeLaTeX - this usually fixes
% the problem NOTE: If you are including an Appendix Latex will complain about
% "something missing" press "r" followed by "Enter" and Tex will ignore the error.

%-------------------------------------C:\Program Files\MiKTeX 2.5\miktex----------------------------------%
% Preamble %

% Define Packages To be used and options %
\include{packages}
\def\UrlFont{\rmfamily} %use this line for Times New Roman
%\def\UrlFont{\sffamily} %use this line for Helvetica

\allowdisplaybreaks  % % This command allows equation arrays and similar environments
% % % to break across pages to improve text flow - use only if needed.

% Prevent figures, tables or algorithms from using a separate page or column alone
\renewcommand{\topfraction}{0.85}
\renewcommand{\textfraction}{0.1}
\renewcommand{\floatpagefraction}{0.75}

% *** Do not adjust lengths that control margins, column widths, etc. ***
% *** Do not use packages that alter fonts (such as pslatex).         ***
% There should be no need to do such things with IEEEtran.cls V1.6 and later.
% correct bad hyphenation here
%\hyphenation{op-tical net-works semi-C:\Program Files\MiKTeX 2.5\miktexconduc-tor}

%------------------------------------------%

% Extra commands or misc formatting such as page alignment or output paper-size commands

%\include{extraparameters}

%------------------------------------------%

% Set your personal and paper information
\SetFullName{Eugene Ragasa}%
\SetThesisType{Dissertation}%{Dissertation} %{Thesis}
\SetDegreeType{Doctor of Philosophy}% {Doctor of Philosophy} {Master of Science}
\SetGradMonth{July}%
\SetGradYear{2019}%
\SetDepartment{Materials Science and Engineering}%
\SetChair{Simon Phillpot}%
%\SetCochair{John W. Carver III}%uncomment this line and enter the name of your cochair inside the braces if you have one.
%If you have a cochair there two places in the ufthesis.cls file that will need to be uncommented as well
%In the "getting personal information" section about line 630
%And the "Abstract" Section around line 556
% Type your title here in all CAPS %
\SetTitle{MACHINE LEARNING TECHNIQUES FOR THE RATIONAL DESIGN OF ANALYTIC INTERATOMIC POTENTIALS}
% Define student-specific info (self-explanatory) %
%\include{userinfo}
%------------------------------------------%

% user defined commands in order to geC:\Program Files\MiKTeX 2.5\miktexnerate new commands, macros, and redefine default commands %
\include{usersetcommands}

%-------------------------------------------------------------------------------------------------------%

% Begin Main Part of Document %
\begin{document}
%% helps against most Overfull hboxes, from
%% dctt, Axel Reichert, Message-ID: <a84us0$plqcm$7@ID-30533.news.dfncis.de>
\tolerance 1414
\hbadness 1414
\emergencystretch 1.5em
\hfuzz 0.3pt
\widowpenalty=10000
\vfuzz \hfuzz
\raggedbottom    

 % % % % % % % % % % % % % % % % % % % % % % % % % % % % % % % % % % % % % %
 % Remember - You MUST get a .bst file that matches the Journal in your
 % field that you choose as your Reference example
 % NONE of these examples will satisfy the Graduate Editorial Office
 % if they don't match your Journal example!!!!
 % NOTE: If you use a numbered reference system and your references
 % are set in parentheses rather than brackets you need to select the
 % Natbib option "numbers sort and compress" in the packages.tex file
 % % % % % % % % % % % % % % % % % % % % % % % % % % % % % % % % % % % % % %


 %Note that the path separator is a forward slash NOT a back slash
 %Place YOUR .bst file in the bst folder and use that filename (without the .bst extension)
 % as your Bibliography Style file

%\bibliographystyle{bst/abbrv}
%\bibliographystyle{bst/abbrvnat}
%\bibliographystyle{bst/abbrvurl_uf}
%\bibliographystyle{bst/alphaurl_uf}
%\bibliographystyle{bst/apa-good}
%\bibliographystyle{bst/Chicago_Web}
%\bibliographystyle{bst/ecology_web}
\bibliographystyle{bst/IEEEtran}
%\bibliographystyle{bst/mla_web}
%\bibliographystyle{bst/mla-good}
%\bibliographystyle{bst/plainnat}
%\bibliographystyle{bst/plainurl_uf}
%\bibliographystyle{bst/Science_Web}
%\bibliographystyle{bst/uf_econ}
%\bibliographystyle{bst/uffull}
%\bibliographystyle{bst/ufinit}
%\bibliographystyle{bst/unsrtnat}
%\bibliographystyle{bst/unsrturl_uf}
%\bibliographystyle{bst/plain}
%\bibliographystyle{bst/ufinit}
%\bibliographystyle{bst/plainurl_uf}


%-----------------------------------------------------------------------%

\maketitle % % % % Creates the Title page from the information entered in userinfo.tex
\makecopyright

%------------------------------------------%

\dedication{% Add your text for the dedication here between the center tags
%\addvspace{4.25in}
%\begin{center}%\singlespacing
\vspace{-20 mm}
For my parents, Eugenio and Gloria, and my brother, Ferdinand.\\
%\end{center}
} % %Creates the dedication - if your dedication is more than a single line
% % % % % % % % % % % % % % % % % %you will need to reduce the vspace amount to keep the text centered verticlly
% % % % % % % % % % % % % % % % % %optional - comment or delete if you are not dedicating to anyone,

%------------------------------------------%

% Make sure to keep the text within the brackets and the output should turn out correct
\acknowledge{%
An acknowledgement of the people I would like to thank is necessarily incomplete, but starts with family.  My parents have always supported me throughout my life's adventures, fortunes and misfortunes.  My brother Ferdinand has been my best friend and biggest fan since his birth.

I chose to University of Florida to work with Simon Phillpot as my advisor, and I could not have asked for a better experience.  Susan Sinnott was influential early during this process, while Richard Hennig has been an invaluable resource.  I would also like the rest of the members of my committee.

I have worked with people that provided key insight into this project.  Jeffrey Larson and Stefan Wild at Argonne National Laboratory provided expertise with the scalability and architecture during early development phases of the optimization software.  Oliver Ruebel and Dmitriy Morozov at Lawrence Berkeley Labs introduced me to visualization and dimensionality-reduction techniques.  Stephen Foiles and Chris O'Brien at Sandia National Laboratories provided expertise and support throughout this process, while John Adam Stephens amd Michael Eldred provided insights into statistical and minimization concepts.  Finally, I would like to thank David Bai, who supervised me during an internship at Idaho National Labs.

The support of members of the Dr. Phillpot's, Dr. Hennig's, and Dr. Sinnotts research groups have provided friendship and support throughout the years.}
 % % % %Required - There is no requirement to acknowledge a particular person
% % % % % % % % % % % % % % % % %but you must acknowledge someone (funding source, committee chair, spouse)?

%------------------------------------------%

% This file includes the file which creates the table of contents %
\include{tex/TOC} %This file creates the Table of Contents, List of Figures, and List of Objects (if any)
% % % % % % % %delete or comment the file you want to remove

%------------------------------------------%

%%This is an optional file. A list of abbreviations is NOT even suggested.
%%Best practice is to define the item the first time it is used in the document

%\include{tex/abbreviations}


%------------------------------------------%
% This line adds the word CHAPTER to the TOC just before the listing of the chapter and subsections begins
\addtocontents{toc}{\protect\addvspace{10pt}\noindent{CHAPTER}\protect\hfill\par}{}% This extra line adds the word CHAPTER to the table of contents %
\phantomsection
% Write in only the text of your abstract, all the extra heading jargon is automatically taken care of
\begin{abstract}
  In this work, we analyze the limitations to the current approaches to the parameterization of empirical potentials.  Typically, a single objective function of the weighted sum of squared errors between the predictions of empirical potentials and reference targets is minimized.  The choice of weights is subjective and an appropriate choice cannot be realistically determined \emph{a priori}.  Scalar optimization identifies local concave solutions which are dependent upon the choice of weights and initial conditions.  This process requires the constant intervention of a skilled potential developer and is not amenable to an automated, algorithmic approach to potential development.

  These concerns are addressed by the presentation of a novel methodology.  The problem is recast as a multi-objective optimization problem where the fidelity of prediction with respect to each material property is treated independently.  This removes the need to communicate performance requirements \emph{a priori}.  In this way, the solution to parameterization expands from a single local solution to an ensemble of potentials through the use of the concept of Pareto optimality.  Since each potential is optimal in some sense, a final potential can then be selected \emph{a posteriori}, using the full knowledge of the performance tradeoffs between the candidate potentials.

  A solution scheme for this methodology is devised, combining Monte Carlo sampling with an iterative scheme to evolve a probability distribution function representing the epistemic uncertainty of Pareto optimal parameterizations.  A Bayesian-like inference process is used to periodically update the distribution function.

  A software library \emph{pypospack}, is developed to automate the simulation tasks, calculate material properties, and execute sampling based optimization routines.

  To demonstrate the flexibility and potency of this approach, applications to specific materials systems are presented.
\end{abstract}
 %The abstract is created using this file and userinfo.tex
% % % % % % % % % % %If you have a c-chair you must uncomment that line in userinfo.tex AND find the
% % % % % % % % % % %co-chair lines in ufthesis.cls and un-comment those as well

%-----------------------------------------------------------------------%

% This section encompasses the main body of the paper from all the content through to the biographical sketch

% Chapters to be included (more can be added by creating a new chapter#.tex %
% file and then implementing the \inlcude{chapter#.tex} command as seen below %
\chapter{INTRODUCTION}\label{intro}

As compututational power has increased so has the size and complexity of the simulations.

The use of analytical empirical potentials has been part of computational materials science from the start, including problems with lattice dynamics, molecular dynamics, and other charges.

Despite the promise of molecular dynamics, the difficulty in developing interatomic potentials leads to long developments when developing empirical potentials due to problems in determining an optimal parameterization.

This work presents and emergent frameworkk for the automated development of potentials based upon sampling from a distribution and evolution of that distribution so that the final distribution represents the set of parameterizations which can be described in a way as efficient.

We present an novel framework for the automated development of potentials.
\section{}
\section{Modernizing Analytical Potential Development}

The modernization of analytical potential development needs to provide: (1) clearly define the problem in it's most general terms in what maybe, at times, in a more rigorous mathematical exposition than what is normally presented, (2) identify the problems with existing methodologies by bringing in terminology and notation from different fields to provide a more general framework for the problem of potential development, (3) provide a baseline implementation of this framework, which while begin largely pedagogical provides results which are analytical, transparent, and robust, and (4) provide an automation framework upon which to do the work.

This work starts with an appropriate level of introduction to the major toolsets in computational materials science in Chapter 2, where we start from a quantum mechanical approaches to solutions to the potential energy surface, a function which maps the configuration space of atomic descriptors onto energies, focusing specifically on Density Functional Theory, which is often used to augment (and often completely replaces) experimental observations.  From here, we look to two tools where analytical potentials are most often used (1) molecular dynamic simulations, and (2) lattice dynamics simulations.  For completeness, a discussion of higher level tools and how these tools are necessary to bridge scale.

In the third chapter, the empirical interatomic potential is described as approximation as the potential energy surface.
In the traditional approach to potential development, an objective function described as the weighted sum of square differences between the analytical potential predicted values for a finite set of material properties and their respective target values, a set property structure relationships which are known as the fitting database.
This objective function in minimized usually through either constrained or unconstrained optimization techniques dependent upon a quadratic programming algorithm dependent upon first-order derivatives for optimization.  In this chapter, we show that this method for even identifying the global minimum due to issues with (1) regularization, (2) necessary conditions for a solution are likely not satisfied (Karush-Kuhn-Tucker conditions), and at the very least impossible to prove.
As a result, quadratic programming techniques are likely unsuitable, except when an initial guess to the target parameterization is already known.
This problem is hinted at in discussions in literature in discussions in a move from local optimization vs global optimization algorithms.
However, more damning is that even when these conditions are met, the optimal paramaterization is dependent upon a vector of weights, which uniquely determines the parameterization.
That is, potential development is inherently a subjective process dependent upon the preferences of the user, which largely cannot be determine with certainty at the beginning of the process.

The ramification of chapter 3 is that potential optimization is NP-hard when preferences are fixed is that probing acceptable potentials by a mechanism of varying weights is computationally brutal, and reevaluates areas in parameter space which have already been solved.
Chapter 4 starts a framework for sampling in parameter space beginning with statistical concepts and links it to approaches in machine learning with genetic algorithms.

This work first starts with what is hoped an appropriate level of theory required to understand the necessities for automating the problem.
In much of the literature, there is rather extensive discussion of techniques and applications of techniques, but the supporting justification and rationalization of these techniques is necessarily short due to the format of publications.
Even when discussing older approaches, appropriate review articles have been identified from different fields into order to provide insight to the problem of potential development.


In the development of a solid methodology and a description of potential development automation, it is necessary first describe the major computational tools used within the framework of machine learning.  Specifically, the tools of Density Functional Theory, Molecular Dynamics, and Lattice Dynamics are briefly discussed to provide a sense of what is common and what is different between these computational approaches.  This is covered in chapter 2.

Chapter 3 describes which to is referred to the the traditional approach to potential development.  Potential development is normally described as a quadratic programming optimization problem, applied to a description of a problem which it is not normally suited for.  However, this problem has never really been discussed clearly and explains why more recently many potential developers have moved from potential development



ather than being competitive, \emph{ab initio} computational techniques are often integrated into potential development
Despite promising work with the development of machine learning potentials, which drastically increases the size of the functional space to describe a potential energy surface, these approaches require large amounts of computational resources to develop the requisite fitting datasets.
This data is necessary for this data hungry approach as into ensure that the problem is not in functionally underdetermined due to large degrees of freedom associated with the functional form.
Moreover, neural networks models themselves are NP-hard problems which maps the current configuration of the neural net onto potential configuration space, requiring an extraordinary computational effort to determine the most effective parameter optimization.
It should not be confused that neural networks are not an optimization solution, but actually an optimization problem, which are aided by different machine learning techniques which are dependent upon sampling

This work takes a contrary, but complimentary approach by addressing the concerns readily brought force by this new avenue of study.
The promise of machine learning potentials is provide \emph{ab initio} level accuracy at a fraction of the cost of \emph{ab initio} techniques.

It is the opinion of the author, that the development of analytical empirical potentials has largely been retarded by the lack of standard tools and analytical frameworks.
Despite the ubiquity of molecular dynamics
Instead of looking to develop the optimal parameterization which is expected to replicate a large range of values.
We accept that the development of an potential largely involves a decision of tradeoffs determined by the potential developer, these expression of preferences is inherently subjective.

Instead of developing a large monolithic application, which is difficult to extend, modify, and implement.  Pypospack is conceived largely as software library which defines structure property relationships, how the structure property relationships are calculated, the process management of molecular dynamic simulations, parallel sampling, and potential selection.
 % INTRODUCTION
\chapter{THE MANY BODY PROBLEM AND ATOMISTIC METHODS}
\hfuzz=20pt
\vfuzz=20pt
\hbadness=20000
\vbadness=\maxdimen
The properties of a system may be obtained by solving the quantum mechanical (QM) wave equation which governs the system dynamics.  For non-relativistic system, this equation is the Schrodinger equation.  For all but the simplest systems, this approach in an impossible task in practice; the resulting many body problem has only been solved for a limited number of system.  Within this chapter we outline the many body problem and its intractibility before considering the Hohenberg-Kohn-Sham formulation of density functional theory (DFT).  This reformulates quantum mechanics, using electron density as the fundamental quantity to solve, rather than the many-electron wavefuction.  This takes the $N$-body problem and recasts it into $N$ single-body problems; which is a dramatic simplification.

We then approach higher order models which reduces computational intensity by looking at classical empirical potentials and their role in both molecular dynamics and lattice dynamics.

% http://cmt.dur.ac.uk/sjc/thesis_prt/node21.html
\section{The Many-Body Problem}

The Hamiltonian for a real material is defined by the presence of interacting nuclei and electrons:
\begin{equation}
	\mathcal{H} = \sum_i \frac{P_i}{2M_i}
	    + \sum_\alpha \frac{p_\alpha}{2m}
	    + \frac{1}{2} \sum_{ij} \frac{Z_i Z_j e^2}{r_{ij}}
	    + \frac{1}{2} \sum_{\alpha\beta} \frac{e^2}{r_{\alpha\beta}}
	    - \sum_{i\alpha} \frac{Z_i e^2}{r_{i\alpha}}
\end{equation}
The first terms are kinetic energy terms, the latter terms are the nuclei-nuclei, electron-electron, and nuclei-electron interactions.
Ideally, the Schr\"{o}dinger equation, $H\Psi=E\Psi$ could be solved providing the total wavefunction $\Psi(\bm{r}_i,\bm{r_\alpha})$ and associated eigenvalues $E(\Psi)$.
Except for the simplest of systems, this approach is impossible computationally.

The Born-Oppenheimer approximation\cite{born1927_bo} simplifes calculation; the kinetic energy is ignored since the heavy nulclei move more slowly than electrons.  For the remaining interaction terms of the Hamiltonian, the nuclear positions are clamped in space, the electron-nuclei interactions are not removed, since the electrons are still influenced by the Coulomb potential of the nuclei.  This allows us to factor the wavefunction as
\begin{equation}
	\Psi(\bm{R}_i,\bm{r}_\alpha) = \Xi(\bm{R}_i)\Phi(\bm{r}_\alpha;\bm{R}_i),
\end{equation}
where $\Xi(\bm{R}_i)$ describes the nuclei, and $\Phi(\bm{r}_\alpha;\bm{R}_i)$ describes the electrons parameterized by the clamped position of $\bm{R}_i$.  In turn, the Hamiltonian is solveable as two Schr\"{o}dinger's equations.  The first equation contains the electronic degrees of freedom.
\begin{equation}
\label{eq:BO_electronic}
     H_{e}\Phi(\bm{r}_\alpha;\bm{R}_i)=U(\bm{R}_i)\Phi(\bm{r}_\alpha;\bm{R}_i)
\end{equation}
where
\begin{equation}
	H_e = \sum_\alpha \frac{p_\alpha}{2m}
	      + \frac{1}{2} \sum_{ij} \frac{Z_i Z_j e^2}{r_{ij}}
	      + \frac{1}{2} \sum_{\alpha\beta} \frac{e^2}{r_{\alpha\beta}}
	      - \sum_{i\alpha} \frac{Z_i e^2}{r_{i\alpha}}
\end{equation}
Eqn. \ref{eq:BO_electronic} gives the energy $U(\bm{R}_i)$ which depends on the clamped coodinates of $\bm{R}_i$.  The electronic effects are contained in $U(\bm{R}_i)$ and are incorporated into the second equation which the motion of the nuclei
\begin{equation}
\label{eq:BO_nuclei}
    H_n\Xi(\bm{R}_i)=E\Xi(\bm{R}_i)
\end{equation}
where
\begin{equation}
\label{eq:H_n}
    H_n = \sum_i \frac{P_i}{2m_i} + U(\bm{R_i})
\end{equation}
Direct solution of the Schr{\"o}dinger equation for the electrons in a molecule is demanding because of the Coulomb repulsion between them.

\section{Hartree and Hartree Fock Methods}

The Hartree method\cite{hartree1928_hartree,slater1928_hartree,gaunt1928_hartree} applies the variational principle to a product \emph{ansatz} of orthogonal wave functions, known as the Hartree product, to represent the ground state function:
\begin{equation}
\label{eq:hartree}
	\Psi(\bm{x}_1,\bm{x}_2,...,\bm{x}_N)=\psi_1(\bm{x}_1)\psi_2(\bm{x}_2)...\psi_n(\bm{x}_n)
\end{equation}
where $\bm{x_i}$ is the set of space-spin coordinates, $\bm{x}_i=\{\bm{r}_i,\omega\}$ with $\omega \in \{\alpha,\beta\}$ being a spin-coordinate.
However, since electrons are fermions they must follow the Pauli exclusion principle, and must be anti-symmetric under exchange of any of the space-spin coordinates.  The Hartree product does not satisfy the anti-symmetry principle\cite{slater1930_antisymmetry,fock1930_antisymmetry}, which can be demonstrated with a two particle system.  For a two-particle system, the anti-symmetric property can be described as
\begin{equation}
\label{eq:antisymmetry}
	\Psi(\bm{x}_1,\bm{x}_2) = -\Psi(\bm{x}_2,\bm{x}_1)
\end{equation}
which Equation \ref{eq:hartree} clearly fails.

The Hartree-Fock method often assumes that the exact, $N$-body wave function of the system can be approximated by a single Slater determinant from a system of electrons.  By invoking the variational method, one can derive a set of $N$-coupled equations for the $N$ spin orbitals.  A solution of these equations yields the Hartree-Fock wave function and energy of the system.

\section{Density Functional Theory}

Density Functional Theory (DFT) is a simplification of the many body electron function by relating the wave function of the system of interest to the electron density of the system, where theoretical underpinning are established by the Hohenberg-Kohn theorems\cite{hohenberg1964_dft}.

The first Hohenberg-Kohn theorem establishes that the external potential $v_{ext}(\bm{r})$, and hence the total energy, is a unique functional of the electron density $n{\bm{r}}$.

Here, the energy functional $E[n(\bm{r})]$ is written in terms of the external potential $V_{ext}(r)$,
\begin{equation}
	\label{eq:hk_functional}
	E[n(\bm{r})] = F_{HK}[n(\bm{r})]+\int V_{ext}(\bm{r})n(\bm{r})d\bm{r}
\end{equation}
where $F_{HK}[n(\bm{r})]$ is unknown, but otherwise a function of the electron density $n(\bm{r})$.  Correspondingly, a Hamiltonian can be written such that electron wave function $\Psi_0$ also gives the ground state $E[n(\bm{r})]=\bra{\Psi}\hat{H}\ket{\Psi}$, where $n_0(\bm{r})$ is the ground state electron density.  This Hamiltonian is
\begin{equation}
	\hat{H}
	= \hat{F} + \hat{V}_{ext}
	= [\hat{T} + \hat{V}_{ee}] + \hat{V}_{ext}
\end{equation}
Here, the electron operator $\hat{F}$ can be decomposed into  $\hat{T}$ is the kinetic energy operator and  $\hat{V}_{ee}$ is the electron-electron operator.
$\hat{V}_ext$ is the external potential.
For all $N$ electron systems, $\hat{F}$ is that same, so $\hat{H}$ is completely defined by the number of electrons $N$ and the external potential $v_{ext}(\bm{r})$.

If are two different potentials $v_{ext,1}(\bm{r})$ and $v_{ext,2}(\bm{r})$, that give rise to the same density $n_0(\bm{r})$.  The associated Hamiltonians $H_1$ and $H_2$ will have different groundstate wavefunctions, ($\Psi_1$ and $\Psi_2$) and ground state energies ($E_1^0$ and $E_2^0$).  The application of the variational principle with Equation \ref{eq:hk_functional} leads to the inequality
\begin{align}
	E_1^0 < \bra{\Psi_2}\hat{H}_1\ket{\Psi_2}
	      &= \bra{\Psi_2}\hat{H}_2\ket{\Psi_2} \\
	      &= E_2^0
				  + \int
					  n_0(\bm{r})[v_{ext,1}(\bm{r})
						            -v_{ext,2}(\bm{r})]
					d\bm{r}
\end{align}
An equivalent inequality can be made with the subscript exchanged, leads to an apparent contradiction.  As a result, the ground state density $n_0(\bm{r})$ uniquely determines the external potential $v_{ext}{(\bm{r})}$.  The electrons determine the positions of the nuclei in the system, and also all groundstate electronic properties because $v_{ext}(\bm{r})$ and $N$ completely define $\hat{H}$.

The second Hohenberg-Kohn theorem defines an energy functional and proves that the correct ground electron density also minimizes this energy functional.  This establishes a one-to-one correspondence between between the ground state electron density $n_0(\bm{r})$ and the corresponding ground-state wavefunction $\Psi_0$.


\section{Kohn-Sham Equations}

\section{The Exchange Correlation Term}

In Kohn-Sham DFT, only the exchange-correlation energy, $E_{XC}$, as a functional of the electron spin densities $n(\bm{r})$ must be approximated.
The local density approximation is discussed by Kohn and Sham\cite{kohn1965_dft} in the introduction of the Kohn-Sham equations.
In the LDA, the exchange energy per particle in each spatial point is taken as the exchange energy per particle from a uniform electron gas with a density equivalent to the density in this same point\cite{ceperley1980_lda}.
Later LDA take similar approaches, \cite{vosko1980_lda_vwm, perdew1981_lda_pz, perdew1992_lda_pw}.

\begin{equation}
	E_{xc}^{LDA}[n]=\int n(\bm{r})e_{xc}^{hom}(n(\bm{r}))d\tau
\end{equation}
The generalized gradient approximation(GGA)\cite{langreth1983_gga_1,becke1988_gga_2} introduces a gradient correction
\begin{equation}
	E_{xc}^{GGA}[n]=\int n e_x^{LDA} F_{xc}^{GGA}(n,s)_{n=n(\bm{r})}d\tau,
\end{equation}
where
\begin{equation}
	s =\left.
	       \frac{\lvert\nabla n\rvert}{2k_F n}
	   \right\rvert_{n=n(\bm{r})}
\end{equation}

\section{Molecular Dynamics}
Molecular dynanmics (MD) is a simulation approach where the time evolution of aset of interacting atoms is followed by numerically solving their equations of motion.  In MD, the behavior of atoms follow Newtonian mechanics:
\begin{equation}
	  \label{eq:newton_eom}
    \bm{M}\frac{d^2\bm{R}(t)}{dt^2} = \bm{F}(\bm{R}(t))
\end{equation}
where $t$ is time, $\bm{R}(t) = (\bm{r}_1(t),...,\bm{r}_N(t))$ represents the forces on the $N$ atoms, and $\bm{M}$ is the mass matrix with the mass of each atom $i$ set across across the diagonal.
The total energy is conserved, even if the kinetic energy and potential energy can change dynamically.

The potential $V$ maps interatomic configurations $\bm{R}$ to provide the potential energy of a system $E_P$.  To evolve a system of atoms, the force $\bm{F}$ is given by the gradient of the potential $F(\bm{r}(t))=-\nabla V(\bm{R}(t))$, which can then be substituted into Equation \ref{eq:newton_eom}.


\section{Numerical Integration}
A dynamical simulation computes atomic position of each atom $i$ as a function of time given their initial positions $\bm{r}_i(t=0)$ and velocities $\bm{v}_i(t=0)$.  Since Newton's equation of motion is 2nd order differential equation, an initial conditions needs to specify both positions and velocities of all atoms at the initial condition.

To solve the equation of motion computationally, time is typically descretized uniformlyby incrementing $t$ by $\Delta t$.  A naive implementation of an integration scheme would use the forward Euler algoirthm which is a truncated Taylor series expansion,
\begin{equation}
	\bm{r}(t+\Delta t)
	=
	\bm{r}(t)
	+ \frac{d \bm{r}(t)}{dt} \Delta t
	+ \frac{1}{2} \frac{d^2 \bm{r}(t)}{dt^2} \Delta t^2
\end{equation}
This is approach is not time-reversible, does not conserve volume in phase space, and suffers from energy drift\cite{allen1987_md}.

Instead, the Verlet algorithm begins by approximating
\begin{equation}
	\frac{d^2 r(t)}
	     {dt^2}
	= \frac{\bm{r}(t+\Delta t) - 2\bm{r}(t) + \bm{r}(t-\Delta t)}
	       {\Delta t^2}
\end{equation}
Thus,
\begin{equation}
  \frac{\bm{r}(t+\Delta t) - 2\bm{r}(t) + \bm{r}(t-\Delta t)}
	     {\Delta t^2}
	=
	- \frac{1}{m} \frac{dU(\bm{r}(t))}{d\bm{r}}
\end{equation}
\begin{equation}
	\bm{r}(t+\Delta t)
	     = 2\bm{r}(t) - \bm{r}(t-\Delta t) - \Delta t \frac{1}{m} \frac{dU(\bm{r}(t))}{d\bm{r}}
\end{equation}

\section{Empirical Interatomic Potentials}
%The interatomic potential $U(\bm{R}_i)$ derived from the Born-Oppenheimer approximation is derived from a quantum-mechanical perspective.
The computational cost of \emph{ab initio} such as DFT can provide accurate structural energies and forces.  However, the computational cost to compute $V(\bm{R}_i)$  makes the scientific inquiry of systems limited to short simulation times and small number of atoms.
An empirical interatomic potential $\hat{V}(\bm{R}_i;\bm{\theta})$ is an analytical function parameterized by $\bm{\theta}=(\theta_1,...,\theta_n)$ which is meant to approximate $V(\bm{R}_i)$, and provide computational efficiency by eliminating electronic structure calculations.

Over the last few decades, a large number of potentials have been developed to descibe various bonding types and environments.
To take representative examples, the Lennard Jones was developed for the van der Waals interactions of noble gases, pair potentials such as the Buckingham potential can be used for ionic solids, the embedded atom model (EAM) was developed for metallic systems, the Assisted Model Building with Energy Refinement (AMBER) for biomolecules, the Tersoff potential for covalently bonded materials.
To deal with bonding and chemical environments for heterogenous materials like metal/metal oxide interfaces have led to extensions such as MEAM, REBO, COMB, and ReaxFF.

The topic of empirical potentials and their development is covered in detail in Chapter \ref{ch:potential_development}.

\section{Energy Minimization}

Function optimization is a calculation that pervades much of numerical analysis.  In the context of material systems, the function to be optimized (in this case minimization) is the energy of a system.  The energy landscape of material may posseses many minima, or conformational substates.  The goal of energy minimization is to find the local energy minimum, which is the atomic configuration corresponding with the bottom of the energy well for that phase.

Physically, energy minimization corresponds to an instantaneous freezing of the system; a static structure in which the forces between atoms have been eliminated at $0 K$

Starting with different configuration of atoms, may lead to different energies.  Each of these energies correspond to a different phase of the material.  The lowest energy phase, the energy at this local minimum maybe much higher than the energy of the global minimum.  The atomic configuration of the global minimum is the ground state configuration.  Other configurations are non-equilibrium phases of the material system at $0$ K which corresponds to the local minimum of that phase.

\section{Thermodynamic Ensembles}
  The statistical ensembles for molecular dynamics simulations can be defined by examination of the ideal gas law:
  \begin{equation}
  \label{eq:ideal_gas_law}
    PV = N k_B T
  \end{equation}

  The left hand of the equation is the potential energy, $E_P$, which is balanced by the right hand of the equation, which is the kinetic energy $E_K$ of the system.  Pressure ($P$), volume ($V$), and temperature ($T$), and the number of atoms ($N$) are the state variable of the system.  Each of the state variables is definable in a molecular dynamics simulation.

  The simplest state variables to obtain are the number of atoms and volume.  These are defined by the number of atoms in the simulation cell, and the volume of that unit cell.  Calculation of temperature is discussed in its regulation in Section \ref{sec:nvt_ensemble} through the use of thermostat algorithms.  Likewise, the calculation of pressure is presented with barostat algorithms in Section \ref{sec:npt_ensemble}.

  In MD simulations, it is often necessary or desireable to study a system under the conditions of constant pressure or constant temperature.  The ensembles most typically used for molecular dynamics, are the microcanonical (NVE), canonical (NPT), and isothermal-isobaric (NPT).

  In this section, the NVE is first discussed, and the other ensembles are discussed as modification of the equations of motion for the NVE ensemble.  The purpose of this section is to provide a broad overview of the topic.

Simulations of the canonical ensemble introduces the regulation of temperature.  Problems with temperature rescaling approaches are discussed.  Two types of thermostats are discussed ones involving the stochastic approach of Langevin dynamics, and the Nose-Hoover thermostat\cite{hoover1985_npt} which modifies the equations of motion.

The isothermal-isobaric (NPT) introduces regulation of pressure.  The discussion here is limited to the approach of Parrinello and Rahman\cite{parrinello1981_barostat}.

\subsection{NVE}
Molecular dynamics can be expressed in Hamiltonian form to specifically express conserved quantities\cite{allen1987_md}, which can then be used to derive the equations of motion,
\begin{equation}
	\frac{d \bm{r}}{dt}=\frac{\partial H(\bm{r},\bm{p}}){\partial \bm{p}},
	\frac{d \bm{p}}{dt}=\frac{\partial H(\bm{r},\bm{p}}){\partial \bm{r}}
\end{equation}

Integration of the classical equations of motion (by Verlet) for a system leads as $t \rightarrow \infty$ to a trajectory mapping a microcanonical ensemble of microstates.  The microcanonic or isenthalpic ensemble.  Without regulation of temperature or volume, this is result which molecular dynamics simulation provides.

  The independent variables of the microcanonical enseble ($N$, $V$, and $E$) are extensive, which means they should be conserved during the simulation,
\begin{equation}
	\frac{dH}{dt}=\frac{\partial H(\bm{r},\bm{p})}{\partial \bm{r}} \frac{d\bm{r}}{dt}
							 +\frac{\partial H(\bm{r},\bm{p})}{\partial \bm{p}} \frac{d\bm{p}}{dt}=0
\end{equation}

  The other state variables are not conserved ($P$, $T$, and the chemical potential $\mu$), and should fluctuate around well-defined average values.


\subsection{NVT and Thermostats}
\label{sec:nvt_ensemble}
The microcanonical ensemble where the energy is kept constant is not often used because experiments where energy is kept constant; this does not correspond with conditions under which most experiments are performed.  Regulation of the temperature is motivated by the need to match experimental conditions, study temperature-dependent processes,

A canonical ensemble (NVT) is the statistical ensemble that represents the possible states of a mechanical system in thermal equilibrium with a heat bath at a fixed temperature.  The principal thermodynamic variable of the canonical ensemble is temperature.  The ensemble also depends on the number of particles in the system, and the system's volume.

A modification of the Newtonian equations of motion to generate a thermodynamic ensemble at a constant temperature is a thermostat.  Thermostats in molecular dynamics provide a mechanism of regulating the simulation temperature $T_0$.  In NVT simulation, themostating modifies the translation velocity of the particles.

A short discussion of thermostats is provided here.  For a more detailed description, see H\"{u}nenberger\cite{hunenberger2005_thermostats}, Allen and Tildesley \cite{allen1987_md}, and Tadmor and Miller \cite{tadmor2011_md}

The use of a thermostat requires the definition of an instaneous temperature which is compared to the temperature of the heat bath which the system is coupled to $T_0$.

From the equipartition theorem provides the quantitative prediction, such as the average kinetic and potential energies of a system.  The equipartition theorem gives the average values of the individual components of energy.  The average kinetic energy $K$ of a system is related to its macroscopic temperature $T$ through
\begin{equation}
  K
  =
  \langle \mathcal{K} \rangle
  =
  \frac{1}
       {2}
  k_B N_{\mathrm{df}} T
\end{equation}
where $N_{\mathrm{df}}$ is the internal degrees of freedom in the system, $\mathcal{K}$ is the instantaneous internal kinetic energy, and the angular brackets indicate either a time or statistical average over the entire ensemble.
The number of internal degrees of freedom is calculated as $N_{\mathrm{df}} = 3 N - N_c$, for a system with $N$ number of atoms and $N_c$ geometrical constraints on the system.
The instanteous temperature $\mathcal{T}$ is
\begin{equation}
  \mathcal{T}
  =
  \frac{2}
       {k_B N_{\mathrm{df}}}
  \mathcal{K},
\end{equation}
then the average temperature $<\mathcal{T}>$ is identical to the macroscopic temperature $T$.  Since the velocity $\bm{v}_i$ of each particle $i$ is obtained from Verlet integration, the instantaneous kinetic energy $mathcal{K}$ can be replaced by the particles' momenta $\bm{p}_i$,
\begin{equation}
  \label{eq:T_kinetic_energy}
  T
  = \mathcal{T}
  = \frac{2}
         {k_B N_{\mathrm{df}}}
    \sum_{i=1}^N \frac{|\bm{p}_i|^2}{2 qm_i}
  = \frac{2}
         {k_B N_{\mathrm{df}}}
         \sum_{i=1}^N {1}{2}m|\bm{v}|^2
\end{equation}

\begin{equation}
  E_k = <K> = \sum_{i=1}^N {1}{2}m|\bm{v}|^2
\end{equation}

\subsubsection{Velocity Rescaling}
An obvious way to regulate the temperature is velocity scaling\cite{tadmor2011_md}.  Denote the temperature at time $t$ as $T(t)$ and combine with Equation \ref{eq:T_kinetic_energy}
\begin{align}
  \Delta T(t)
  &= T(t) - T(t-1)
  &= {1}{2}
     \sum_{i=1}^N 2 \frac{m_i (\lambda \bm{v}_i)^2}
                         {N_{\mathrm{df}} k_B}
     -
     {1}{2}
     \sum_{i=1}^N 2 \frac{m_i \bm{v}_i^2}
                         {N_{\mathrm{df}} k_B}
  &= (\lambda^2 - 1) T(t)
\end{align}
where $\lambda = \sqrt{T_0/T_0}$.  Thus, the simplest way to control the temperature is to multiply the velocities at each time step by $\lambda$.

This algorithm keeps the instantaneous temperature set to exactly $T_0$.  Since this approach does not allow fluctuations in temperature, it is inconsistent with the statistical mechanics of the canonical ensemble, where we know that the kinetic energy is not constant even when a system is at thermal equilbrium with its heat bath.

Here, we discuss three thermostats for the canonical ensemble: (1) the Berendsen thermostat, (2) the Lagevin thermostat, and the (3) Nos\'e-Hoover thermostat.

\subsubsection{Berendsen thermostat}

The Berendsen thermostat\cite{berendsen1984_nvt} is an algorithm to re-scale the velocities of the particles in molecular dynamics simulations to control the simulation temperature.  Here the temperature of the system is corrected such the deviation exponentially decays with some time constant $\bm{\tau}$
\begin{equation}
  \label{eq:npt_berendsen_1}
  \frac{dT(t)}{dt} = \frac{1}{\tau} (T_0 - T(t)).
\end{equation}
The coupling parameter $\bm{\tau}$ determines how tighly the bath and the system are coupled together.  The change of motion between successive time steps is:
\begin{equation}
  \label{eq:npt_berendsen_2}
  \Delta T(t) = \frac{1}{\tau} (T_0 - T(t)) \Delta t
\end{equation}
This thermostat suppresses fluctuations in the kinetic energy of the system and therefore cannot produce trajectories consistent with the cannonical ensemble.

\subsubsection{Langevin thermostat}

In the Langevin thermostat\cite{allen1987_md}, the regulation of the temperature is maintained by transforming Newton's equation of motion into a stochastic differential equation.
The Langevin thermostat incorporates random forces into the equations of motion.  These random forces models the effect of the heat bath.

The original Langevin equation\cite{langevin1908_equation} described Brownian motion,
\begin{equation}
  \label{eq:langevin}
  m_i \frac{d^2\bm{r}_i}{dt^2}
  =
  -\lambda \frac{d\bm{r}}{dt} + \bm{\nu}(t).
\end{equation}
The force acting on the article is written as a sum of a viscous force proportional to the particle's velocity (from Stokes' law), and a stochastic term $\bm{\nu}(t)$, which has a Gaussian probability distribution is correlation function usually expressed as
\begin{equation}
  \label{eq:langevin_nu}
  <\nu_i(t) \nu_j(t')> = 2 \lambda k_B T_0 \delta_{ij} \delta(t-t'),
\end{equation}
where $\nu_i$ is the $i$th component on the vector $\nu_i(t)$.  Equation \ref{eq:langevin_nu} is described in terms of an autocorrelation properties to mathemtically describe Brownian motion.

A description from Brownian motion from probability theory notation\cite{karatzas1991_sde} of Brownian motion is more clear than Equation \ref{eq:langevin_nu}.Denote $B(t)$ as a standard brownian motion, then $B(t)$ as independent increments with respect to time $t$.  For every $t > 0$, the future increments $B(t+\Delta t) - B(t)$, for $u \geq 0$, are independent of the past values $W_s$, $s < t$.  Brownian motion has Gaussian increments $B_{t+\delta t}-B_{t}$ with mean $0$ and variance $\delta t$; $B_{t+\delta t}-B_{t}~N(0,\delta t)$.
Finally, $B(t)$ has continuous paths and is continuous in $t$.
Since $B(t)$ is nowhere differentiable, the stochastic differential $dB(t)$ informally denotes the corresponding It\^{o} integral
\begin{equation}
  B(t+\delta t) - B(t) = \int_t^{t+\delta{t}} dB(\tau),
\end{equation}
and then $d\bm{B}(t)/dt$ denotes a white noise process.

With this description, $\bm{nu}(t)$ is based on the evolution of a standard Brownian motion,
\begin{equation}
  \bm{\nu}(t) = \sigma_i \frac{d\bm{B}(t)}{dt}
\end{equation} rescaled by
$\sigma_i^2 = 2 m_i \lambda_i k_B T_0$, and $\bm{B}(t) \in \mathbb{3}$ with $B_i$ independent from $B_j$ for $i \neq j$.  Then $\bm{nu}(t)$ models the heat flow from a solvent at a constant temperature $T_0$.  The necessary properties implied by \ref{eq:langevin_nu} follow immediately from the definition of standard Brownian motion.

The Langevin thermostat\cite{grest1986_langevin} was originally developed to explain the the interaction of the solvent (i.e. the heat bath) on the solute (i.e. the simulated system) and has two levels of interaction.  First, the atoms interact with each other through the interatomic potential.  Second, the atoms are in constant collision with some medium that constitutes the heat bath with the assumption that interaction with the heat bath is more frequent exert smaller forces than the interatomic iteraction.  This allows to decouple interatomic interactions from the interaction from the heat bath.  Adding the terms from Equation \ref{eq:langevin} to the Newton's equation of motion, we get the new governing equation for atom $i$.
\begin{equation}
  \label{eq:langevin_eos}
  \frac{d^2\bm{r}_i}{dt^2}
  =
  \frac{1}{m_i} F_i
  - \frac{\gamma_i}{m_i} \frac{d \bm{r}_i}{dt}
  + \frac{1}{m_i} \bm{\nu}_i(t)
\end{equation}
where $\gamma_i$ is now dampening constant associated with atom $i$.

In the Langevin thermostat, the choice $\gamma$ becomes arbitrary.  The choice of too small of $\gamma$ leads to slow equilbrium, while a large value of the dampening constant leads to the stochastic term dominating the equations of motion.

\section{Nos\'e-Hoover thermostat}
% https://www2.mpip-mainz.mpg.de/~andrienk/journal_club/thermostats.pdf
The method to specify the temperature in a molecular dynamics simulation was proposed by Nos\'e\cite{nose1984_npt_1,nose1984_npt_2}.  In this approach, Nose proposed an extended form of the Hamiltonian which introduces a fictitious particle, which couples the system to the heath bath.  The Nose equations of motion are smooth, deterministic, and time-reversible, but stretched the timescale making them impractical for simulation purposes.  These equations were refined by Hoover\cite{hoover1985_npt} to reformulate Nos\'e equations in term of real system variables.  This formulation is known as as the Nose-Hoover thermostat.

The Langrangian equations of motion for the Nos\'e-Hoover thermostat are
\begin{equation}
  \label{eq:nose_hoover_1}
  \frac{d^2\bm{r]}}
       {dt^2}
  =
  \frac{\bm{F}_i}{m_i} - \gamma \bm{r_i},
\end{equation}
with
\begin{equation}
  \label{eq:nose_hoover_2}
  \frac{d\gamma}{dt}
  =
  \frac{1}{M}
   \left(
       T(t) - N_{\mathrm{df}} k_B T_0
   \right)
\end{equation}

Equation \ref{eq:nose_hoover_1} is similar to the Langevin thermostat but without the stochastic term.  Since the time-evolution is described by a second order differential equation, the heat may flow to and from the system causing oscillations in the system. Equation \ref{eq:nose_hoover_2} provides the feedback between the $\mathcal{T}$ and $T_0$.

The Nose-Hoover thermostat has one free parameter $M$ which is the mass of the ficticious particle.  This mass serves as mechanism to control the oscillations in the system as a large $M$ will provide a large inertia and make heat flow between the heat bath and the simulated system sluggish.  If the mass is too small, the thermostat will not control the temperature adequately.

Since the equations of motion have changed, the Verlet integration scheme needs to be modified

\subsection{NPT and Barostats}
\label{sec:npt_ensemble}
In the isothermal-isobaric ensemble (NPT), the pressure has a specified average value, while the instantaneous volume $\mathcal{V}$ of the system can fluctuate.

The usual approach for the evaluation of pressure $P$ in an MD simulation involves an ensemble average of the instataneous pressure $\mathcal{P}$\cite{allen1987_md}.  For a system with $N$ atoms in a volume $\mathcal{V}$, the instantaneous pressure can be defined as the contributions of each atom $i$
\begin{equation}
  \mathcal{P}
  =
  \frac{1}{\mathcal{V}}
  \left(
    \frac{1}{3}
    \sum_{i=1}^N m_i \bm{v}_i  \cdot \bm{v}_i
    +
    \frac{1}{3}
    \sum_{i=1}^N \bm{r}_i \cdot \bm{F}_i
  \right)
\end{equation}

For a system with interactions, the pressure can be written as virial equation\cite{tadmor2011_md}
\begin{equation}
  \label{eq:virial_equation}
  P = \langle \frac{N k_B T}{\mathcal{V}}  \rangle
      + \langle \frac{1}{3\mathcal{V}}
        \sum_{i=1}^N
        \langle \bm{r}_i \cdot \bm{F}_{i} \rangle
\end{equation}
The first term of the right hand side of Equation \ref{eq:virial_equation} is the kinetic contribution and the second term is the Clausias virial function modeling residual contribution arising from atoms interacting through the potential $V$.  Since all quantities to calculate pressure are easily acessible in an MD simulation, Equation \ref{eq:virial_equation} can be used to calculate pressure.  The macroscopic pressure $P$ is simply obtained as $P=\langle \mathcal{P} \rangle$.

The Parrinello-Rahman barostat\cite{parrinello1980_barostat,parrinello1981_barostat} allows the simulation cell to vary in time as a result of the difference between varying the internal microscopic stress tensor and the constant external stress tensor $\bm{\omega}$.  The simulation cell is characterized by the tensor $\bm{H}=(\bm{a}_1,\bm{a}_2,\bm{a}_3) \in \mathbb{R}^{3 \times 3}$, where $\bm{a}_1$, $\bm{a}_2$, and $\bm{a}_3$, are the lattice vectors of the simulation cell.

The discussion provided here follows the exposition by Ray and Rahman\cite{ray1984_npt}.  The Parrinello-Rahman equations of motion can be derived from the Hamiltonian from the coordinate system $(\bm{r},\bm{p})$ to the canonical variables $\hat{\bm{r}},\hat{\bm{p}}$, which incorporates $H$,
\begin{equation}
  \hat{\bm{r}}_i = \bm{H}^-1 \bm{r}.
\end{equation}
$\hat{\bm{r}}_i$ is a position vector in fractional coordinates defined by $\bm{H}$.
\begin{equation}
  \hat{\bm{p}}_i = \bm{H}^T \bm{p}_i
\end{equation}

This allows the box to deform in response to an applied stress, so that the equilibrium shape of the box corresponds to the state where the internal stress is equal to the imposed stress.  Then an extended form of the Hamiltonian which adds the potential energy and kinetic energy associated with the shape of the simulation cell $H$, which adds the potential energy due to deformation of the simulation cell and kinetic energy of the box.  The kinetic energy of the box is a ficticious construct which allows the modification of $H$ so that simulation cell is in equilibrium with the external stress tensor.

Since the pressure includes a kinetic component due to particle velocities, barostatting requires regulation of both temperature and pressure.
The Parrinello-Rahman equations can be combined with thermostats to then generate the NPT ensemble\cite{shinoda2004_nosehoover} and then integrated\cite{tuckerman2006_timeintegrator}.

%Which produces the following Lagrangian from which the equations of motion can be derived.

%\begin{equation}
%  \mathcal{L}
%  \left(\hat{\bm{r}},
%    \bm{H},
%    \frac{d\bm{\hat{r}}}
%         {dt}
%    \frac{d\bm{H}}{dt}
%  \right)
%  =
%  \frac{1}{2}
%  \sum_{i=1}{N}
%      m_i
%      \left(
%				\frac{d \bm{\hat{r}} }
%             {dt}
%      \right)^T
%      \bm{G}
%      \left(
%				\frac{d \bm{\hat{r}} }
%             {dt}
%      \right)
%  - V(\bm{r},\bm{H})
%  + \frac{1}{2}
%    W
%    \mathrm{tr}\left(
%      \left(\frac{d\bm{H}}{dt}\right)^T
%      \left(\frac{d\bm{H}}{dt}\right)
%    \right)
%  - \sigma \mathcal(V).
%\%end{equation}
%Here, $\bm{G}=\bm{H}^T\bm{H}$ and $\bm{W}$ is a constant with the dimensionality of weight that serves as a dampening parameter.
 % COMPUTATIONAL TOOLS
\chapter{A PARETO APPROACH TO PARAMETER OPTIMIZATION}

Many decision and plannning problems involve multiple conflicting criteria which must be considered simultaneously.  In the field of optimization, problems which have multiple criteria are deferred to as multiple critieria decision making problems (MCDM) and the algorithms used to solve them as multiple-objective optimization (MOO).

Here the set of feasible solutons is not known in advance, but is restricted by constraint functions.  We concentrate on nonlinear multiobjective optimization and ignore approaches designed for multiobjective linear programming.

Additionally, the approach described within this chapter is described briefly and is followed on by more detailed discussions and application in following chapters and appendices.

The emphasis of the development of this methdology will be on the mathematical aspects of the subject and it's applications to the development of classical empirical potentials.  The intent is to provide tools for a decision maker, rather than to convince which particular optimization to use.

Secondly, another purpose of the development of this methdology is the development of parameterizations which can be expressed as an ensemble of potentials described as a probability distribution function which can be used in UQ propogation.

\section{Multiobjective optimization}

The general multi-objection optimization problem using the notation of Marler and Arora\cite{Marler2004_moo_survey}:
%\begin{equation}
\begin{align}
  & \underset{\bm{x}}{\text{minimize}}
        &\bm{F}(\bm{x}) = [ F_1(\bm{x}),
                            F_2(\bm{x}),
                            ...,
                            F_k(\bm{x})]^T\\
  & \text{subject to}
        &g_j(\bm{x}) \leq 0, j=1,2,...,m \\
  &     &h_l(\bm{x}) = 0, l=,1,2,...,e \\
  &     &\bm{x} \in \bm{X}
\end{align}
%\end{equation}
where $k$ is the number of objective functions, $m$ is the number of inequality constraints, and $e$ is the number of equality constraints.
The vector $\bm{x} \in \bm{X} \subseteq \mathbb{R}^n$ is a vector design variables $x_i$, and $X$ is feasible design space.
$\bm{F}(\bm{x}) \in \mathbb{R}^k$
are called objectives, cost functions, or criteria.  The feasible critereon space $Z$ is defined as
$\{\bm{F}(\bm{x}) | \bm{x} \in \bm{X} \}$.

For MOOs, the objectives are generally conflicting, preventing simultaneous optimization.

\subsection{Pareto optimality}

If all functions are for minimization, a feasible solution $\bm{x}_1$ is said to dominate another feasible solution $\bm{x}_2$, denoted $\bm{F}(\bm{x}_1 \succ \bm{x}_2)$, if and only if $F_i(\bm{x}_1)\leq F_i(\bm{x}_2)$ for $i = 1,...,k$ and $F_j(\bm{x}_1)< F_j(\bm{x}_2$) for at least one objective function $j$.  A solution is said to be Pareto optimal if it is not dominated by any other solution in the solution space.  A Pareto optimal solution cannot be improved with respect to any objective without worsening at least one objective function.  The set of all feasible non-dominated solutions in $X$ is referred to as the Pareto optimal set, and for a given Pareto optimal set, the corresponding values in the objective space are called the Pareto Front.



\section{Pareto Front}

In multiobjective optimization problems, it is characteristic that no unique solution exists, but a set of mathematically equally good solutions can be identified.  These solutions are known as nondominated, efficient, noninferior or Pareto optimal solutions.  In MCDM literature, these terms are synomous.

In MCDM literature, the idea of solving a multiobjective optimization problem is understood as helping a human decision maker (DM) in understanding the multiple objectives simultaneously and finding a Pareto optimal solution.  Thus, the solution process requires some interaction with the DM in the form of specifying preference information and the final solution is determined by these preferences.

In potential development, the preferences of potential developer likewise influences are particular parameterization, which has results in the development of empirical potentials as somewhat of a black art.  In the end, empirical potentials are simplified models which predict structure property relationships.

In classical potential optimization, the identification of an optimal parameterization is determined by the minimization of a cost function which couples multiple objective functions, usually a weighted sum of squares, and different weights are used in an interactive fashion until an acceptable parameterization is determined.
\section{Surveys of Methods}
Chankong and Haimes 1983
Hwang and Masud 1979
Marler and Arora 2004
Miettinen 1999
Sawaragi et al 1985
Steuer 1987
Vincke 1992

We start our review of methods using Hwang and Masud 1979 and Miettinen 199, to classify the different classes of approaches by methological approach rather than technical techniques.
\subsection{no preference methods}
The task is to find some neutral compromise solution without any additional information.  This means instead of asking the DM for preference information, some assumption are made about what a reasonable compromise could be like.
\subsection{\emph{a priori} methods}
In \emph{a priori}, the DM first articulates preference information and the solution tries to find a Pareto optimal solution satisfying them as well as possible.

\subsection{\emph{a posteriori} methods}
A representation of a set of Pareto optimal solution is first generated and then the DM is supposed to select the most preferred one among them.  This approach gives the DM an overview of different solutions available but if there are more than two objectives in the problem, it may be difficult for the DM to analyze the large amount of information.

\subsection{Interactive methods}
After each iteration, some information is provided to the DM in order to specify preference information.  What is noteworthy is that the DM can specify and adjust one's preferences between each iteration and at the same time learn about interdepencies between each iteration and at the same time learn about interdependencies in the problem as well as one's own preferences.


\section{Solution Methods}
MOO solution methods fall under the category of scalarization or non-scalarization methods.  Scalarization is the primary method for MOO problems [Miettinen 1999].  Scalarization converts the MOO problem into a paramterized single-objective problem which can be solved using using well-established single-objective optimization methods.
\subsection{Scalarization Methods}
\subsubsection{Weighting Method}
Gass and Saaty 1955
Zadeh 1963

For a interatomic potential being fit with respect to $k$ quantities of interest,
\begin{equation}
  \begin{aligned}
  & \underset{\bm{\theta}}{\text{minimize}}
        & & \sum_{i=1}^{k}w_{i} \varepsilon_i^2(\bm{\theta}) \\
  & \text{subject to}
        & & \bm{\theta} \in \bm{\Theta}
  \end{aligned}
\end{equation}
where $w_i \geq 0$ for $i=1,...,k$
Weakly Pareto optimal.

  In the development of interatomic potentials, the DM is asked to specify weights in which case the method is used as an \emph{a priori} method.

Algorithms for multiobjective optimization should produce Pareto optimal solutions, and that any Pareto optimal solution can be found.  Censor1977 discusses the conditions which the whole Pareto set can be generated by the weighting metho when positive weights are presented.  In this respect, the weighting method has a serious shortcoming.  Any Pareto optimal solution can be found by altering weights only if the problem is convex.  Some Pareto optimal solutions of nonconvex problems cannot be found regardless of how the weights are selected.

The problems of the weighting schemes have ben explored by the classical potential development community.  The method may jump from one vertex to anohter vertex leaving intermediate solutions undetected with relatively small changes in the weighting schemes.

Scaling of the objective functions.

The weighting method can be used as an \emph{a posteriori} method where different weight can be used to generate different Pareto optimal solutions, and then the DM selects the most satisfactory solution.  Systemic methods of perturbing the weights to obatain different Pareto optimal solutions are suggested (Chankong and Haimes 1983), but Das and Dennis 1997 illustrates that an evenly distributed set of weights does not necessarily produce an evenly distributed representation of the Pareto optimal set, even when the problem is convex.

When the weighting scheme is used as an \emph{a priori} method, the DM is expected to represent his/her preferences in the form of weights.  Roy and Mousseau (1996) suggests that the role of weights in expressing preferences maybe mis leading.  Although the relative importance of weights show the relative importance of the objective functions it is not clear what underlies this notion.  The relative importance of objective functions is usually understood globally, for the entire decision problem, while many practical applications show that the importannce typically varies for different objective function values, that is, the concept is only meaningful locally. (Podinovsky 1994).

Weights that produce a certain Pareto optimal solution are not necessarily unique, and different weights may produce similar solutions.  On the other hand, a small change in weights may cause big differences in the objective function.  It is not easy for the potential developer to control the solution process because weights behave in an indirect way.  The solution process then becomes an interactive one where the DM trues to guess such weights that would produce a satisfactory solution, and this is not desirable because the DM cannot be properly suppported which leads to frustation complications in potential development.  Instead, in such cases it is advisable to use real interactive methods where the DM can better control the solution process with more intuitive preference information.

The weighting method is also difficult
%https://books.google.com/books?id=NHZqCQAAQBAJ&pg=PA1&source=gbs_toc_r&cad=4#v=onepage&q&f=false

\section{Optimization Methods}

\begin{equation}
  \begin{aligned}
  & \underset{\bm{x}}{\text{minimize}}
        & &  f(x)\\
  & \text{subject to}
        & & g_i(x) \leq 0
        & & h_j(x) = 0
        & & x \in X
  \end{aligned}
\end{equation}
here $x$ is the optimization variable, $f$ is the objective function, $g_i$ are inequality constraints, and $h_j$ are equality constraint functions.

\subsection{Convex Optimization}
Numerical algorithms make heavy use of scalarization results, and most papers in the field of MOO and economics deal with non-linear programming problems, corresponding duality theorems, and the repeated application of the simplex method.

However, within the literature of potential development approaches focus upon local minimization techniques and global optimization techniques.

objective function is concave.  constraint set is convex.  KKT requirements for uniqueness.

\subsection{Global Approaches}

Genetic algorithms are a popular meta-heuristic that is particularly well-suited for this class of problems.  Traditional GA are customized to accomodate multi-objective problems by using specialized fitness functions and introducing methods to promote solution diversity.

The second general approach is to determine an entire Pareto optimal solution set or a representative subset. A Pareto optimal set is a set of solutions that are nondominated with respect to each other. While moving from one Pareto solution to another, there is always a certain amount of sacrifice in one objective(s) to achieve a certain amount of gain in the other(s). Pareto optimal solution sets are often preferred to single solutions because they can be practical when considering real-life problems since the final solution of the decision-maker is always a trade-off. Pareto optimal sets can be of varied sizes, but the size of the Pareto set usually increases with the increase in the number of objectives.

The ultimate goal of a multi-objective optimization algorithm is to identify solution in the Pareto optimal set.  However, identifiying the entire Pareto optimal set, for multi-objective problems, is impossible to its size.  Proof of solution optimality is computationally infeasible.  Therefore, a practical approach is achieve successively better approximations of the Pareto surface that represent the Pareto set as well as possible.

A multi-objective optimization approach should achieve the following conflicting goals as described by Zitzler \emph{et al}\cite{zitzler2000_moo_evolve}: (1) the best known Pareto front should be as close as possible to the true Pareto front.  Ideally, the best-known Pareto set should be a subset of the Pareto set, (2) solutions in the best known Pareto set should be uniformly distributed and diverse over the Pareto front in order to provide the decision-maker a true picture of trade-offs, and (3) the best-known Pareto front should capture the whole spectrum of the Pareto front at the extreme ends of the spectrum.  While the first two goals are important for multi-objective optimization, the last goal is erroneous.  In general, when developing potentials, the DM is interested in compromise solutions and a parameterization with high fidelity with respect to one material property at the expense of a loss of fidelity with respect to all other prediction would be a pathological parameterization.

\subsubsection{Genetic Algorithms}
The method which will be proposed in chapter 5 is not a genetic algorithm, but has many similarities as Genetic Algorithms but tailored to create an ensemble of Pareto optimal parameters.  However, it is a genetic solution and the iterative approach of generating new populations is akin to previous solutions.  As a result, the section of review in this section is necessarily incomplete but refer to an introductory review by Konak \emph{et al}\cite{Konak2006_moo_ga} as well as the book by Deb\cite{deb2001_moo_ga}

The concept of genetic algorithms were inspired by evolutionist theories explaining the origin of species\cite{holland1992_ga}.  In nature, weak and unfit speicies within their environment are faced with extinction by natural selection, while strong ones pass on their genes to future generations through reproduction.  In the long run, species carrying the correct combinatioon in their
    genes become dominant in their population.

In GA terminology, a solution vector $\bm{x}\in\bm{X}$ is called an individual or a chromosome.  Chromosomes are made of descrete units called genes.  Each gene controls on or more features of the chromosome.  Normally, a chromosome corresponds to a unique solution $\bm{x}$ in the solution space.  This requires a mapping mechanism between the solution space and chromosome.  GA operates with a collection of chromosomes, called a population.  As the search evolves, the poulation includes fitter and fitter positions, eventually it converges, meaning that it is dominated by a single solution.  Two operators are defined crossover and mutation.  In the crossover operator, two parent solutions are combined togehter to form offspring.  The mutation operator introduces random changes into the population.

The first multi-objective GA, called vector evaluated GA (or VEGA), was proposed by Schaffer [5]. Afterwards, several multi-objective evolutionary algorithms were developed including Multi-objective Genetic Algorithm (MOGA) [6], Niched Pareto Genetic Algorithm (NPGA) [7], Weight-based Genetic Algorithm (WBGA) [8], Random Weighted Genetic Algorithm (RWGA)[9], Nondominated Sorting Genetic Algorithm (NSGA) [10], Strength Pareto Evolutionary Algorithm (SPEA) [11], improved SPEA (SPEA2) [12], Pareto-Archived Evolution Strategy (PAES) [13], Pareto Envelope-based Selection Algorithm (PESA) [14], Region-based Selection in Evolutionary Multiobjective Optimization (PESA-II) [15], Fast Non-dominated Sorting Genetic Algorithm (NSGA-II) [16], Multi-objective Evolutionary Algorithm (MEA) [17], Micro-GA [18], Rank-Density Based Genetic Algorithm (RDGA) [19], and Dynamic Multi-objective Evolutionary Algorithm (DMOEA) [20]. Note that although there are many variations of multi-objective GA in the literature, these cited GA are well-known and credible algorithms that have been used in many applications and their performances were tested in several comparative studies.



\emph{Vector Evaluated Genetic Algorithm (VEGA)}.  Schaffer proposed VEGA for finding multiple solutions to multiobjective problems.  He created VEGA to find and maintain multiple classification rules in a set covering problem.  VEGA tried to achieve this goal by selecting a fraction of the next generation using one of the objective functions.

Fitness Sharing encourage the search in unexplored section of a Pareto front by artificially thinning solutions in densely populated area.  To achieve this goal, densely populated areas are identified and a penalty method is used to penalize the solutions located in such areas.  This approach was recommended by Goldberg and Richardson\cite{goldberg1987genetic} and used by Fonseca and Fleming\cite{fonseca1993multiobjective} to penalize clustered solutions.
\begin{equation}
    dz(\bm{x}_1,\bm{x}_2)
    = \sqrt{\sum_{k=1}^{K}  \left(\frac{z_k(\bm{x_1})-z_k(\bm{x_2})}
                                       {z_{k}^{max}-z_{k}^{min}}
                            \right)^{2}
      }
\end{equation}

based on these distances, calculate a niche count for each solution $\bm{x}\in\bm{X}$ as
\begin{equation}
  nc(\bm{x}_1,t)=\sum_{\bm{x}_2\in\bm{X},r(\bm{x}_2,t)=r(\bm{x}_1,t)}
      \max\left\{ \frac{\sigma_{share}-dz(\bm{x}_1,\bm{x}_2)}
                       {\sigma_{share}},0
          \right\}
\end{equation}
where $\sigma_{share}$ is the niche size by defining a neighborhood of solutions in the objective space.  Solutions in the same neighborhood contribute to each other's nich count.  Therefore, a solution in a crowded neighborhood will have a higher niche count, reducing the probability of selecting that solution from being culled from the survivor set.

\section{Visualization}
This problem is dealt with in discussions about visualization and and analysis of the large amounts of data generated from a posteriori approaches to solving these problems.
Edgeworth 1881
Koopmans 1951
Kuhn Tucker 1951
Pareto 1896, 1906
\section{Treatment}

Our treatment of the mapping of the empirical potential is treated as a bijective mapping into two measure spaces.

Let us define parameter space with the probability measure space $(\Theta,\mathcal{F}(\Theta),\mathbb{P})$.

Then we define the error space of the structure property relationships with the probability measure space $(\mathcal{E},\mathcal{F}(\mathcal{E})),\mathbb{Q})$.
 % PARETO
\chapter{PROBABILITY METHODS} \label{materials}

\section{Probability}
A random variable $X$ is a variable whose possible values are outcomes of a random phenemon.  As a function, a random variable is required to be measurable, which rules out pathological issues.

The underlying foundation of any probability  distribution is the sample space, which is the set of all probable outcomes denoted as $\Omega$.  The realization of an outcome is denoted $\omega \in \Omega$.

The events for the measure space are defined in such a way that a probability measure can be assigned.  This allows to assign probability measures on complex events to characterize groups of outcomes.  The collection of all such events is a $\sigma$-algebra $\mathcal{F}$ of subsets of $\Omega$.  Not every subset of the sample space $\Omega$ must be considered an event, the $\sigma$-algebra restricts $\mathcal{F}$ to subsets of $\Omega$ for which $\mathbb{P}$ can be asssigned.

The probability measure, $\mathbb{P}$, a function which maps $\mathbb{P}:\mathcal{F}\rightarrow[0,1]$.  A probability is a real number between zero and one.  Within this work we do not disguish the difference between impossible events which have probability zero, and probability-zero events which are not necessarily impossible.  Events of probability one is an event that happens almost surely, with almost total certainty.

The triplet $(\Omega,\mathcal{F},\mathbb{P})$ is the probability measure space.

\section{Random Variable}
A random variable has a probability distribution, which specificies the probability falls in.  Specifically, $X:\Omega\rightarrow\mathbb{R}$.  If a random variable $X:\Omega \rightarrow \mathbb{R}$ is defined on the probability space $(\Omega,\mathcal{F},\mathbb{P})$.  Then the probability of an event $A$ occuring is $\{\omega:X(\omega)=A\}$ which is denoted as $\mathbb{P}(X=A)$.

A probability density function for a random variable $X$ has a density $f_X$, where $f_X$ is a non-negative function:
\begin{equation}
  \mathbb{P}[a \leq X \leq b]=\int_{a}^{b}f_{X}(x)dx
\end{equation}

\subsection{Expectation}
If $X$ is a random variable with a finite number of outcomes, $\{x_1,x_2,...,x_k\}$ occurs with the probabilities $\{p_1,p_2,...p_k\}$.  Then the expectation of $X$ is defined as
\begin{equation}
  \mathbb{E}[X]=\sum_{i=1}^{k}x_i p_i
\end{equation}
If $X$ is a continuous random variable, then
\begin{equation}
  \mathbb{E}[X]=\int_{\Omega} X(\omega) d\mathbb{P}(\omega)
\end{equation}
If $X$ admits a density $f(x)$, then the expected value is defined as
\begin{equation}
  \mathbb{E}[X]=\int_{\mathbb{R}}x f_{X}(x) dx
\end{equation}


\section{Estimation of a Distribution}

\subsection{Non-parametric Estimation}
 Fusce eget tempus lectus, non porttitor tellus. Aliquam molestie sed urna quis convallis. Aenean nibh eros, aliquam non eros in, tempus lacinia justo. In magna sapien, blandit a faucibus ac, scelerisque nec purus. Praesent fermentum felis nec massa interdum, vel dapibus mi luctus. Cras id fringilla mauris. Ut molestie eros mi, ut hendrerit nulla tempor et. Pellentesque tortor quam, mattis a scelerisque nec, euismod et odio. Mauris rhoncus metus sit amet risus mattis, eu mattis sem interdum.

\subsection{Kernel Density Estimation}
\subsubsection{Selection of Bandwidth Parameters}
\section{Sampling}

\section{Markov Chain Monte Carlo Methods}
 % PROBABILITY METHODS
\chapter{AN EVOLUTIONARY ALGORITHM FOR GENERATING PARETO EFFICIENT POTENTIALS}
\label{ch:methodology}

%https://pdfs.semanticscholar.org/3b19/b1f81e1b1a8c28a2f0896b6c84d4c88d57df.pdf
The purpose of this chapter to introduce a cost-efficient algorithm to determine the appropriate tradeoffs in concurrent minimization problems associated with multiple criteria.

In the presence of tradeoffs, the set of Pareto optimal points becomes the solution to minimization with respect to multiple objective functions to be minimized.  We refer to the process of estimating the ensemble of rational potential parameterizations which produce this optimal set as Pareto optimization.   The term ‘rational’ will be discussed in detail, but involves the use of a simple algorithm to identify potential parameterizations that make sense to consider further. This is the central idea of our approach and is very different from a conventional potential fitting approach.

Our algorithm has the following goals: (1) to identify the strengths and weaknesses of alternative parameterizations through analysis of the Pareto optimal solutions, (2) to generate estimates of the Pareto optimal front in a series of iteratively better approximations, and (3) to describe the candidate parameterizations through the use of a distribution function and use of Monte Carlo sampling.  Periodically, the sampling scheme is updated by keeping the Pareto optimal points.  This increases the sampling intensity in the region of parameter space where Pareto optimal points have been identified.

The end state of this algorithm is to a identify the probability distribution which produce Pareto optimal results.  This distribution can be used to integrate this algorithm into existing and well-defined UQ methodologies.  In the development of this evolutionary algorithm, some of the techniques used in genetic algorithims and evolutionary algorithms are consciously not implemented.

\section{Evolutionary Optimization Strategy}
\label{sec:strategy}

%The use of evolutionary strategies in potential optimization was pioneered by Frederiksen \emph{et al.} \cite{fredericksen2004_bayesian_fitting} introduces a Bayesian approach to potential parameterization.
%Neural network approaches were pioneered by Behler and Parrinello \cite{behler2007_NN_potdev} and Sanville \emph{et al.} \cite{sanville2008_NN_potdev_si}.
%Genetic algorithm approach\cite{marques2008_ga_potdev} and \cite{hunger1998_ga_potdev}.

\subsection{Problem Definition}

In taking a new approach to potential optimization, we recall the fundamental challenge, which is the multi-objective optimization problem of minimizing the errors for a large number, $N_Q$, of QOIs. In this case, each objective function is the absolute difference between the predicted material property $\hat{q}_i$, and the target values, $q_i$.  If  $\hat{V}(\bm{\theta})$ is an interatomic potential parameterized by the vector $\bm{\theta} = (\theta_1,...,\theta_{N_P})$ with $N_P$ parameters, then predicted material properties becomes $\hat{\bm{q}}(\bm{\theta}) = (\hat{q}_1,..,.\hat{q}_{N_Q})$.
Using the absolute loss function for our choice of objective functions, the problem then becomes the minimization of  $\bm{L}(\bm{\theta}) = (|\hat{q}_1-q_1|,...,|\hat{q}_{N_Q}-q_{N_Q}|)$, simultaneously.
The optimization problem we thus face is minimizing each objective function in $N_Q$ target properties described by a fixed functional form with $N_P$ parameters.  That is, both error space and parameter space typically have a large number of dimensions.  Here we take the approach of generating a large number of parameter sets through random sampling of the parameter space, and algorithmically determining if these candidate potentials are optimal in a multi-objective optimization sense.  The algorithmic approach we take is centered on the idea of Pareto optimality.  To this multi-objective optimization problem, we can impose both equality and inequality constraints on any parameter $\theta_i$ and inequality constraints on any QOI $\hat{q}_j$.

For a set of $N$ parameterizations of a single potential, denoted $\bm{\Theta}= \{\bm{\theta}_1,...\bm{\theta}_N\}$, a single parameterization $\bm{\theta}_0 \in \bm{\Theta}$ is Pareto optimal if and only if there is no other $\bm{\theta} \in \bm{\Theta}$, such that $|\hat{q}_i(\bm{\theta})-q_i| \leq |\hat{q}_i(\bm{\theta})|$
for all $i<N_Q$, and $|\hat{q}_i(\bm{\theta})-q_i| \leq |\hat{q}_i(\bm{\theta})|$ for at least $i$.  That is, a parameterization $\bm{\theta}$ is Pareto optimal if there does not exist another parameterization which improves performance with regard to one QOI without degrading performance with regard to any another.  Contrariwise, a parameterization is not Pareto optimal if there exists at least one other parameterization which provides a better prediction with respect to predicting one material property without degrading the performance of the any other.

In scalar optimization, there is a unique solution for the global optimization problem.   In the proposed approach, the solution to a multi-objective optimization problem is the Pareto surface.  It identifies Pareto optimal potentials in a $N_Q$-dimensional performance space without explicitly defining a scalar cost function.  Throughout the definition of this methodology, the process was designed to avoid \emph{a priori} assumptions.

%\begin{figure}[ht]
%	\centering
%	\includegraphics{chapter5/img/cost_function_optimization}
%	\caption{Depiction of the the typical approach to optimization.}
%	\label{fig:cost_function_optimization}
%\end{figure}

\begin{figure}[ht]
	\centering
	\includegraphics{chapter5/img/pareto_optimization}
	\caption{Schematic of stochastic optimization with Pareto filtering.}
	\label{fig:pareto_optimization}
\end{figure}

We present an optimization scheme that evolves a probability distribution function representing knowledge about the location of Pareto optimal parameters.  As more and better potentials are discovered, the probability distribution function improves the search.  A schematic of the Pareto optimization strategy is depicted in Figure \ref{fig:pareto_optimization}.  This technique is iterative; initial conditions are determined and a series of optimization steps are taken until a convergence condition is met.  Since this process produces an ensemble of potentials, a method of potential selection now occurs.

The red component in Figure \ref{fig:pareto_optimization} is the initial problem definition and incorporation of prior knowledge.  This step consists of the problem definition for parameter optimization.  The scalar optimization of a cost function is replaced with a multi-objective optimization problem.  While scalar optimization results in a single potential $\bm{\theta}^*$, the solution of a MOO problem are the Pareto set $\bm{\Theta}^*$.  Implicit in this process is the identification of structures, properties, and target values which are used in potential fitting. The specific values can be taken from experiment, from higher fidelity calculation methods such as quantum chemical or density functional theory (DFT) calculations, or a combination thereof. Our approach in this step is similar to standard potential development approaches, as discussed in Chapter \ref{ch:potential_development}.  In this approach, a \emph{prior} distribution expressing the uncertainty about the distribution of parameters is defined as starting condition.

The green components of Figure \ref{fig:pareto_optimization} are representative of a Monte Carlo scheme.  Here parameter constraints are applied either by calculation, in the case of equality constraints, or a simple acceptence/rejection mechanism to enforce inequality constraints.  If a generated potential parameter set violates any of the constraints, it is rejected.  Acceptable parameters are then sent to a simulation process which produces predictions on material properties, referred here to as quantities of interest (QOIs).  This process is modeled here as a black box function. These are the set of simulations, as discussed in Chapter \ref{ch:potential_development}.  A framework for implementing the calculation of the QOIs is discussed in Chapter \ref{ch:software}.

The blue components filter the results from the Monte Carlo scheme to provide the optimized results for that iteration.  Dominated points are removed to insure the final set of candidate parameters are a set of Pareto optimal potentials.  Performance filters are applied to satisfy constraints on prediction of material properties.  The last filter proposed removes potential candidates that survive the first two filters, but produce pathologically poor predictions.

The orange components of Figure \ref{fig:pareto_optimization} updates the sampling probability distribution.  A non-parametric probability density function provides more flexiblity than standard parametric probability density functions, which have strong \emph{a priori} assumptions on the underlying distribution.  With a more refined probability distribution function to represent the potential parameters, the prior distribution is updated and a new iterative step can proceed.  To terminate the iterative loop, the algorithm can either continue for a fixed number of iterative steps or until a convergence condition is met.

The yellow component in Figure \ref{fig:pareto_optimization} is the potential selection process.  Unlike the methods discussed in Chapter \ref{ch:potential_development}, this method produces an ensemble of possible parameterizations.  All these potentials meet our criteria: they are Pareto optimal and satisfy constraints on the potential parameters and QOI predictions; any of these potentials can be rationally chosen.  At this point, the selection of potentials becomes subjective.  However, the potential developer can make a more informed decision based upon the tradeoffs available.  Scoring functions are introduced here to simplify analysis.

\subsection{Structure of this Chapter}

The remainder of Section \ref{sec:strategy} provides a broad overview of this algorithm and compares it to the scalar optimization approach presented in Chapter \ref{ch:potential_development}.  While the two approaches have similar process analogues, the typical potential development technique is a deterministic optimization problem; ours is a stochastic process.

Section \ref{sec:probability} starts with a discussion of probability and random variables.  This methodology uses the ideas and methods from a wide rage of probability methods, such as Monte Carlo sampling, Bayesian analysis and uncertainty quantification.  This section introduces notation and terminology and has been adapted for our problem context.

Section \ref{sec:prior_information} incorporates incorporation of prior information before starting the potential optimziation procedure.  Topics such as the selction of the fitting database, the application of constraints on both potential parameters and predictions of material properties were discussed extensively in Chapter \ref{ch:potential_development}.  As a result, this section focuses on the definition of \emph{prior} distribution, which is the initial probability distribution function which will be evolved through sampling and analysis to update the prior distribution iteratively.

Section \ref{sec:sampling} deals with process of sampling parameters $\bm{\theta}$ from a defined probability distribution.  These potential parameterizations ($[\bm{\theta}_1,...,\bm{\theta}_N]$) are set of predictions can be acquired $[\hat{\bm{q}}(\bm{\theta}_1),...,\hat{\bm{q}}(\bm{\theta}_N)]$. This approach is typical of Monte Carlo solution techniques.

Section \ref{sec:filtering} discusses the application of a variety of filters to determine the optimal candidate potentials.  The predominant feature of this methodology is the Pareto filter, but other filters eliminate candidate potentials which violate inequality constraints on predicted material properties.  Additionally, as the number of Pareto optimal candidates increases, the new potential identification becomes more difficult.  To reduce sampling dispersion, a filter to eliminate pathological potentials is devised; this improves sampling in the region of interest by concentrating sampling in the region of the surviving potentials and reduces dispersion due to the spread of the correlation matrix.

Section \ref{sec:iterative_loop} defines the iterative loop by updating the prior distribution used for sampling parameter space. A non-parametric probability density function provides more flexibility over standard parametric distribution function. Issues involving convergence of the probability density function as a stopping condition are also discussed in this section.

Section \ref{sec:selection} discusses of down selection and selection techniques.  Since chapters proceeding this chapter discusses the selection of rational potentials in more detail, this discussion here is cursory.

\section{Probability}
\label{sec:probability}
What follows is a necessarily brief introduction to probability theory.  For a more rigorous approach, the classic textbooks of Rudin\cite{rudin1987_realanalysis} for a treatment of measure theory and Chung\cite{chung2001_probabilitytheory} for a measure theory construction of probability are recommended.

Let $(\Omega,\mathcal{F},\mathbb{P})$ be a measure space with $\mathbb{P}(\Omega)=1$.  Then $(\Omega,\mathcal{F},\mathbb{P})$ is a probability space, with sample space $\Omega$, event space $\mathcal{F}$, and probabilty measure $\mathbb{P}$.  The underlying foundation of any probability distribution is the sample space, which is the set of all probable outcomes denoted as $\Omega$.  The realization of an outcome is denoted $\omega \in \Omega$.

The events for the measure space are defined in such a way that a probability measure can be assigned.  This allows the assignment of probability measures on complex events to characterize groups of outcomes.  The collection of all such events is a $\sigma$-algebra $\mathcal{F}$ of subsets of $\Omega$.  Not every subset of the sample space $\Omega$ must be considered an event, the $\sigma$-algebra restricts $\mathcal{F}$ to subsets of $\Omega$ for which $\mathbb{P}$ can be asssigned.

The probability measure, $\mathbb{P}$, is a function which maps $\mathbb{P}:\mathcal{F}\rightarrow[0,1]$.  A probability is a real number between zero and one.  Within this work, we do not distinguish between impossible events which have probability zero, and probability-zero events which are not necessarily impossible.  Events of probability one happens almost surely, with almost total certainty.

\subsection{Random Variables}

A random variable $X$ is a variable whose possible values are outcomes of random phenomena.  As a function, a random variable is required to be measurable, which rules out pathological issues.  The probability measure $\mathbb{P}$ assigns probabilities for a random variable $X$ over the possible values $x \in X$.  Specifically, $X:\Omega\rightarrow\mathbb{R}$.  If a random variable $X:\Omega \rightarrow \mathbb{R}$ is defined on the probability space $(\Omega,\mathcal{F},\mathbb{P})$, then the probability of an event $A$ occuring is $\{\omega:X(\omega)=A\}$ which is denoted as $\mathbb{P}(X=A)$.

Formally, a random variable is a particular type of measurable function.  However, there is an important difference which may be more philosophical than mathematical due to the difference in the underlying domains.  A random variable operates on a set of outcomes or processes defined by $\Omega$, while deterministic variables does not.  This isn't a mathematical difference, as the underlying domains are just sets, and the $\sigma$-algebra and measures provide the relevant mathematical structure.

For a probability space, even if a specific random process isn't mentioned, there is an implied assumption that such a random process exists.  Otherwise, there would be no need for probability to have its own language.

A random variable is evaluated by performing an unpredictable experiment.  That is, one in which $\omega \in \Omega$ is not predetermined.  The selection of a specific $\omega$, isn't the evaluation of a random variable, it is simply the evaluation of a measurable function.  We denote the realization of a random variable as $x(\omega) \in X(\Omega)$, to differentiate from the deterministic elements of a set, which uses the $x \in X$ notation.


\subsection{Probability Density Functions}

In the potential optimization process, the potential parameters are modeled as random variables.  This is denoted $\bm{\theta}(\omega_i) \in \bm{\Theta}(\Omega)$ with $\omega \in \Omega$.  The random variable $\bm{\Theta}(\Omega)$ has specific realizations $\bm{\theta}(\omega)$ on the event space $\omega \in \Omega$ with probabilities determined by the probability measure $\mathbb{P}$.
Since a specific outcome $\omega_i$ produces $\bm{\theta}$, $\bm{\theta}_i=\bm{\theta}(\omega_i)$ is no longer probabilistic, but deterministic.  Thus, $N$ candidate parameterizations generated from $\bm{\Theta}(\Omega)$ can be denoted as a set of $N$ constant vectors,
$
 \{\bm{\theta}(\omega_1),...,\bm{\theta}(\omega_N)\}
=\{\bm{\theta}_1,...,\bm{\theta}_N\}
=\bm{\Theta}
$.  In this work, the $\omega$ notation usually indicates simulations to be done that still need to be realized, and the lack of of an $\omega$ notation indicates that the simulation have already been done.

If $\bm{\Theta}(\Omega)$ is a random variable with a finite number of outcomes, then the sequence of $\{\bm{\theta}_1,...,\bm{\theta}_N\}$ occurs with the probabilities $\{p_1,p_2,...p_k\}$, and $\mathbb{P}(\bm{\theta})=p_{\Theta}(\bm{\theta})$ is a \emph{probability mass function}.
To simplify notation, $p_\Theta(\bm{\theta})=p(\bm{\theta})$ omits the subscript $\Theta$ since $\bm{\theta}\in\bm{\Theta}$ implies it.

A continuous random variable $\bm{\Theta}(\Omega)$ admits the \emph{probability density function} $p_{\Theta}(\bm{\theta})$. The probability density function is used to compute probabilities over ranges of values. The probability of an event $\bm{\Theta}' \subseteq \bm{\Theta}$ occuring is defined by the integral.
\begin{equation}
	\label{eq:continuous_random_variable}
  \mathbb{P}(\bm{\Theta}=\bm{\Theta}')
	=
	\int_{\theta \in \Theta'} p_{\Theta}(\bm{\theta}) d \bm{\theta}.
\end{equation}
For continuous random variables, the probability of a specific event happening $\mathbb{P}(\theta(\omega))$ is vanishingly small.  Since the bounds of integration on Equation \ref{eq:continuous_random_variable} vanish,  $\mathbb{P}(\bm{\theta}(\omega))=0$, almost surely.  The probability density function (PDF) still remains an expression of likeliness.
If $p(\bm{\theta}_1)>p(\bm{\theta}_2)$, then $\bm{\theta}_1$ is more likely to happen than $\bm{\theta}_2$.

 The expectation of a random variable $\mathbb{E}$ is the average of the outcomes weighted by their probability mass function or probability density function.  For discrete random variables,
 \begin{equation}
	 \mathbb{E}[X] = \sum_{x\in X} x p(x).
	\end{equation}
For continuous random variables, this summation becomes the
\begin{equation}
	\mathbb{E}[X] = \int_{-\infty}^{+\infty} x p(x) dx
\end{equation}
The expectation of a function $f$ dependent upon a random variable $X$ can likewise be defined as
\begin{equation}
	\mathbb{E}[f(X)] = \int_{-\infty}^{+\infty} f(x) p(x) dx
\end{equation}
\subsection{Monte Carlo Methods}
Monte Carlo methods\cite{thomopoulos2013_montecarlo} are a broad class of computational algorithms which are dependent upon random sampling to obtain numerical results.
The essential aspect of these approaches is to use randomness to solve problems which might be deterministic in principle.
Monte Carlo simulations sample from a probability distribution for each variable to produce an arbitrarily large number of possible outcomes; the results are analyzed to get probabilities of different outcomes occuring.  The process for generating random samples from a probability distribution function is discussed more thoroughly in Section \ref{sec:sampling}.

In scalar optimization problems, random sampling methods are ubiquituous for global minimization, including techniques such as simulated annealing\cite{kirkpatrick1983_simmulated_annealing}, genetic algorithms\cite{holland1975_ga}, and tabu search\cite{glover1986_tabu}.  Although these approaches require a large number of samples, they do not require \emph{a priori} knowledge of the objective function to achieve the global mimima.
Here we take the approach of generating a large number of parameter sets through random sampling of the parameter space, and then algorithmically determining if the set of candidate potentials are optimal in the multi-objective context.  The algorithmic approach we take is centered on the idea of Pareto optimality.

\subsection{Interpretation of Probabilities}

The meaning of the probabilities assigned to potential values of a random variable $\bm{\Theta}(\Omega)$ is not related to probability theory itself, but instead related to philosophical context over the interpretation of probability.  In most scientific applications, the probability interpretation is physical in nature, in which probabilities are objective and an event's probability is the limit of its relative frequency as the number of samples becomes large.  This interpretation supports the statistical needs and inference requirements for experimental scientists since probabilities can be found by a repeatable objective process (e.g. experiments), and are thus devoid of opinion.  Within this context, probabilities are \emph{aleatory}.  Uncertainty is irreducible and inherent to to the physical process observed.

For finite temperature simulation above $0$ K, molecular dynamics simulations allow sampling from themodynamic ensembles.  Although the simulations are replicable since time-integration and the empirical potential are deterministic, sampling from the time-series or by randomly assigning initial velocities, allows molecular dynamics to model the aleatory uncertainty associated with a $T > 0$ K simulations.

Within the context of potential development, probabilities are evidentiary and can be assigned to any problem statements even through no random process is involved, as a way to represent subjective plausibility, or to the degree in which the statement is supported by available evidence.  Evidential probabilities represent degrees of belief, such as the Laplace interpretation where probabilities are defined in the disposition to gamble at certain odds\cite{laplace1902_probability}.  These coherent subjective belief systems follow the laws of probability\cite{ramsey2016truth,definetti1980_foresight}.  Here, uncertainty is \emph{epistemic} and is based on current evidence, but is mutable based upon receiving new observations\cite{ramsey2016truth,definetti1980_foresight,jaynes2003_probability}.  This philsophical interpretation is more closely associatiated with Bayesian inference\cite{gelman_bayesian}.

In practice, the uncertainty associated with any problem has both \emph{aleatory} and \emph{epistemic} components.  Within this context, the field of uncertainty quantification\cite{oberkampf2010_vvuq} is involved with the quantitative characterization and reduction of uncertainties in both computational and real world applications.  The difficulty of applying Bayesian frameworks to potential development is constrained to the size of the training dataset and the biases produced in assigning preferences due to the inevitable tradeoffs in material properties prediction.  Moreover, the inability to define uncertainities associated with target properties in a meaningful probabilistic sense frustrates standard VVUQ and Bayesian approaches.

\section{Incorporation of Prior Information}
\label{sec:prior_information}

All iterative techniques have initial starting conditions.  In a deterministic method such as scalar minimization using gradient descent techniques, the starting conditions are the initial estimate of the solution.  In our case, the location of the best estimate may not be known; the initial condition requires not only an expression of location but also of range.  Bayesian techniques refer to this process as the definition of the prior distribution.  Here, we generalize this concept by not only defining the prior distribution, but also constraints on the potential optimization problem.

For existing formalisms, one can incorporate parameter information from existing potentials.  For each parameter, define an uppper and lower bound to define the uniform distribution.  Having a uniform prior means that the probability density is constant over its support, defined as $\theta_{i}^{\text{min}} \leq \theta_i \leq \theta_{i}^{\text{max}}.$  A uniform prior doesn't favor any particular value of $\theta_i$, since it supposes that all values in the range are equally likely.  This is referred to in Bayesian literature as a \emph{vague} or \emph{uninformative} prior.  Strictly speaking, the use of uniform priors are optimal since an uniformative prior information for $\bm{\theta}$ must be identical over its support.    However, complete ignorance about the distribution of a potentials parameter may not be true.

Based upon application, different probability distribution functions can be selected to represent prior knowledge.  The starting condition for Pareto analysis could be based upon the existing parameterization, $\bm{\theta}_0^*$, of a potential formalism.  This optimization choice is based upon a set of undocumented optimization conditions, such as optimization method, the choice of weighting vectors of the cost function, $C(\bm{\theta}|\bm{w})$, and the initial starting condition, $\bm{\theta}_0$.  In this case, the potential developer may be interested in a Pareto analysis of the local energy basin in the neighborhood of  $\bm{\theta}_0^*$.  Sensitivity analysis evaluates changes in the predictive performance of a potential caused by perturbations on $\bm{\theta}_0^*$.  Since a potential developer would never rationally choose a potential that was not Pareto optimal, it is desirable to impose the restriction that changes in potential parameterization must remain Pareto efficient.  A probabability distribution such as the multivariate normal distribution could be selected with the mean vector equal to $\bm{\theta}_0$ to localize the region of sampling.  To disperse the sampling around the mean, the choice of a covariance matrix $\bm{\Sigma}$ is necessary.  A covariance matrix can be defined with $\bm{\sigma}_{\theta_i}$ for the diagonal elements.
Here specifying $\bm{\mu} = \bm{\theta}_0$ encodes prior information about the location of Pareto optimal potentials, while the uncertainty of $\bm{\Sigma}$ remains subjective.  For each $\theta_i$, the potential developer encodes the \emph{epistemic} uncertainty associated with $\theta_i$ with $\sigma_i^2$.

The choice of the \emph{prior} distribution is not important other than it encompasses the region of interest without making the sampling region excessively large.  If the sampling region is too large, then Monte Carlo sampling will produce undesirable \emph{discrepancies} in the sampling sequence; the sequence of random variates will fail to adequately fill the probability space.

In this methodology, potential development is iterative.  To save computational resources, we can identify a set of potentials that meet a small subset of desirable conditions, such as the cohesive energy $E_c$ and the lattice parameter $a_0$ being close to the reference values, $q_{E_c}$ and $q_{a_0}$ respectively, without the constraints of Pareto optimality.  We call this process \emph{pre-conditioning} as it provides enough data points to avoid catastrophic simulation failure due to an ill-conditioned covariance matrix.  Here, the entire population is used as an expression of the prior distribution and can be used to estimate a parametric probability distribution or used to define a non-parametric probability distribution.

\section{Sampling from the Prior Distribution}
\label{sec:sampling}

\subsection{Sampling distribution}

The sampling distribution, $p(\hat{\bm{q}}|\bm{\theta})$, is the distribution of the QOIs conditioned upon the prior distribution $p(\bm{\theta})$.  Instead of analytically determining $p(\hat{\bm{q}}|\bm{\theta})$ and sampling from it, we note that $\hat{\bm{q}}(\bm{\theta})$ is a deterministic function.  As a result, the random variable $\hat{\bm{Q}}(\Omega)$ is a function dependent upon upon $\bm{\Theta}(\Omega)$, which we will exploit as an alternative method of sampling from $\hat{\bm{Q}}(\Omega)$.  We now proceed to outline the process for sampling from $\hat{\bm{Q}}(\Omega)$.

First, generate $\bm{\theta}(\omega_1) \in \bm{\Theta}(\Omega)$, using the prior distribution described in Section \ref{sec:prior_information}.  For a simulation length of $N$, then random variates of potential parameterizations are the independent samples $\{\bm{\theta}(\omega_1),...,\bm{\theta}(\omega_N)\}$ with $\omega_i \in \Omega$ to define the sequence
\begin{equation}
\label{eq:theta_seq}
	  \bm{\Theta} = \{\bm{\theta}_1,...,\bm{\theta}_N\}
\end{equation}

To sample from $p(\hat{\bm{q}}|\bm{\theta})$, the sequence defined in Equation \ref{eq:theta_seq} is evaluated through simulation machinery to calculate
\begin{equation}
\label{eq:qoi_seq}
	\hat{\bm{Q}}
	= \hat{\bm{Q}}(\bm{\Theta})
	= \{\hat{\bm{q}}(\bm{\theta}_1),...,\hat{\bm{q}}(\bm{\theta}_N)\}
\end{equation}

To show that sequence $\bm{Q}$ is generated from $\hat{\bm{Q}}(\Omega)$, consider that each $\bm{\theta}_i$ is a specific realization from $\bm{\Theta}(\Omega)$ with $\omega_i \in \Omega$.
For each $\hat{\bm{q}}(\bm{\theta}_i)$,
\begin{equation}
	\hat{\bm{q}}(\bm{\theta}_i) = \hat{\bm{q}}(\bm{\theta}(\omega_i)) = \hat{\bm{q}}(\omega_i) \in \hat{\bm{Q}}(\Omega)
\end{equation}
with $\omega_i \in \Omega$.

\subsection{Enforcement of Parameter Constraints}

The process of sampling makes the application of parameter constraints trivial.  Equality and inequality constraints are the two classes of potential parameter constraints which have to be enforced.

The enforcement of the inequality constraint $g_{\theta}$ on the parameter $\theta \in \Theta$, imposes $p(\theta_i)=0$ for every $\theta_i$ which violates $g_{\theta}$.  Instead of analytically modifying the prior distribution, an acceptance/reject mechanism is implemented.  For a parameter set $\bm{\theta}_j$, the parameter set is rejected and not included in $\bm{\Theta}$ when if $\theta_{ij}$ violates $g_{\theta}$,

Equality constraints on the parameter $\bm{\theta}_i$ is a function $h_{\theta_i} (\{\bm{\theta}_j; i \neq j\},A)$ where $\{\bm{\theta}_j; i \neq j\}$ are remaining parameters and $A$ is a constant.  To prevent simulation failures, equality constraints should always be imposed explicitly.  Random variates of the free parameters are generated.  Evaluating the function $h$ calculates the equality constrained parameter directly.

Since our proposed process uses a Monte Carlo sampling strategy, the implementation of parameter constraints only requires the definition of constraints as fully determined function of the free parameters.  In the conventional approach of described in Chapter \ref{ch:potential_development}, constraints by implementing either an approximating simplex algorithm.  For exact solutions, non-linear programming strategy must be devised, and the becomes NP-hard when $\hat{\bm{q}}(\bm{\theta})$ is convex\cite{bertsekas1995_nlp}.

\subsection{Generation of Random Variates}
\label{sec:generating_random_variates}

The generation of random variates from a probability distributon is a well-defined process, provided in many statistical software libraries.  However, this approach does present some computational challenges due to numerical stability, which are addressed here.

The commonly used technique to generate random variates of a random variable with a probability distribution function is the inverse transform method.

Let $U(\Omega)$ be a uniform random variable in the range $[0,1]$.  For the potential parameter $\theta(\omega) \in \Theta(\Omega)$ with the probability density function $p_\Theta(\theta)$, the monotonically increasing cumulative density function can be defined,
\begin{equation}
	  F(\theta) = \mathbb{P}(\Theta < \theta) = \int_{-\infty}^\theta p_\Theta(x) dx
\end{equation}

The random variable $\Theta(\Omega)$ can now be defined as a function of the $U(\Omega)$, $\Theta(\Omega)=F^{-1}(U(\Omega))$.  Therefore, if we have a random number to generate numbers according to the uniform distribution, we can generate any random variable with a known probability density function.

The generating of random variates for multivariate probability density functions is considerably more complicated since the the cumulative density function has no mulitvariate analog.  More typically, for multivariate Monte Carlo sampling schemes, the choice of the prior distribution must be some transformation of the multivariate normal distribution, which has convenient mathematical properties\cite{devroye1986}.

The generation of draws from a univariate normal distribution $y \in N(\mu,\sigma^2)$ can be done from the rescaling of a standard normal distribution $x \in N(0,1)$ by the transformation
\begin{equation}
\label{eq:normal_scaling}
	  y = \mu + \sigma x
\end{equation}
which is dependent upon a high quality method to draw a simulated $N(0,1)$, which is a mature technology.

The multivariate normal distribution (MVN) is defined by $\bm{\mu}$ and covariance matrix $\bm{\Sigma}$.
Since  $\sigma_{ij}=\sigma_{ji}$, $\Sigma$ must be symmetric.  The necessity for the MVN having a positive definite covariance matrix comes from its probability density functions.
\begin{equation}
	  p_{\mathrm{MVN}}(\bm{x})
		=
		\frac{1}
		     {(2\pi)^{1/2} |\bm{\Sigma}|^{1/2} }
		\exp\left( -\frac{1}{2} (\bm{x}-\bm{\mu})^T
		                        \bm{\Sigma}^{-1}
														(\bm{x}-\bm{\mu})\right)
\end{equation}
which requires invertibility of the covariance matrix.

% https://crmda.dept.ku.edu/guides/42.mvn/mvn-generator-1.pdf
The multivariate version of the rescaling described in Equation \ref{eq:normal_scaling} is described in Scheuer and Stoller\cite{scheuer1962_mvn_rv}, which is the predominant implementation of generating random variates in statistical software.  A vector $\bm{x}$ of $N$ values is drawn from an $\mathrm{MVN}(\bm{0},\bm{I})$ process.  Setting the mean vector $\mu$ to $\bm{0}$ centers the distribution at $\bm{0}$.
Settting the covariance matrix $\bm{\Sigma}$ to the identity matrix $\bm{I}$, ensures that every component of $\bm{x}$ is independent from each other since all off-diagonal terms are set to zero.  The multivariate standard normal distribution is equivalent to stating that each element of $\bm{x}$ can be drawn independently from $x_i \sim N(0,1)$.
We want to apply a transformation so that the result is $\bm{y} \sim \mathrm{MVN}(\bm{\mu},\bm{\Sigma})$.  This transformation is $\bm{y}=\bm{\mu} + \bm{S} \bm{x}$ where $\bm{S}$ is a square scaling matrix.  $\bm{S}$ has the same purpose that $\sigma$ has in Equation \ref{eq:normal_scaling}.
If the scaling matrix is chosen so that $\bm{\Sigma}_Y = \bm{S}\bm{S}^T$, this transformation produces the correct expectations
\begin{equation}
	  \mathbb{E}[\bm{y}]
		= \mathbb{E}\left[\bm{\mu} + \bm{S}\bm{x}\right]
		= \bm{\mu} + \bm{S} \mathbb{E}[\bm{x}]
		= \bm{\mu}
\end{equation}
the covariance matrix of $\bm{y}$ is
\begin{equation}
	\bm{\Sigma}_Y
	= \bm{S} \bm{\Sigma}_X \bm{S}^T
	= \bm{S} \bm{I} \bm{S}^T
	= \bm{S}\bm{S}^T
\end{equation}
To establish that $\bm{y}$ is an MVN from this transformation, any linear fuction of a vector of jointly normally distributed variables is also normally distributed\cite{greene2003}.  Thus, for $\bm{x} \sim N(\bm{\mu},\bm{\Sigma})$,
\begin{equation}
    \bm{A} \bm{x} + \bm{b}
		\sim N(\bm{A} \bm{\mu} + \bm{b},
		       \bm{A} \bm{\Sigma} \bm{A}^T)
\end{equation}

In the univariate case, $\sigma$ is the square root of the variance $\sigma^2$.  The variance is a parameter of the normal distribution, and $\sigma$ is a derived quantity and could have taken either a negative or positive value.  Since in Equation \ref{eq:normal_scaling} we interpret $\sigma$ to be the standard deviation of a variable, we choose $\sigma > 0$.  Likewise, matrix square roots are grossly different from each other, and there are many approaches known as matrix decompositions to determine them.  In many statistical packages, the Cholesky decomposition of $\bm{\Sigma}$ is calculated to find an lower triangular matrix $\bm{L}$ so that $\Sigma = \bm{L}\bm{L}^T$\cite{golub1996_matrices}.

Since $\bm{\Sigma}$ is positive definite, then all of the eigenvalues of $\bm{\Sigma}$ are greater than zero.  However, an eigenvalue which is 0 or a small positive number can lead to a negative number due to truncation error in numerical routines.  For random variate methods depending upon the Cholesky decomposition, this leads to catastrophic failure of the generation of samples dependent upon generating random variates of the MVN.  For our methodology, later iterations are dependent upon drawing samples from the the MVN.  To correct this problem, the eigenvalue decomposition on $\bm{\Sigma}$ is calculated to check that $\bm{\Sigma}$ is positive semi-definite by inspection of the eigenvalues.  If the eigenvalue is tolerably negative, that eigenvalue is either set to $0$ or a sufficiently small number, and the modified covariance matrix is used.

Accounting for the equality constraints requires modification of the sampling procedure.  If $\bm{\theta}_i$ is a linear combination of other potential parameters, then the underlying covariance matrix will be rank deficient.  By definition, an equality constraint imposes a correlation coefficient of unity to an off-diagonal term, which prevents the Cholesky decomposition of the covariance matrix.  The problems of rank deficiency of the correlation matrix is avoided by evaluating equality constrained potential parameters by direct evaluation of the constraining function after the unconstrained potential parameters have been generated.

\section{Filtering Simulation Results}
\label{sec:filtering}

The final step in the Bayesian process is determination of the posterior distribution.  The posterior distribution is the distribution of the parameters $p(\bm{\theta}|\hat{\bm{q}})$ after taking to account the observed data.  In the deterministic approach, the deterministic approach outlined in Chapter \ref{ch:potential_development}, $\bm{\Theta}^*$ is the Pareto set.  Here $\bm{\Theta}^*(\Omega)$ is a random variable with a probability density function $p(\bm{\theta}^*)$ which produces potential parameterization which are likely to be Pareto efficient.

At this point, our approach deviates significant from a Bayesian inference approach which determines this probability distribution \emph{ex ante}.  That is the probability distribution is estimated numerically, and then we can generate a population sample from the defined random variable.

Instead, we exploit that the target reference values $\bm{q}$ are deterministic.  Rather than analytically determining $p(\bm{\theta}|\hat{\bm{q}})$, we filter the simulation results generated from $p(\hat{\bm{q}}|\bm{\theta})$.

To aquire the potential parameters which produce Pareto optimal results, we take the candiate potentials and filter out the dominated points.  We start with the results of the simulations from Section \ref{sec:sampling}, where our simulated results are $\hat{\bm{Q}}= \{\hat{\bm{q}}(\bm{\theta}_1),...,\hat{\bm{q}}(\bm{\theta}_N)\}$.
 If the dominated points are removed from this set and the parameters re-indexed, then there are $M$ Pareto efficient points
 $\$\hat{\bm{Q}^*}
  = \{\hat{\bm{q}}(\bm{\theta}_1^*)
	  ,...
		,\hat{\bm{q}}(\bm{\theta}_M^*)\}$
  with $M < N$.  The set of parameters which produce the Pareto efficient points is
	$\bm{\Theta}^*=\{\bm{\theta}_1^*,...,\bm{\theta}_M^*\}$

To formalize the concept of a filter, here we define a set of filter operations $(\mathcal{F}_1,...,\mathcal{F}_{N_F})$ for $N_F$ number of filters which imposes a series of constraints on the performance of QOIs.  When $\mathcal{F}$ is applied to the set of  predicted QOIs $\hat{\bm{Q}}$ as defined in Equation \ref{eq:theta_seq}, then $\mathcal{F}(\hat{\bm{Q}})$ is the set of potentials which conform to the constraints defined in $\mathcal{F}$ and $\mathcal{F}(\hat{\bm{Q}}) \subseteq \hat{\bm{Q}}$.
Then the set of which conforms to all our constraints is
\begin{equation}
  \hat{\bm{Q}}^*
	=
	\bigcap_{i=1}^{N_F} \mathcal{F}_i(\hat{\bm{Q}})
\end{equation}
Since each  $\hat{\bm{q}}(\bm{\theta}^*) \in \hat{\bm{Q}}^*$, the construction of the potential parameterization which produce $\hat{\bm{Q}}^*$ is trivial, which we denote  $\hat{\bm{\Theta}^*}$.
As a result, the filtered population of candidate potentials becomes the basis to estimate the posterior distribution \emph{ex post}, using the typical frequentist methods of estimating probability density functions, achieving our goal of estimating $p(\bm{\theta}|\hat{\bm{q}})$.

The \emph{ex post} filtering framework provides considerable flexibility, since a variety of constraints can now be imposed on $\hat{\bm{q}}$ without regard to the construction of a likelihood function.
A variety of constraints, such as requiring the candidate potentials to be Pareto optimal and enforcing desirable inequality constraints on the QOI predictions now become possible.

These filters should be interpreted as an acceptance/rejection strategy \emph{ex post} on $\bm{Q}(\Omega)$ to impose Pareto optimality conditions and inequality constraints on the QOIs.  This approach has significantly more flexiblity to encode desirable \emph{a priori} constraints than the conventional approach described Chapter \ref{ch:potential_development}, while at the same time eliminating the problematic expression of preferences \emph{a priori} through a weighting scheme.

A third filter eliminates pathological potentials.  While not theoretically necessary, this filter is necessary to improve the ability of our sampling scheme to identify new Pareto optimal potentials.  A pathological potential survives the Pareto filter and the QOI constraint filter, but clearly has undesirable properties.  As an extreme example, a pathological potential may have best predictive performance respect to $q_i$, but have extremely poor performance with respect to $q_j$.  The potential survives the filtering process due to its superior predictive ability of $q_i$, but a rational potential developer would reject the potential due to its poor performance in $q_j$.

  Since the representation of $\bm{\Theta}(\Omega)$ is a non-parametric distribution and dependent upon the covariant matrix, keeping this potential poses several issues in our sampling scheme.  First, in a non-parametric distribution each potential parameterization kept in $\bm{\Theta}(\Omega)$ becomes an elevated region of increased probability density in $p(\bm{\theta})$.  As a result, computational resources are expended searching a region which produces pathological potentials.  When a potential which is an outlier in prediction of QOIs, it remains indeterminate whether or not that potential is an outlier in parameter space.  In the case, where the potential is an outlier in parameter space, the point has high leverage which distorts orientation of the dispersion defined by the covariance matrix in the direction of the pathological parameterization.  This means not only is a portion of the search localized in the region of the pathological potential, but also the direction of the search defined by the covariance matrix is in the direction of the pathological potential.  The case where a pathological potential is close to a non-pathological potential indicates a region of parameter space where prediction $\hat{\bm{q}}(\bm{\theta})$ has high sensitivity to small changes changes in the parameters $\bm{\theta}$.  In this case, it is clearly desirable to be aggressive in removing the potential.

	A simple method to remove these undesirable potentials potentials is defined by defining a scaled distance from $\bm{q}$.  This is the ideal performance of a potential, where $\bm{\epsilon}(\bm{\theta})=0)$, and is not attainable since we cannot minimize all $\epsilon_i(\bm{\theta})$ simulatenously.  Since each $q_i$ varies in magnitude, it is necessary to scale the distance.  This scaled distance metric then becomes
	\begin{equation}
		d(\bm{\theta}) = \left(\sum_{i=1}^{N_Q} \left(\frac{\epsilon}{q_i}\right)^2\right)^{1/2}
	\end{equation}
	At the end of each iteration, a small percentile of the population is eliminated.  Those potentials with the largest values of $d$ are excluded from membership in $\bm{\Theta}^*$.  A range of percentile values have been tried with 5\% providing robust results.  If the percentile value is too small, the process will fail to identify new Pareto optimal parameters after only a few iterations.  If this value is too large, this filter becomes too aggressive and localizes sampling in the region of QOI space where $\hat{\bm{q}}(\bm{\theta})$ is closest to $\bm{q}$ as defined by our scaled distance metric.  This truncates the Pareto surface which may eliminate regions of iterest.

\section{Defining the Iterative Loop}
\label{sec:iterative_loop}
Since this process is heavily influenced by Bayesian optimization techniques, we use the mathematical notation of Bayesian estimation, although this approach does not use calculated likelihood functions to determine the posterior distribution.

Instead, we identify Pareto optimal points, and use potential candidates to estimate a non-parametric distribution, allowing the algorithm to focus the search in regions of parameter space where Pareto optimal parameterizations have been already identified.

This probabilty distribution updates the prior distribution, and we can restart the procedure until a convergence condition is met.

In this manner, successively better approximations of the probability density, $p(\bm{\theta}^*)$ predicting the location of Pareto optimal parameterizations are achieved.
\subsection{Non-parametric Distribution}
%% https://nic.schraudolph.org/teach/ml03/ML_Class4.pdf
We discuss the univariate case of modelling probability distributions.
Let $(\theta_1,...,\theta_n)$ be a univariate and identically distributed sample drawn from some distribution with an unknown density $p$.
In this case, determining the prior distribution for the Monte Carlo process now becomes a statistical problem; a probability distribution function needs to be estimated from the data.  The problem can be solved using \emph{parametric} approach distributions by estimating the parameters from the data of a suitable standard distribution or in a \emph{non-parametric} approach by estimating the shape of this function $p$.

A non-parametric approach provides significant benefits to this algorithm.  A parametric approach imposes \emph{a priori} assumptions on the probability density function used to model the distribution of Pareto optimal potentials.  For example, it is possible that the underlying distribution is multi-modal, a violation of standard distribution functions.  The existence of local minima in the optimization process suggests that each of these energy basins would manifest itself as region of elevated probability density.  In this case, the location of the mean would be a weighted expectation of the modes and would concentrate sampling in a region which may not be Pareto optimal.

%In H\"ormann and Bayer\cite{horman2000_kde}

Let $X$ be a random variable and the $(x_1,...,x_N)$ be independent and identically distributed (IID) samples with $x(\omega) \in X(\Omega)$, where the the probability density function $p(x)$ is not known.  We are interested in using the kernel density estimators (KDE) introduced by Rosenblatt\cite{rosenblatt1956_kde} and Parzen\cite{parzen1962_kde} to estimate $p(x)$.   These estimators $\hat{p}$ are defined by
\begin{equation}
\label{eq:kde_univariate}
    \hat{p}_h(x) = \frac{1}{n} \sum_{i=1}^N K_h(x-x_i)
\end{equation}
where $h$ is the bandwidth and $K$ is kernel, $K_h(u)=K(u/h)/h$.  The following assumptions are made on the kernel:
\begin{itemize}
    \item $K$ is symmetric, $K(u)=K(-u)$,
    \item $\int_\mathbb{R} K(u)du = 1$,
\end{itemize}

The kernel density estimator $\hat{p}_h(x)$ then has the intuitive motivation that a probability mass $K$ is placed at each point $x_i$ and then averaged by the number of points $n$.  A variety of kernels can be chosen, but here we use the Gaussian kernel due to convenient mathematical properties.  The Gaussian kernel $\phi$ is,
\begin{equation}
  \phi(x)=\frac{1}{\sqrt{2\pi\sigma}}\exp{\left(-\frac{t^2}{2\sigma^2}\right)}
\end{equation}
Here, we focus on the choice of $h$ since the choice of the bandwidth parameter is much more important than the choice of the kernel for the behavior of $\hat{p}_h(x)$\cite{silverman1986_kde}.

Extensive literature exists on the choice of bandwidth estimators.  Turlach\cite{turlach1993_kde_bw} is recommended as an introduction to the general approach to objective, data-driven bandwidth selection methods.  Heidenreich \emph{et al.}\cite{heidenreich2013_kde_bw} provides a more modern review of the topic.  This work uses the least-squares cross-validation selection bandwidth estimator of Chiu\cite{chiu1991_kde_bw}.

Sampling from a KDE uses mutiple applications of the inverse transformation method discussed earlier.  If there are $N$ candidate potentials, then there are $N$ kernels located at $\bm{\theta}_i$ for $1 \leq i \leq N$.  Generate a random integer $I$ uniformly distributed on $(1,...,N)$ to select the location of a specific kernel.  Since the kernels are identical by construction, a random variate $W$ is generated from the kernel's PDF which disperses the sampling away from the location of the kernel.  The random variate from the kernel density estimate is then $\bm{\Theta}(\Omega) = \bm{\theta}_I + h W(\Omega)$, where $h$ is the smoothing parameter in Equation \ref{eq:kde_univariate}.

In the multivariate case, the kernels are the MVN distribution and the random variates are generated using the method discussed in Section \ref{sec:generating_random_variates}.  The bandwidth matrix of $\bm{\Theta}$ is scaled from its covariance matrix by the bandwidth factor $h$.  In practical use, two simulation errors manifest themselves; the covariance matrix $\bm{\Sigma}$ can be singular and the covariance matrix can be non-positive definate.  This computational issue was discussed in detail in Section \ref{sec:sampling}.  In this case, a \emph{pre-conditioning} process is recommended.  The Monte Carlo scheme is executed to evaluate a large number of potentials against a small number of QOIs, typically the cohesive energy and the lattice parameter.  The parameterizations close to the target values are retained to update the prior; the Pareto filter is not applied.  This process is continued for a number of iterations.

%% https://brilliant.org/wiki/gaussian-mixture-model/

\subsection{Convergence Condition}
The Kullback-Leiber divergence\cite{kullback1951_kld}, $D_{KL}(\rho_1\vert\vert\rho_2)$, measures how one probability distribution, $f$, diverges from a second probability distribution function, $g$.  For continuous random variables, $D_{KL}$, is defined as the integral
\begin{equation}
\label{eq:kld1}
   D(f \parallel g) = \int f(x) \log \frac{f(x)}
                                          {g(x)} dx
\end{equation}
and has the following properties: (1) self-similarity, $D(f \parallel f) = 0$, (2) self-identification, $D(f \parallel g) = 0$ only if $f=g$, and (3) positivity, $D(f \parallel g) \geq 0$ for all $f$ and $g$.

This value is used as a metric for convergence of $p(\bm{\theta})$.
Since early iterations are more likely to cause changes in the set approximating $\bm{\Theta}^*$ than later simulations, changes in $D_{KL}$ will initially be large.  However, it becomes increasingly more difficult to identify further Pareto-optimal solutions later in the simulation.  As the number of iterations, $t$, increases, then Kullbach-Leiber divergence $D_{KL}(p_{t-1}\vert\vert p_t)$ convergences asymtotically to zero.

For our application, our distributions are KDEs, so the evaluation of the integral can be done by Monte Carlo integration.  The integral in Equation \ref{eq:kld1} can be calculated from Monte Carlo\cite{hershey2007_kld_approx}, by drawing a sample $x_i$, from the statstical distribution of $f$ such that $\mathbb{E}\left[\log\frac{f(x)}{g(x)}\right] = D(f \parallel g)$.  Using $N$ IID samples $\left\{x_i\right\}_{i=1}^N$, we have
\begin{equation}
  \label{eq:kdmc}
  D_{MC}(f \parallel g) = \frac{1}{N}\sum_i^N \log \frac{f(x)}{g(x)}
      \rightarrow D(f \parallel g)
\end{equation}
as $N \rightarrow \infty$.  To compute {$D_{\mathrm{MC}}(f \parallel g)$}, we need to generate samples $\left\{x_i\right\}_{i=1}^N$ from $f$.  Then for $1 \leq i \leq N$, evaluate $f(x_i)$ and $g(x_i)$ to calculate $D_{\mathrm{MC}}$
for $k < \infty$ iterations.

\section{Selection of Potentials}
\label{sec:selection}
The situation of appraising inference vs. decision arises when we start applying probability theory to our problem.  One can think about our methdology as an optimization problem, which optimizes the distribution of $\bm{\Theta}^*$, by systematically reducing the epistemic uncertainty of our probability distribution.  Here, probability theory determines the state of knowledge about the distribution of parameters which produces Pareto optimal results; it does not tell the potential developer which parameter is the best, or even which region of parameterizations are the best.

This methdology can only solve the inference problem; it can only provide a probability distribution which represents the final state of knowledge with all available prior information and data taken into account.  The job is not finished at this point.  An gap in this design of this methodology is a set of rules which converts the final probability assignment into a definative selection of potentials.  A simple system of scoring and ranking each potential is suggested in Chapter \ref{ch:ionic_MgO}

Evident in Equation \ref{eq:kde_univariate} is the implicit assumption that $p_{\bm{\Theta}}$ is homoskedastic; the covariance matrix is identical for each kernel.  Chapter \ref{ch:pareto_si}, provides an example where the direction of dispersion defined by the correlation matrix is different for different regions of parameter space and QOI space.  Different orientations of the covariance matrix in QOI space suggests that candidate potentials can be classified through clustering techniques, since the eigenvector corresponding to the largest eigenvalue approximates the vector normal to hyperplane parallel to the Pareto surface for that cluster.  This can reduce the tasks of selecting from thousands of potential alternatives to a single representative potentialq from each cluster.

The Gaussian mixture model (GMM) is a probabilistic model for representing normally distributed subpopulations within an overall population. Mixture models in general don't require knowing which subpopulation a data point belongs to, allowing the model to learn the subpopulations automatically. Since subpopulation assignment is not known, this constitutes a form of unsupervised learning.

By partitioning the parameter space, a KDE can be defined for each cluster, each cluster with its own bandwidth matrix.  This allows each region to have a sampling scheme where the bandwidth matrix has the optimal bandwidth scaling factor $h$ based upon covariance matrix of the local population, but the direction of the dispersion in aligned with with the Pareto surface.

The chapters proceeding this chapter provide data driven approach to parameter selection in the case of the development of a Buckingham potential for MgO.  In the case of developing a potential for Si, a different approach is taken.  Here, the $\bm{\Theta}^*$ is partitioned based upon similar performance characteristics, this allows us to characterize regions of parameter space.
 % DESCRIPTION OF THE FITTING METHOD
\chapter{POTENTIAL DEVELOPMENT SOFTWARE}

Classical interatomic potential reduce the quantum-mechanical interactions of electrons and nuclei to an effective interaction between a collection of atoms described by an analytical set of functions.
This greatly reduces the computational effort in molecular dynamics (MD) simulations.

Potentials are typically obtained by determining the potential paramters which optmize a set of reference data, which typically includes experimental values such as lattice contants, cohesive energies, elastic constants, and are supplemented with \emph{ab-initio} obtained data such as defect formation energies.

In this section, the potential development software Python Potential Operational Software Package (\emph{pypospack}), is presented which is a software library which automates the development of analytical interatomic potentials using evolutionary stochastic optimization techniques, described in the previous chapter.

Like other potential development software packages, \emph{pypospack} was developed to seprate the process of parameter optimization from the selection of the analytical form of the potental.  \emph{pypospack} separates itself from other packages by designed using object-oriented methods which enables users of this software to quickly integrate new potentials, material properties, or even optimization techniques.

Additionally, \emph{pypospack}

Moreover, \emph{pypospack} takes a different approach to the development of potentials by identifying a set of Pareto optimal potentials through an evolutionary stochastic optimization algorithm.

Software for potential software development is by necessity a complicated piece of software which draws expertise from many disciplines for software development.  The goal of pypospack is not to deliver a monolithic software application, but provide a software with a flexible software architectural library from which potential developers can quickly create their own potential optimziation software, by leveraging an object-orient software framework based upon a series of core packages, which deliver specific functionality to \emph{pypospack}.

\section{Software Architecture}

\emph{pypospack} is a object-oriented framework, written in Python3 and targeted to a packages curated and maintained to the Anaconda software distribution.

\subsection{Data}
\subsection{Atomic Structure Files}

\emph{pypospack} uses a a custom class SimulationCell which describes atomic positions as a representative unit volume, bounded by the three lattice vectors, $\bm{a}_1$, $\bm{a}_2$, and $\bm{a}_3$, where $\bm{a}_i=[a_{i1},a_{i2},a_{i3}]$.  These are stored locally by an $H$-matrix representation,
\begin{equation}\label{eq:H-matrix}
     H=\left[\right] =
      \begin{bmatrix}
        a_{11}&a_{12}&a_{13}\\
        a_{21}&a_{22}&a_{23}\\
        a_{31}&a_{32}&a_{33}
      \end{bmatrix}
\end{equation}

\subsection{Configuration File}
One of the goals of \emph{pypospack} is to make it easy for any user to get started quickly

In particular, the software is written to alleviate the potential developer from understanding the complete implementation of all the algorithms for sampling, potential development, calculation of

\section{Implementation}
Pypospack was written in python to leverage the strength of python as a high-level language for writing scientific applications.  Python has a liberal open source license which makes the distribution of the application without license issues.  Since Python is available on many platform, there are few issues with portability between platform.

Python flexible syntax allows potential developers to either use the pypospack library to define a potential develop process as a simple procedural script, or allow the developers to encapulate implementation details away from the end users.

We look at the success of the materials project, where the materials properties can be predicted by solving fundamental laws of physics using quantum mechanical appoximation such as density functional theory (DFT).  This virtual testing of materials was employed to design and optimize materials \emph{in silico}.

\emph{pypospack}
\subsection{Material System Representation}

Currently, \emph{pypospack} uses the POSCAR file format utilized by VASP as the primary input definition of atomic configuration files to represent different types of structures.

In addition, \emph{pypospack} has a compatibiity layer with the atomistic simulation envioronment (ASE).  This provides capability with a wide variety of structure files.

\subsection{Parallelization}

Since \emph{pypospack} uses Monte Carlo sampling as the workhorse for estimation, the computational effort can be parallelized over the parameter space being searched.  Concurrency is implemented by assigning each processor a different random seed and number of simulations is divded equally amongst the processors.  To avoid issues with file access, each rank is given it's own directory and the results are processed by the rank $0$ processor at the end of each iteration.

Currency is done though \emph{mpi4py} which provides a python interface to the standard Message Passing Interface (MPI), since the implementation details are taken care of by \emph{mpi4py}, \emph{pypospack} supports the broad range of MPI implementations.

\subsection{Energy Evaluations}

The energy evaluation of \emph{pypospack} is done through an integration layer which has been implemented for both GULP and LAMMPS.  The choice to use an external energy calculator rather than implementing one internally was chosen to minimize
The success of these approaches is largely driven that epistemic uncertainty associated with DFT calculations are largely driven by the unknown functional form of the exchange correlation functional.

Contrast this to molecular dynamics, where $V(\bm{r}_{ij})$ has uncertainty in both the functional form and parameterization.  Before attacking problems with functional form,

Let us first consider, three fairly simple systems which are widely studied, where the fundamental physics are significantly different.  Metal systems are described by

\subsection{Statistical Sampling}

Statistical Sampling

\subsection{Machine Learning Algorithms}

SciKitLearn

\emph{In silico} approaches to molecular dynamics simulation is largely constrained by long development times required to develop a potential.  Molecular dyanamics simulations are largely dominated

Within the ICME approach, different levels of theory lead to different methods of simulation which span the quantum mechanical level, currently dominated by \emph{ab initio} techniques such as Density Functional Theory (DFT) to the engineering scale.  Between these two scales there are atomistic level Molecular Dynamics/Monte Carlo techniques and meso scale simulations.

Pypospack is written as a library to help bridge the \emph{ab initio} techniques to high si



Error estimation may be monitored through the time-evolution of the ensemble average.  However, this is a necessary, but not sufficient condition for convergence, because plateaus of the ensemble average often conceal anomalous overlap of the density of stats characterizing the initial and final states.  The latter should be the key criterion to ascertain the local convergence of the simulation for those degrees of freedom that are effectively sampled.

Statistical errors in FEP calculations may be estimated by means of a first-order expansion of free energy, which involves an estimation of the sampling ratio of the latter of the calculation (Straatsma, 1986)

\begin{equation}\label{eq:helmoholtz_free_energy}
  A = U-TS + \sum \mu_i N_i
\end{equation}

\begin{equation}\label{eq:gibbs_free_energy}
  G = U+PV-TS + \sum \mu_i N_i = H - TS + \sum \mu_i N_i
\end{equation}

This software is a collection of software libaries which can be scripted together to evaluate software.

Weight free approaches to developing potentials, such as genetic algorithms, exist and use the concept of Pareto efficiency as metric in which to efficiency of a potential.

Pypospack helps to resolve the problems for potential developers by eliminating the requirements for developing code for MPI and complex queueing systems.  Instead functionality is developed by exposing high-level APIs to potential developers, while the implementation by lower level APIs is done by the software itself.

This allows potential developers to focus on the creation of a testing database, to define reference structure property relationships.  Pypospack then uses an evolutionary algorithm based upon evolving a probability distribution which describes the density of Pareto optimal points in the parameter set.

A series of reusable analytical tools are also developed for the purposes of ad hoc data analysis, and automated data analysis, which aids the potential developer to select potentials ex-post, and then to test these ensembles of potentials against other structure property relationships which are more expensive.

Since the description of the solution of potentials is represented as a probability distribution, the package can be adopted to Bayesian inference techniques which will enable UQ on predictions which represent the propagation of uncertainty to potential development.

Finally, this package also provides some tools to begin to tackle problems of model form uncertainty, transferribility, etc.

This project required the development of software for the development of Pareto frontier.

\section{Representation of Atomic Structures}


\section{Quantities of Interest}

Let us start our discussion of the calculation of point defects by starting reviewing the notion of Kr\"oger-Vink notation\cite{kroger1956_notation}

\section{Tasks}

\section{Processing Pipeline}
Currently the architecture consists of server side application which are intended to evaluate a large numbers of parameterizations for a defined functional form.

The parameter sampling framework was designed as to make parallelization trivially simple.

Within a single simulation

\section{Implementation of the OpenKIM API}

In order to support the largest number of classical potentials, software was written in python using object oriented techniques so that new types of simulations, new potentials, new quantities of interest, and new simulation software.

\section{Possible Scalability Issues}

At the current time, pypospack built upon the scientific python stack.
Evaluation of interatomic potentials.  This software subprocesses either serial version of LAMMPS or GULP to calculate properties of interest.  Parallelization is batched processed across iterations, which each processor rank being given a unique directory space to prevent IO conflicts.

Supports Buckingham, Tersoff, and EAM potentials.  PYPOSPACK was largely developed to support LAMMPS simulations, but has limited support for GULP as well.

Support for atomic strucutures.  PYPOSPACK defines structures in terms of the lattice basis of the simulation cell, and the atomic basis of the simulation cell.  Typically, the software uses the POSCAR format for atomic structures used in VASP software.  However, support for creating surfaces, interfaces, and other file formats from integration with the ASE toolkit.

PyposmatEngine.  The purpose of this class is to evaluate a given parameterization for a QOI set by managing.  This class allows potential optimization to be done through any software framework where the algorithm provides a set of parameters to be evaluated, and can return the result of structure property estimation or their predictive errors with respect to a predefined training set.  Monitoring of spawned subprocesses and timeouts for poor parameterizations allows a rank to recover from structural minimization routines of unstable parameterizations.

QoiManager

TaskManager


The Materials Project\cite{Jain2013_materialsproject}
Materials Application Programming Interface (MAPI) \cite{ong2015_mapi}
Python Materials Genomics\cite{ong2013_pymatgen}
 % SOFTWARE DEVELOPMENT
\chapter{APPLICATIONS TO IONIC SYSTEMS}

Lewis Catlow potential\cite{lewis1985_buckingham}

Henkelman \emph{et al}\cite{henkelman2005_buckingham_MgO}

Lewis and Catlow\cite{lewis1985_buckingham} parameterized a wide range of oxide systems with a pairwise potential, between two atoms $i$ and $j$.
\begin{equation}
  V_{ij}(r_{ij})=\frac{Z_i Z_j}{r_{ij}}+A_{ij}\exp(-r_{ij}/\rho_{ij})-\frac{C_{ij}}{r_{ij}^6}
\end{equation}
The physics of ionic systems is modelled with three pairwise terms.
The first term is a Coulombic interaction dependent on the the point charges, $Z_i$ and $Z_j$.
The other terms is a Buckingham potential the repulsion between the two ions due to the Pauli exclusion principle with the exponential function combined with a term for a weak van der Waals interaction.
In the Lewis and Catlow parameterization (LC), the charges are assumed to have the formal charges of $\pm 2e$ on the magnesium and oxygen ions,respectively.
Additionally, the cations only interact with each other though the Coulomb interaction.

Two potentials unpublished Buckingham potentials were developed by Ball and Grimes (BG1 and BG2), but were used in the work of Henkelman \emph{et al}\cite{henkelman2005_buckingham_MgO}.
 % APPLICATION TO IONIC SOLIDS
\chapter{APPLICATIONS TO NICKEL EMBEDDED ATOM POTENTIALS}

The aim of this chapter is to apply the earlier described methods and tools involved in developing potentials for metallic systems.  The defining difference between metals and nonmetals is the lack of an energy gap between ground and excited states.  Metallic systems are normally characterized as a system existing within a sea of delocalized electrons.

\section{Insights from Quantum Mechancial Techniques}

In the development of the Finnis-Sinclair method, the second-moment tight-binding(3)

\section{Embedded Atom Model}

For metallic systems, the most common potential forms are those of the Finnis-Sinclar method\cite{finnis1984_fs} and the embedded atom model (EAM) \cite{daw1983_eam,daw1984_eam}.
Both of these approaches have similar formalisms which combine a pair-potential function and an energy functional dependent upon the electron density contributions from an atoms neighbors with the total energy of the system $V$ being described as

\begin{equation}
  V=\sum_{i<j}\phi_{s_i s_j}(r_{ij})
	+\sum_{i}F_{s_i}(\bar{\rho}_{i})
\end{equation}

The first term is the summation over all neighbors of a pair potential of a pair potential energy($\phi_{ij}$) between two atoms, $i$ and $j$, with the chemical species, $s_i$ and $s_j$.
The second term is the embedding energy ($F_i$) necessary to place the atom $i$ at its position in an electron gas density ($\bar{\phi}_i$) influenced by all the neighbors of the atom.
In EAM, the host electron density is given by the sum of the contributions $\rho_i$ is represented by the sum of the contributions $\rho_{s_j}(r)$ from all neighboring atoms $j$.
\begin{equation}
	\bar{\rho}_i = \sum_{j \neq i} \rho_{s_j}(r_{ij})
\end{equation}

In the Finnis-Sinclair potential, $F_{s_i}(\bar{\rho})=-\sqrt{\bar{\rho}_i}$

\section{Approaches to EAM Potential Development}

The original formulation of the EAM represented the reepulsion between the atomic cores (nuclei and inner electron shells), represented by a power law or an Born-Mayer type exponential function.  In these cases, the pair potential, $\phi$, is purely repulsive, while the embedding energy $F$ was attractive.
Later EAM pair potentials used Morse Functions[10,19,20], which has a distinct energy well, while the embedding function in these cases serves as a corrective term.

In either case, the electron density function and the pair potential are given as analytic functions of the radial separation of two atoms $r_{ij}$ with fitted parameters.  The embedding function is sometimes specified as having a specific functional form, or determined by fitting the embedding function to an equation of state such as Rose \emph{et al}[21].

For Nickel, Johnson and Oh\cite{johnson1989_eam_embedded_bjs}, Voter-Chen[19,23], Angelo \emph{et al}[20] have used parametric functional forms, where Ercolessi and Adams[24] and Mishin \emph{25} implemented cubic-spline approaches, which do not specify an analytical functional form.

In early potential development, the parameters are adjusted to match properties such cohesive energy, lattice parameters, elastic properties, and vacancy formation energies.  With additional computational power, stacking fault energies, phase order differences, and surface energies to fit to important relevant environments thought to be important to produce transferrable potentials.

\section{Development of an EAM potential}

Even when a potential has an analytical functional form. EAM potentials are specified in a functional form which is dependent a specific code format which precalculates evaluation of the $\phi(r_{ij})$, $\rho(r_{ij})$, and $F(\bar{\rho})$.  Specifics of this calculation are provided in Appendex \ref{appendix:setfl}.

\subsection{Potential Formalisms}

A variety of potential formalisms have been explored.

\subsubsection{Density Function}


Mishin\cite{mishin2004_eam_NiAl} and Purja Pun and Mishin\cite{purjapun2009_eam_NiAl} use the following density function (\verb|density_mishin2004|):

\begin{equation}
\label{eq:density_mishin2004}
	\rho(r) = A_0 (r-r_0)^y e^{-\gamma (r-r_0)}(1+B_0 e^{-\gamma (r-r_0} + C0,
\end{equation}
Mishin sets $A_$ to be zero, but can arbitrarily set to any number since

which is mollified by the function $\psi(x) = x^4 / (1+x^4), such that the mollified density function is 

\begin{equation}
    \bar{\rho}(r) = \psi \left(\frac{r-r_c}{h}\right)\rho(r)
\end{equation}

\subsection{Pair Potential Functions}


The Ni pair potential takes the form of a generalized LJ potential as in Mishin\cite{mishin2004_eam_NiAl}

\begin{equation}
\label{eq:pair_mishin2004}
\end{equation}

\section{Generalized Stacking Fault in FCC}
The generalized stacking fault in FCC metals describe the the slip of \{111\} planes of the face centered cubic cell in the <112> direction, which represents the energy
In the early works of Frenkel and Mackenzie describe this motion as a function of the macroscopically measured shear modulus, the Burgers vector and the interplanar spacing of the \{111\} planes.

Rice[3] unstable stacking faults and stable stacking faults

Vitek\cite{vitek1966_gsf,vitek1968_gsf} develops the notion of the generalized stacking fault, which cannot be measured experimental except at a single point knows as the intrinsic stacking fault $\gamma_{ISF}$

Calculation of unstable stacking faults\cite{sun1990_eam_esf,sun1993_eam_esf,farkas1997_eam_usf} was done with EAM potentials, while DFT has been used for the calcualtion of the GSF curve [14,15,16]

Zimmerman takes the approach of creating a simulation of the FCC crystal oriented in the <111> direction, with the basal plane formed by the <112> and the <111> direction.  For the ease of creating the stacking fault, the <112> direction is chosen on the x-axis, and the <110> is chosen for the y-axis, and the z-axis on the <111>.  To create the stacking fault, the lower half remains fixed, while the upperhalf is displaced in the <112> direction in small increments.  Zimmerman uses 6000 atoms consisting of 30 {111} planes.  After lateral displacement, the atoms are allowed to relax laterally (in the <111> direction) except for three {111} planes at the top and bottom.  This method is used in [11-13]

To create a surrogate model, we use the atomic simulation enviornment (ASE)

\subsection{Density Functional Theory}



\subsection{Molecular Dynamics}

First generate a simulation cell of a representative fcc bulk with \hkl[1 1 2] in the x direction, \hkl[-1 1 0] in the y direction, and \hkl[-1 -1 1] in the z-direction, which was adapted from the script of Spear\cite{spear2012_lammps_gsf}.
 % EAM POTENTIAL DEVELOPMENT
%\chapter{Techniques for Dimensionality Reduction}

\section{Principal Components Analysis}

Principal component analysis (PCA) is a statistical procedure that uses an orthogonal transformation to convert a set of observations of possibly correlated variables into a set of values linearly uncorrelated variables called principal compoents.

Pearson[1] and Hotelling[2]
 % SI
%\chapter{SUMMARY OF WORK}

\section{General Implications}

\section{Future Work}
 % MACHINE LEARNING
%\chapter{SUMMARY OF WORK}
\label{ch:summary}

\section{Significant Contributions}
This work set out to present an emergent framework for the automated development of analytical potentials.  Here we summarize key contributions in this work.

In Chapter \ref{ch:potential_development}, a critical assessment of the conventional process for potential paramaterization was provided.  We identified already known weakness with the process, but presented them in a more rigorous way.  Using cardinal optimization techniques is inappropriate because conditions for convergence to a global solutions are not met.  Convergence to global minima can only be met if the process is repeated across a dense range of initial conditions in space of possible potential parameterizations, or if random sampling techniques are employed.  In addition, the cost function formulation is unable to identify regions in the predictive performance when the optimal performance region is convex.  This creates high sensitivity to encoded preferences in convex regions due to the existence of degenerate solutions.  Finally, the \emph{a priori} expression of weights is required to be encoded at the beginning of the process.

We introduce the concept of ordinal minimization to the field of potential development, where the solution to parameter optimization is not expressed in an arbitrary cost function, which serves as an artifical heuristic, but in ordinal terms where we calculate an indifference curve based upon the concept of Pareto optimality where each candidate parameterization is mutally non-dominated.  This formulation elminates problems of local minima, expression of \emph{a priori} expression of weights, only imposes the condition the preservation ordinality to measure losses with respect to individual material properties of interest, and can identify solutions in convex regions of performance space normally occluded to cardinal optimization methods.

When the optimal performance has convexity or discontinuities, we show that the computational efficiency of Monte Carlo sampling techniquqes provides an order of magnitude improvement in  numerical efficiency in estimating the Pareto optimal surface, while simultaneously providing an order of magnitude computational efficiency in the number of simulations required.  In addition, ordinal optimization uses independent sampling which by definition provides near linear scalability in the ability of the algorithm to use processors.

Chapter \ref{ch:methodology} builds upon those concepts.  The implementation of constraints both in the domain of potential parameter design variable and the predictive performance is difficult within a cardinal optimization context, since Karush-Kuhn-Tucker conditions are likely not met, and most certaintly cannot be proved.  Cardinal optimization techniques have the undesirable task of implementing a non-linear programming scheme based upon simulations which must be treated as black box functions.  Here, we presented a modified Monte Carlo technique which uses acceptance-rejection sampling to modify the the distribution of the random variable \emph{ex post}.  This technique is considerably more flexible as ordinal optimization schemes are based upon criteria which are only dependent upon expressing an inequality.  This technique starts with the expression of the epistemic uncertainty that a potential developer has respect the location of the desireable potential parameters, and updates it in a Bayesian manner using evidence generated from Monte Carlo sampling.  This defines a new evolutionary algorithm for potential optimization that applicable for any multi-objective optimization program.

Chapter \ref{ch:software} implements these concepts into software.  A new software code \emph{pypospack} was developed to provide a framework for optimizing interatomic potentials.  Here the implementation of potentials, simulation tasks, and material properties are decomposed as independent components in an object-oriented architecture.  The use of inheritance allows us to encapsulate implementation details such as process handling, parallelization, and workflow management.  In turn, this lowers the necessary effort and expertise to contribute to the project.  An application \emph{pyposmat} which uses the \emph{pypospack} software libraries is used to implement the methodology described in Chapter \ref{ch:methodology}, and is applied to specific material systems in Chapters \ref{ch:ionic_MgO} and \ref{ch:pareto_si}.

Chapter \ref{ch:ionic_MgO} demonstrates that our unsupervised algorithm can deliver similar performance as expert-developed potentials for the magnesium oxide ionic system.  However, our Pareto optimization strategy has the inverse problem of cardinal optimizaton.  While cardinal optimization produces too few solutions.  The Pareto optimization strategy produces too many.  To solve this problem, we apply biobjective visualization techniques which projects the high dimensional data into a series of two-dimensional subspace.  This allows us to identify the performance tradeoffs in a global bivariate sense.

Chapter \ref{ch:pareto_si} introduces classification algorithms to the set of candidate potentials.  In the potential develop of a Stillinger-Weber potential for silicon, a large number of potential parameterizations are identified.  The number of potentials is so large that visualization techniques fail.  Here we apply the Gaussian mixture model (GMM) to partition the population and select a much smaller number of components determined by calculating the information criteria, which penalizes the selection of a more complicated model that adds too many components to the GMM.  This produces clusters which can be described with analytical multi-variate normal distributions.  The Student's $t$-test is used to assess the predictive performance of each material property of each cluster, and aggregate test for the performance was constructed.  This provides an algorithmic method for eliminating clusters with undesirable predictive based upon a $p$-value statistical tests.

\section{Future Work}

This new methodology is promising, but recommendations for future work and improvements are now discussed.

In order to provide continual improvement of the Pareto surface, it is necessary to constantly remove candidate potentials.  This process defines a cardinal scoring functions then ranks the potentials eliminating the poorest performance with respect to this metric.  Here, we scale the absolute errors by the target values and eliminate the worst performers by varying the volume of a hypersphere in this scaled space until a desired percentile of the candidate potentials are contained within this hypersphere.  Here, the choice of scaling is somewhat arbitrary and approximate, but serves to scale the errors to the same order of magnitude.  However, this filter seems to govern the optimization process and should be studied further.

This process produces are large number of candidate parameterizations.  However, the analysis space is high dimensional both in the design space of the potential parameters as well as the response space of predictive performance.  Linear projections into a two-dimensional subspace results in the loss of data.  The ill-conditioning of the covariance matrix suggests variance reduction techniques could be performed using eigenvalue-eigenvector analysis using a process such as principle components analysis, to project into a lower dimensional subspace without sacrificing too much information loss.

In addition, the distribution of parameters and predicted material properties are likely clustered due the existence both convex and concave regions of the Pareto surface.  We have demonstrated that clustering techniques are promising, but methodologies to translate these results into potential selection and improved sampling techniques have not been adequately explored.

On the software wide, we implement a simple parellization scheme which partitions the sample space and each processor processes an equal number of potentials.  However, some parameterizations may take more time to produce a result than others.  This leads to some processors finishing before others.  A more efficient concurrency scheme could be implemented.

Currently, the calculation of the Pareto surface and the calculation of the bandwidth parameter using cross-valdiation techniques are computational limitations since these calculations are performed on a single processor which increases memory requirements and computational time.   Parallelization schmes for both these processes would improve scalability.

Our potential produces a large amount of data.  Currently, the sampling scheme only uses information about the location of Pareto optimal points in determining random variates.  However, since the Pareto surface is the performance envelope between feasible and infeasible space, the dominated potentials provide considerable information on where to direct sample dispersions.  In addition, \emph{pypospack} produces a large amount of structured information stored in flat files.  The interactivity in analysis can be improved by integrating this information into a database, which can ease workstation analysis requirements due to the considerable memory and input-output demands of analysis.

Finally, the methodology provided here is specifed in probabilistic terms to make it compatible with VVUQ techniques.  This goal has not been accomplished due to the inadequacies of the formalism of the interatomic potential.  The reduction of epistemic uncertainty when performance tradeoffs are large implies a high degree of uncertainty in distribution of the potential parameters.  The modification of standard VVUQ techniques are necessary to deal with the inevitable biases when selecting a region of parameterization, particularly when dealing with propagating parametric uncertainty to dynamic simulations.
 % CONCLUSION
%\chapter{SUMMARY OF WORK}
\label{ch:summary}

\section{Significant Contributions}
This work set out to present an emergent framework for the automated development of analytical potentials.  Here we summarize key contributions in this work.

In Chapter \ref{ch:potential_development}, a critical assessment of the conventional process for potential paramaterization was provided.  We identified already known weakness with the process, but presented them in a more rigorous way.  Using cardinal optimization techniques is inappropriate because conditions for convergence to a global solutions are not met.  Convergence to global minima can only be met if the process is repeated across a dense range of initial conditions in space of possible potential parameterizations, or if random sampling techniques are employed.  In addition, the cost function formulation is unable to identify regions in the predictive performance when the optimal performance region is convex.  This creates high sensitivity to encoded preferences in convex regions due to the existence of degenerate solutions.  Finally, the \emph{a priori} expression of weights is required to be encoded at the beginning of the process.

We introduce the concept of ordinal minimization to the field of potential development, where the solution to parameter optimization is not expressed in an arbitrary cost function, which serves as an artifical heuristic, but in ordinal terms where we calculate an indifference curve based upon the concept of Pareto optimality where each candidate parameterization is mutally non-dominated.  This formulation elminates problems of local minima, expression of \emph{a priori} expression of weights, only imposes the condition the preservation ordinality to measure losses with respect to individual material properties of interest, and can identify solutions in convex regions of performance space normally occluded to cardinal optimization methods.

When the optimal performance has convexity or discontinuities, we show that the computational efficiency of Monte Carlo sampling techniquqes provides an order of magnitude improvement in  numerical efficiency in estimating the Pareto optimal surface, while simultaneously providing an order of magnitude computational efficiency in the number of simulations required.  In addition, ordinal optimization uses independent sampling which by definition provides near linear scalability in the ability of the algorithm to use processors.

Chapter \ref{ch:methodology} builds upon those concepts.  The implementation of constraints both in the domain of potential parameter design variable and the predictive performance is difficult within a cardinal optimization context, since Karush-Kuhn-Tucker conditions are likely not met, and most certaintly cannot be proved.  Cardinal optimization techniques have the undesirable task of implementing a non-linear programming scheme based upon simulations which must be treated as black box functions.  Here, we presented a modified Monte Carlo technique which uses acceptance-rejection sampling to modify the the distribution of the random variable \emph{ex post}.  This technique is considerably more flexible as ordinal optimization schemes are based upon criteria which are only dependent upon expressing an inequality.  This technique starts with the expression of the epistemic uncertainty that a potential developer has respect the location of the desireable potential parameters, and updates it in a Bayesian manner using evidence generated from Monte Carlo sampling.  This defines a new evolutionary algorithm for potential optimization that applicable for any multi-objective optimization program.

Chapter \ref{ch:software} implements these concepts into software.  A new software code \emph{pypospack} was developed to provide a framework for optimizing interatomic potentials.  Here the implementation of potentials, simulation tasks, and material properties are decomposed as independent components in an object-oriented architecture.  The use of inheritance allows us to encapsulate implementation details such as process handling, parallelization, and workflow management.  In turn, this lowers the necessary effort and expertise to contribute to the project.  An application \emph{pyposmat} which uses the \emph{pypospack} software libraries is used to implement the methodology described in Chapter \ref{ch:methodology}, and is applied to specific material systems in Chapters \ref{ch:ionic_MgO} and \ref{ch:pareto_si}.

Chapter \ref{ch:ionic_MgO} demonstrates that our unsupervised algorithm can deliver similar performance as expert-developed potentials for the magnesium oxide ionic system.  However, our Pareto optimization strategy has the inverse problem of cardinal optimizaton.  While cardinal optimization produces too few solutions.  The Pareto optimization strategy produces too many.  To solve this problem, we apply biobjective visualization techniques which projects the high dimensional data into a series of two-dimensional subspace.  This allows us to identify the performance tradeoffs in a global bivariate sense.

Chapter \ref{ch:pareto_si} introduces classification algorithms to the set of candidate potentials.  In the potential develop of a Stillinger-Weber potential for silicon, a large number of potential parameterizations are identified.  The number of potentials is so large that visualization techniques fail.  Here we apply the Gaussian mixture model (GMM) to partition the population and select a much smaller number of components determined by calculating the information criteria, which penalizes the selection of a more complicated model that adds too many components to the GMM.  This produces clusters which can be described with analytical multi-variate normal distributions.  The Student's $t$-test is used to assess the predictive performance of each material property of each cluster, and aggregate test for the performance was constructed.  This provides an algorithmic method for eliminating clusters with undesirable predictive based upon a $p$-value statistical tests.

\section{Future Work}

This new methodology is promising, but recommendations for future work and improvements are now discussed.

In order to provide continual improvement of the Pareto surface, it is necessary to constantly remove candidate potentials.  This process defines a cardinal scoring functions then ranks the potentials eliminating the poorest performance with respect to this metric.  Here, we scale the absolute errors by the target values and eliminate the worst performers by varying the volume of a hypersphere in this scaled space until a desired percentile of the candidate potentials are contained within this hypersphere.  Here, the choice of scaling is somewhat arbitrary and approximate, but serves to scale the errors to the same order of magnitude.  However, this filter seems to govern the optimization process and should be studied further.

This process produces are large number of candidate parameterizations.  However, the analysis space is high dimensional both in the design space of the potential parameters as well as the response space of predictive performance.  Linear projections into a two-dimensional subspace results in the loss of data.  The ill-conditioning of the covariance matrix suggests variance reduction techniques could be performed using eigenvalue-eigenvector analysis using a process such as principle components analysis, to project into a lower dimensional subspace without sacrificing too much information loss.

In addition, the distribution of parameters and predicted material properties are likely clustered due the existence both convex and concave regions of the Pareto surface.  We have demonstrated that clustering techniques are promising, but methodologies to translate these results into potential selection and improved sampling techniques have not been adequately explored.

On the software wide, we implement a simple parellization scheme which partitions the sample space and each processor processes an equal number of potentials.  However, some parameterizations may take more time to produce a result than others.  This leads to some processors finishing before others.  A more efficient concurrency scheme could be implemented.

Currently, the calculation of the Pareto surface and the calculation of the bandwidth parameter using cross-valdiation techniques are computational limitations since these calculations are performed on a single processor which increases memory requirements and computational time.   Parallelization schmes for both these processes would improve scalability.

Our potential produces a large amount of data.  Currently, the sampling scheme only uses information about the location of Pareto optimal points in determining random variates.  However, since the Pareto surface is the performance envelope between feasible and infeasible space, the dominated potentials provide considerable information on where to direct sample dispersions.  In addition, \emph{pypospack} produces a large amount of structured information stored in flat files.  The interactivity in analysis can be improved by integrating this information into a database, which can ease workstation analysis requirements due to the considerable memory and input-output demands of analysis.

Finally, the methodology provided here is specifed in probabilistic terms to make it compatible with VVUQ techniques.  This goal has not been accomplished due to the inadequacies of the formalism of the interatomic potential.  The reduction of epistemic uncertainty when performance tradeoffs are large implies a high degree of uncertainty in distribution of the potential parameters.  The modification of standard VVUQ techniques are necessary to deal with the inevitable biases when selecting a region of parameterization, particularly when dealing with propagating parametric uncertainty to dynamic simulations.

%\chapter{SUMMARY OF WORK}
\label{ch:summary}

\section{Significant Contributions}
This work set out to present an emergent framework for the automated development of analytical potentials.  Here we summarize key contributions in this work.

In Chapter \ref{ch:potential_development}, a critical assessment of the conventional process for potential paramaterization was provided.  We identified already known weakness with the process, but presented them in a more rigorous way.  Using cardinal optimization techniques is inappropriate because conditions for convergence to a global solutions are not met.  Convergence to global minima can only be met if the process is repeated across a dense range of initial conditions in space of possible potential parameterizations, or if random sampling techniques are employed.  In addition, the cost function formulation is unable to identify regions in the predictive performance when the optimal performance region is convex.  This creates high sensitivity to encoded preferences in convex regions due to the existence of degenerate solutions.  Finally, the \emph{a priori} expression of weights is required to be encoded at the beginning of the process.

We introduce the concept of ordinal minimization to the field of potential development, where the solution to parameter optimization is not expressed in an arbitrary cost function, which serves as an artifical heuristic, but in ordinal terms where we calculate an indifference curve based upon the concept of Pareto optimality where each candidate parameterization is mutally non-dominated.  This formulation elminates problems of local minima, expression of \emph{a priori} expression of weights, only imposes the condition the preservation ordinality to measure losses with respect to individual material properties of interest, and can identify solutions in convex regions of performance space normally occluded to cardinal optimization methods.

When the optimal performance has convexity or discontinuities, we show that the computational efficiency of Monte Carlo sampling techniquqes provides an order of magnitude improvement in  numerical efficiency in estimating the Pareto optimal surface, while simultaneously providing an order of magnitude computational efficiency in the number of simulations required.  In addition, ordinal optimization uses independent sampling which by definition provides near linear scalability in the ability of the algorithm to use processors.

Chapter \ref{ch:methodology} builds upon those concepts.  The implementation of constraints both in the domain of potential parameter design variable and the predictive performance is difficult within a cardinal optimization context, since Karush-Kuhn-Tucker conditions are likely not met, and most certaintly cannot be proved.  Cardinal optimization techniques have the undesirable task of implementing a non-linear programming scheme based upon simulations which must be treated as black box functions.  Here, we presented a modified Monte Carlo technique which uses acceptance-rejection sampling to modify the the distribution of the random variable \emph{ex post}.  This technique is considerably more flexible as ordinal optimization schemes are based upon criteria which are only dependent upon expressing an inequality.  This technique starts with the expression of the epistemic uncertainty that a potential developer has respect the location of the desireable potential parameters, and updates it in a Bayesian manner using evidence generated from Monte Carlo sampling.  This defines a new evolutionary algorithm for potential optimization that applicable for any multi-objective optimization program.

Chapter \ref{ch:software} implements these concepts into software.  A new software code \emph{pypospack} was developed to provide a framework for optimizing interatomic potentials.  Here the implementation of potentials, simulation tasks, and material properties are decomposed as independent components in an object-oriented architecture.  The use of inheritance allows us to encapsulate implementation details such as process handling, parallelization, and workflow management.  In turn, this lowers the necessary effort and expertise to contribute to the project.  An application \emph{pyposmat} which uses the \emph{pypospack} software libraries is used to implement the methodology described in Chapter \ref{ch:methodology}, and is applied to specific material systems in Chapters \ref{ch:ionic_MgO} and \ref{ch:pareto_si}.

Chapter \ref{ch:ionic_MgO} demonstrates that our unsupervised algorithm can deliver similar performance as expert-developed potentials for the magnesium oxide ionic system.  However, our Pareto optimization strategy has the inverse problem of cardinal optimizaton.  While cardinal optimization produces too few solutions.  The Pareto optimization strategy produces too many.  To solve this problem, we apply biobjective visualization techniques which projects the high dimensional data into a series of two-dimensional subspace.  This allows us to identify the performance tradeoffs in a global bivariate sense.

Chapter \ref{ch:pareto_si} introduces classification algorithms to the set of candidate potentials.  In the potential develop of a Stillinger-Weber potential for silicon, a large number of potential parameterizations are identified.  The number of potentials is so large that visualization techniques fail.  Here we apply the Gaussian mixture model (GMM) to partition the population and select a much smaller number of components determined by calculating the information criteria, which penalizes the selection of a more complicated model that adds too many components to the GMM.  This produces clusters which can be described with analytical multi-variate normal distributions.  The Student's $t$-test is used to assess the predictive performance of each material property of each cluster, and aggregate test for the performance was constructed.  This provides an algorithmic method for eliminating clusters with undesirable predictive based upon a $p$-value statistical tests.

\section{Future Work}

This new methodology is promising, but recommendations for future work and improvements are now discussed.

In order to provide continual improvement of the Pareto surface, it is necessary to constantly remove candidate potentials.  This process defines a cardinal scoring functions then ranks the potentials eliminating the poorest performance with respect to this metric.  Here, we scale the absolute errors by the target values and eliminate the worst performers by varying the volume of a hypersphere in this scaled space until a desired percentile of the candidate potentials are contained within this hypersphere.  Here, the choice of scaling is somewhat arbitrary and approximate, but serves to scale the errors to the same order of magnitude.  However, this filter seems to govern the optimization process and should be studied further.

This process produces are large number of candidate parameterizations.  However, the analysis space is high dimensional both in the design space of the potential parameters as well as the response space of predictive performance.  Linear projections into a two-dimensional subspace results in the loss of data.  The ill-conditioning of the covariance matrix suggests variance reduction techniques could be performed using eigenvalue-eigenvector analysis using a process such as principle components analysis, to project into a lower dimensional subspace without sacrificing too much information loss.

In addition, the distribution of parameters and predicted material properties are likely clustered due the existence both convex and concave regions of the Pareto surface.  We have demonstrated that clustering techniques are promising, but methodologies to translate these results into potential selection and improved sampling techniques have not been adequately explored.

On the software wide, we implement a simple parellization scheme which partitions the sample space and each processor processes an equal number of potentials.  However, some parameterizations may take more time to produce a result than others.  This leads to some processors finishing before others.  A more efficient concurrency scheme could be implemented.

Currently, the calculation of the Pareto surface and the calculation of the bandwidth parameter using cross-valdiation techniques are computational limitations since these calculations are performed on a single processor which increases memory requirements and computational time.   Parallelization schmes for both these processes would improve scalability.

Our potential produces a large amount of data.  Currently, the sampling scheme only uses information about the location of Pareto optimal points in determining random variates.  However, since the Pareto surface is the performance envelope between feasible and infeasible space, the dominated potentials provide considerable information on where to direct sample dispersions.  In addition, \emph{pypospack} produces a large amount of structured information stored in flat files.  The interactivity in analysis can be improved by integrating this information into a database, which can ease workstation analysis requirements due to the considerable memory and input-output demands of analysis.

Finally, the methodology provided here is specifed in probabilistic terms to make it compatible with VVUQ techniques.  This goal has not been accomplished due to the inadequacies of the formalism of the interatomic potential.  The reduction of epistemic uncertainty when performance tradeoffs are large implies a high degree of uncertainty in distribution of the potential parameters.  The modification of standard VVUQ techniques are necessary to deal with the inevitable biases when selecting a region of parameterization, particularly when dealing with propagating parametric uncertainty to dynamic simulations.

%\appendix{Pareto Optimization Problem}
\label{ap:pareto optimization_test}

Test function provide an artificial functional landscape which are useful to evaluate the characteristics of optimization algorithms for properties such as convergence rate, precision, robustness, and general performance.

The first of the test optimization function of a smooth convex multi-objective optimization problem described by Schaffer\cite{schaffer1984_pareto}.  Here the multi-objective function $F:\mathbb{R} \rightarrow \mathbb{R}^2$.
\begin{equation}
\begin{aligned}
  &\min_{x}
      \begin{aligned}
           &f_1(x) = x^2 \\
           &f_2(x) = (x-2)^2 \\
      \end{aligned}
\end{aligned}
\end{equation}

The second of the test optimization function is a discontinuous convex mutli-objective optimization problem described by Kursawe \cite{kursawe1991_pareto}.
\begin{equation}
\begin{aligned}
  &\min_{\bm{x}}
    \begin{aligned}[t]
      &f_(\bm{x}) = \sum_{i=1}^2
          \left[
            -10 \exp\left(-0.2 \sqrt{x_i^2} + x_{i+1^2}\right)
          \right] \\
      &f_x(\bm{x}) = \sum_{i=1}^3 \left[ \vert|x_{i} \vert|^{0.8} + 5\sin(x_i^3)\right] \\
    \end{aligned}
  &\text{subject to} \\
    \begin{aligned}[t]
      &-5 \leq x_i \leq -5 \\
      & 1 \leq i \leq 3 \\
    \end{aligned}
  \end{aligned}
\end{equation}
 % test pareto problems
%-----------------------------------------------------------------------%

% Use the appropriate file depending upon the number of appendices you have
%\include{tex/TwoOrMoreAppendices} %Use this file if you have two or more appendices
%\include{tex/OneSingleAppendix} %Use this file if you have one and only one appendix

%------------------------------------------%

% Make List of References (BibTeX implemented using the Natbib package)
% un-comment your preferred bibliography style and replace the
% bibliography file "sample" with the name of your .bib file
% REMEMBER!!! If you want un-numbered references comment the Natbib package with
% The numbered options in the packages.tex file and un-comment the package with the authoryear option
% See the included pdfs of the various styles to see the differences.
% The citation style differences are from the \citet{key} and \citep{key} commands
% More options are available; see the Natbib documentation for details

\bibliography{chapter2/body,chapter3/body,chapter4/body,chapter6/body,chapter7/ionic,chapter8/eam.bib,chapter9/body.bib}
%\bibliography{bib/sample,chapter2/body,chapter3/body,chapter4/body,chapter5/body,chapter6/body,chapter7/ionic,chapter8/eam,chapter9/body,chapter10/body,chapter11/body}
%chapter1/body.bib,chapter2/body.bib,chapter3/body.bib,chapter4/body.bib,chapter5/body.bib,chapter6/body.bib,chapter7/body.bib,chapter8/body.bib,chapter9/body.bib,chapter_silicon/silicon.bib,chapter10/body.bib}
% You can have more than one library of references - put the .bib file
% in the bib folder and call it here
%------------------------------------------%

% Bio Sketch is required and should be in third person, past tense%
% Just type your bio in between the brackets
\biography{%
Eugene Ragasa was born in Stockton, California, where he attended Saint Mary's High School.
He graduated from the United States Military Academy at West Point, New York where he studied mathematics and economics.
Upon graduation, he served as an infantry officer in the United States Army for several years before moving to New York City.
While living in New York City, he worked as a computer consultant serving a variety of industries such as telecommunications, e-commerce, and finance.
He attended Columbia University, New York where he earned a master's degree in mathematical finance, while teaching mathematics at Xavier High School in New York City.
He then worked as a quantitative analyst and trader for a variety of trading firms.
Additionally, he has been a small business owner at various times owning a computer consultancy business and a trading firm that made markets in a variety of exchange traded products.

After careers in technology and finance, he moved back to his hometown with an interest in applying his skills to science and engineers.
He earned a master's degree in mechanical engineering from the University of the Pacific.  With an interest in computational simulations, he joined the research group of Simon Phillpot to study atomistic simulations, which he currently does do this day.
}


%------------------------------------------%

\end{document}

%-------------------------------------------------------------------------------------------------------%
