\chapter{SUMMARY OF WORK}

\section{General Implications}

\section{Future Work}

\subsection{Removal of Pathological Potentials}

\subsection{Scalability and Stability of Numerical Algorithms}
A common problem in these numerical algorithms comes from ill-condition matrices and the use of numerical algorithms in scikit-learn.  Even though all variance-covariance matrices are by definition positive definitive, an ill-conditioned matrix can cause the algorithms to to fail.  The Cholesky decomposition should be replaced with the single value decomposition

\subsection{Algorithms to Prevent Population Crowding}

\subsection{Evolve distribution away from dominated regions}

\subsection{Dimensionality Reduction Methods}

\subsection{UQ quantification applications}


\section{Robustness of Potentials}
Sensitive analysis.  Many models use cutoff-radii as an additional parameter to be determined by the fitting process.  Because of the nature of the stacking fault, the GSF energy must depend upon non-nearest neighbor interactions.
However, a slight alteration of the cutoff-radius was found to have a dramatic impact on the behavior of EAM potentials by Zimmerman.%\cite{Zimmerman2000_Ni_gsf}.
He sugests that parameterizeed models should be only slightly sensitive to a change of value for any of it's parameters.  High sensitivity indicates the possibility of unphysical behavior of a model.  This issue should be address when creating a potential to be used in simulations that involve stacking faults.
