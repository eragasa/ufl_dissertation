\chapter{THE MANY BODY PROBLEM AND ATOMISTIC METHODS}

The properties of a system may be obtained by solving the quantum mechanical (QM) wave equation which governs the system dynamics.  For non-relativistic system, this equation is the Schrodinger's equation.  For all but the simplest systems, this approach in an impossible task in practice; the resulting many body problem has only been solved for a limited number of system.  Within this chapter we outline the many body problem, it's intractibility before considering the Hohenberg-Kohn-Sham formulation of density functional theory (DFT), particularly in it's formulation it's application for systems with periodic boundary conditions.  This reformulates quantuim mechanics, using electron density as the fundamental parameter to solve, rather than the many-electron wavefuction.  This takes the $N$-body problem and recasts it into $N$ single-body problems; which is a dramatic simplification.

We then approach higher order models which reduces computational intensity by looking at classical empirical potentials and their role in both molecular dynamics and lattice dynamics.

% http://cmt.dur.ac.uk/sjc/thesis_prt/node21.html
\section{The Many-Body Problem}
Hartree and Hartree-Fock Methods
\section{Density Functional Theory}
\subsection{The Exchange Correlation Term}
\section{Density Functional Theory for Solids}
\subsection{Representation of an Infinite Solid}
\subsection{Bloch's Theorem}
\subsection{Plane Wave Formulation}
\subsection{k-point sampling}
\section{Empirical Interatomic Potentials}
The interatomic potential $U(\bm{R}_i)$ derived from the Born-Oppenheimer approximation is derived from a quantum-mechanical perspective.  The computational cost of \emph{ab initio} such as density-functional theory (DFT) can provide accurate structural energies and forces, but their computational cost limits approaches to compute $U(\bm{R}_i)$ makes the scientific inquiry of systems requiring longer simulation times or larger number of atoms to captures relevant feature sizes unreasonable.

An empirical interatomic potential $V(\bm{R}_i;\bm{\theta})$ is an analytical function parameterized by $\bm{\theta}=(\theta_1,...,\theta_n)$ which is meant to approximate $U(\bm{R}_i)$.  The total energy of a potential of $N$ atoms with an interaction described by the empirical potential, $V$, can be expanded in a many body expansion.
\begin{equation}
	V(\bm{r}_1,...,\bm{r}_N)= \sum_i V_1(\bm{r}_i)
	                          + \sum_i \sum_{i<j} V_2(\bm{r}_i,\bm{r}_j)
				  + \sum_i \sum_{i<j} \sum_{j<k} V_3(\bm{r}_i,\bm{r}_j,\bm{r}_k) + ...
\end{equation}
The first term $V_1$ is the one body term, due to an external field or boundary conditions, which is typically ignored in classical potentials.  The second term $V_2$ is the pair potential, the interaction of the term is dependent upon the distance between $\bm{r}_i$ and $\bm{r}_j$.  The three-body term potential $V_3$ arises when the interaction interaction of a pair of atoms is modified by the presence of a third.  Based upon this expansion, we can classify certain potentials into two classes: pair potentials when only $V_2$ is present and many-body potentials when $V_3$ and higher order terms are included.

Over the last few decades, a large number of potentials have been developed to descibe various bonding types and environments.  To take representative examples, the Lennard Jones was developed for the van der Waals interactions of noble gases, pair potentials such as the Buckingham potential can be used for ionic solids, the embedded atom model (EAM) is developed for metallic systems, the Assisted Model Building with Energy Refinement (AMBER) for biomolecules, the tersoff potential for covalently bonded materials.  To deal with bonding and chemical environments for heterogenous materials like metal/metal oxide interfaces have led to extensions such as MEAM, REBO, COMB, and ReaxFF.

More recently, potentials such as GAP and SNAP represent the atomic environment of an atom not from a collection of a vectors of atomic positions which feed into formulaic functional forms, but to calculate the bispectrum of the neighborhood of atoms.  The bispectrum combined with an orthogonal expansion of components is dependent upon large amounts of density functional images to use in the fitting dataset to produce DFT fidelity reproductions of interatomic forces on an atom.
\subsection{Pair Potentials}
\subsubsection{Leonnard Jones}
\subsubsection{Coulomb Potential}
\subsubsection{Born-Mayer Potential}
\subsubsection{Morse Potential}
\subsubsection{Buckingham Potential}
\subsection{Many Body Potentials}
\subsubsection{Tersoff Potentials}
\subsubsection{Stillinger Weber Potentials}
\subsubsection{Embedded Atom Method Potentials}
This chapter reviews typical approaches to fitting empirical potentials.
\section{Interatomic Potentials}
The justification for the use of classical empirical potentials can be demonstrated from the Born-Oppenheimer approximation\cite{born1927_bo}.  The Hamiltonian for a real material is defined by the presence of interacting nuclei and electrons:
\begin{equation}
	H = \sum_i \frac{P_i}{2M_i}
	    + \sum_\alpha \frac{p_\alpha}{2m}
	    + \frac{1}{2} \sum_{ij} \frac{Z_i Z_j e^2}{r_{ij}}
	    + \frac{1}{2} \sum_{\alpha\beta} \frac{e^2}{r_{\alpha\beta}}
	    - \sum_{i\alpha} \frac{Z_i e^2}{r_{i\alpha}}
\end{equation}
The first terms are kinetic energy terms, the latter terms are the nuclei-nuclei, electron-electron, and nuclei-electron interactions.  Ideally, the solution of Schr\"{o}dinger's equation, $H\Psi=E\Psi$ could be solved providing the total wavefunction $\Psi(\bm{r}_i,\bm{r_\alpha})$.  Except for the simplest of systems, this approach is impossible computationally.
The Born-Oppenheimer approximation \cite{born1927_bo} is ubiquitous in \emph{ab initio} calculations, and forms the justification for classical empirical potentials.  The kinetic energy is ignored since the heavy nulclei move more slowly than electrons.  For the remaining interaction terms of the Hamiltonian, the nuclear positions are clamped at certain positions in space, the electron-nuclei interactions are not removed, since the electrons are still influenced by the Coulomb potential of the nuclei.  This allows us to factor the wavefunction as
\begin{equation}
	\Psi(\bm{R}_i,\bm{r}_\alpha) = \Xi(\bm{R}_i)\Phi(\bm{r}_\alpha;\bm{R}_i)
\end{equation},
where $\Xi(\bm{R}_i)$ describes the nuclei, and $\Phi(\bm{r}_\alpha;\bm{R}_i)$ describes the electrons parameterized by the clamped position of $\bm{R}_i$.  In turn, the Hamiltonian is solve able as two Schr\"{o}dinger's equations.  The first equation contains the electronic degrees of freedom.
\begin{equation}
\label{eq:BO_electronic}
     H_{e}\Phi(\bm{r}_\alpha;\bm{R}_i)=U(\bm{R}_i)\Phi(\bm{r}_\alpha;\bm{R}_i)
\end{equation}
where
\begin{equation}
	H_e = \sum_\alpha \frac{p_\alpha}{2m}
	      + \frac{1}{2} \sum_{ij} \frac{Z_i Z_j e^2}{r_{ij}}
	      + \frac{1}{2} \sum_{\alpha\beta} \frac{e^2}{r_{\alpha\beta}}
	      - \sum_{i\alpha} \frac{Z_i e^2}{r_{i\alpha}}
\end{equation}
Eqn. \ref{eq:BO_electronic} gives the energy $U(\bm{R}_i)$ which depends on the clamped coodinates of $\bm{R}_i$.  The electronic effects are contained in $U(\bm{R}_i)$ and is incorporated into the second equation which the motion of the nuclei
\begin{equation}
\label{eq:BO_nuclei}
    H_n\Xi(\bm{R}_i)=E\Xi(\bm{R}_i)
\end{equation}
where
\begin{equation}
\label{eq:H_n}
    H_n = \sum_i \frac{P_i}{2m_i} + U(\bm{R_i})
\end{equation}
This later equation does not contain any electronic degrees of freedom, because all electronic effects are incorporated into $U(\bm{R}_i)$ which is the interatomic potential.  For molecular dynamics, Schr\"{o}dinger's equation is replaced with Newton's equation of motion.
\section{Molecular Dynamics}
Molecular dyanmics (MD) is a simulation approach where the time evolution of aset of interacting atoms is followed by numerically solving their equations of motion.  In MD, the behavior of atoms follow Newtonian mechanics:
\begin{equation}
    M\frac{d\bm{r}(t)}{dt} = F(\bm{r}(t))=-\nabla V(\bm{r}(t))
\end{equation}
where $t$ is time, $\bm{r}(t) = (\bm{r}_1(t),\bm{r}_2(t),...,\bm{r}_N(t))$ represents the forces on the particles, and $M$ is the mass matrix, which is a diagonal matrix with the mass, $m_k$, for $M_{k,k}=m_k$ for all diagonal entries.
The total energy is conserved, even if the kinetic energy and potential energy can change dynamically.  In Hamiltonian form, the Newtonian equation of motion can be written Hamiltonian form (Allen, Tildesley, et al 1989).
\begin{equation}
	\frac{d \bm{r}}{dt}=\frac{\partial H{\bm{r},\bm{p}}}{\partial \bm{p}},
	\frac{d \bm{p}}{dt}=\frac{\partial H{\bm{r},\bm{p}}}{\partial \bm{r}}
\end{equation}
Therefore,
\begin{equation}
	\frac{dH}{dt}=\frac{\partial H(\bm{r},\bm{p})}{\partial \bm{r}} \frac{d\bm{r}}{dt}
						   +\frac{\partial H(\bm{r},\bm{p})}{\partial \bm{p}} \frac{d\bm{p}}{dt} = 0
\end{equation}

\subsection{Numerical Integration}
A dynamical simulation computes atomic positions as a function of time given their initial position $\bm{r}(t=0)$ and velocities $\bm{v}(t=0)$.  Since Newton's equations of motion are 2nd order diffferential equations, an initial condition needs to specify both positions and velocities of all atoms at the initial condition.  To solve the equation of motion computationally, we need to descretize time.  Usually, time is descretized uniformly, $t_n=n\Delta t$, where $\Delta t$ is referred to as the time step.  The task of the simulation algorithm is to find $\bm{r}(t_n)$ for $i=1,2,3...$ (Allen, Tildesley \emph{et al} 1989).

The Verlet algorithm begins by approximating
\begin{equation}
	\frac{d^2 r(t)}
	     {dt^2}
	= \frac{\bm{r}(t+\Delta t) - 2\bm{r}(t) + \bm{r}(t-\Delta t)}
	       {\Delta t^2}
\end{equation}

Thus,
\begin{equation}
  \frac{\bm{r}(t+\Delta t) - 2\bm{r}(t) + \bm{r}(t-\Delta t)}
	     {\Delta t^2} = - \frac{1}{m} \frac{dU(\bm{r}(t))}{d\bm{r}} \\
	\bm{r}(t+\Delta t)
	     = 2\bm{r}(t)
			 		- \bm{r}(t-\Delta t)
	     		- \Delta t \frac{1}{m} \frac{dU(\bm{r}(t))}{d\bm{r}}
\end{equation}


\subsection{Thermodynamic Ensembles}
\subsubsection{Microcanonical (NVE) ensemble}
\subsubsection{Canonical (NVT) ensemble}
\subsubsection{NPT}
\section{Lattice Dynamics}
\section{Calculation of Material Properties}
\subsection{Structural Properties}
\subsubsection{Minimization Techniques}
\emph{Greatest Descent}
\emph{Conjugate Gradient}
\subsection{Phase Order Properties}
\subsection{Point Defect Formation Energies}
\subsection{Surface Energies}
\subsection{Stacking Fault Energies}

\section{Notation}
\subsection{Simulation Cell}
A simulation cell is defined by the lattice basis and the atomic basis.  The lattice vectors which describes the periodic boundary conditions three lattice vectors $\bm{a},\bm{b},\bm{c}$Euclidean space which forms the basis for the crystallographic system when periodic boundary conditions are applied.  The translational properties of a crystal allows the simulation of an infinite bulk material from a fixed volume.  In traditional crystallography, the boundaries of the unit cell were defined as $a,b,c$ corresponding to the length of each lattice vector and the angles $\alpha,\beta,\gamma$.  In computatonal materials, a more convenient representation
