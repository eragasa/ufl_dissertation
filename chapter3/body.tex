\chapter{A PARETO APPROACH TO PARAMETER OPTIMIZATION}

Many decision and plannning problems involve multiple conflicting criteria which must be considered simultaneously.  In the field of optimization, problems which have multiple criteria are deferred to as multiple critieria decision making problems (MCDM) and the algorithms used to solve them as multiple-objective optimization (MOO).  Here the set of feasible solutons is not known in advance, but is restricted by constraint functions.  We concentrate on nonlinear multiobjective optimization and ignore approaches designed for multiobjective linear programming.

Additionally, the approach described within this chapter is described briefly and is followed on by more detailed discussions and application in following chapters and appendices.  The emphasis of the development of this methdology will be on the mathematical aspects of the subject and it's applications to the development of classical empirical potentials.  The intent is to provide tools for a decision maker, rather than to convince which particular optimization to use.

Secondly, another purpose of the development of this methdology is the development of parameterizations which can be expressed as an ensemble of potentials described as a probability distribution function which can be used in UQ propogation.

\section{Multiobjective optimization}

The general multi-objection optimization problem using the notation of Marler and Arora\cite{Marler2004_moo_survey}:
%\begin{equation}
\begin{align}
  & \underset{\bm{x}}{\text{minimize}}
        &\bm{F}(\bm{x}) = [ F_1(\bm{x}),
                            F_2(\bm{x}),
                            ...,
                            F_k(\bm{x})]^T\\
  & \text{subject to}
        &g_j(\bm{x}) \leq 0, j=1,2,...,m \\
  &     &h_l(\bm{x}) = 0, l=,1,2,...,e \\
  &     &\bm{x} \in \bm{X}
\end{align}
%\end{equation}
where $k$ is the number of objective functions, $m$ is the number of inequality constraints, and $e$ is the number of equality constraints.
The vector $\bm{x} \in \bm{X} \subseteq \mathbb{R}^n$ is a vector design variables $x_i$, and $X$ is feasible design space.
$\bm{F}(\bm{x}) \in \mathbb{R}^k$
are called objectives, cost functions, or criteria.  The feasible critereon space $Z$ is defined as
$\{\bm{F}(\bm{x}) | \bm{x} \in \bm{X} \}$.

For MOOs, the objectives are generally conflicting, preventing simultaneous optimization.

\subsection{Pareto optimality}

If all functions are for minimization, a feasible solution $\bm{x}_1$ is said to dominate another feasible solution $\bm{x}_2$, denoted $\bm{F}(\bm{x}_1 \succ \bm{x}_2)$, if and only if $F_i(\bm{x}_1)\leq F_i(\bm{x}_2)$ for $i = 1,...,k$ and $F_j(\bm{x}_1)< F_j(\bm{x}_2$) for at least one objective function $j$.  A solution is said to be Pareto optimal if it is not dominated by any other solution in the solution space.  A Pareto optimal solution cannot be improved with respect to any objective without worsening at least one objective function.  The set of all feasible non-dominated solutions in $X$ is referred to as the Pareto optimal set, and for a given Pareto optimal set, the corresponding values in the objective space are called the Pareto Front.

\section{Parameter Estimation as a Multiobjective Optimization Problem}

It is rather trivial to state that parameter estimation is an MOO problem.  However,
while this is often stated occasionally in potential development literature.  It is often stated within the context of the use of global optimization techniques.  The purpose of this section is to provide a clear methodological approach to determining optimal parameters within the context of MOO, and eludicate the problems often encountered in potential development, specifically to the choice of the optimization techniques often employed in potential optimization.

\subsection{Parameters}
Let $V$ be an empirical potential parameterized by $P$ number of parameters $\bm{\theta}=[\theta_1,\theta_2,...\theta_P]$.
\subsection{Constraints on parameters}
\subsection{Structure Property Relationships}
\subsection{Constraints on structure property relationships}
\subsection{Fitting Database}
A fitting database is a collection of structure property functions $q_i$ with an associated structures $S_j$.  The notation of $q$ comes from verification, validation, and uncertainty quantification literature where the term quantity of interest (QOI)

Lattice constant, bulk modulus, vacancy formation energy, or anything that can be defined from energy structures.  In the fitting database, the structure proerty functions evaluated using an empirical potentials and compared to target reference values, with values either determined from experimental values or a a high-fidelity structure such as DFT.

The collection of structure property relationships, is denoted $\bm{q}=(q_1,q_2,...q_N)$ for $N$ structure property relationships.  Usually accuracy and transferribility are tested against an external database.
\subsection{Prediction Error function}
In order to assess the prediction errors of the structure property functions, we denote the
      $\hat{\bm{q}}(\bm{\theta})=(
          \hat{q}_1(\bm{\theta}),
          \hat{q}_2(\bm{\theta}),
          ...,
          \hat{q}_N(\bm{\theta}))$
    as the predicted material properties

The difference between the prediction values and target values of the QOIs produces a vector of error functions, $\bm{\epsilon}(\bm{\theta})=(
        \hat{q}_1(\bm{\theta})-q_1,
        \hat{q}_2(\bm{\theta})-q_2,
        ...,
        \hat{q}_N(\bm{\theta})-q_N)$,
\subsection{Parameter Optimization Problem as MOO formulation}
DEFINITION OF CONFIGURATION SPACE

The incorporation of first-principles data in the fitting database significantly improves the reliability of semi-empirical potentials by sampling a larger area of configuration space[21-28]

Ercolessi F, Adams JB. Europhys Lett 1994;26:583.
Mishin Y, Farkas D, Mehl MJ, Papaconstantopoulos DA. Phys Rev B 1999;59:3393
Baskes MI, Asta M, Srinivasan SG. Philos Mag A 2001;81:991
Mishin Y, Mehl MJ, Papaconstantopoulos DA, Voter AF, Kress JD. Phys Rev B 2001;63:224106
Mishin Y, Mehl MJ, Papaconstantopoulos DA. Phys Rev B 2002;65:224114
Li Y, Siegel DJ, Adams JB, Liu XY. Phys Rev B 2003;67:125101
Zope RR, Mishin Y. Phys Rev B 2003;68:024102
Mishin Y. Acta Mater 2004;52:1451

\section{Pareto Front}

In multiobjective optimization problems, it is characteristic that no unique solution exists, but a set of mathematically equally good solutions can be identified.  These solutions are known as nondominated, efficient, noninferior or Pareto optimal solutions.  In MCDM literature, these terms are synomous.

In MCDM literature, the idea of solving a multiobjective optimization problem is understood as helping a human decision maker (DM) in understanding the multiple objectives simultaneously and finding a Pareto optimal solution.  Thus, the solution process requires some interaction with the DM in the form of specifying preference information and the final solution is determined by these preferences.

In potential development, the preferences of potential developer likewise influences are particular parameterization, which has results in the development of empirical potentials as somewhat of a black art.  In the end, empirical potentials are simplified models which predict structure property relationships.

In classical potential optimization, the identification of an optimal parameterization is determined by the minimization of a cost function which couples multiple objective functions, usually a weighted sum of squares, and different weights are used in an interactive fashion until an acceptable parameterization is determined.
\section{Surveys of Methods}
Chankong and Haimes 1983
Hwang and Masud 1979
Marler and Arora 2004
Miettinen 1999
Sawaragi et al 1985
Steuer 1987
Vincke 1992

We start our review of methods using Hwang and Masud 1979 and Miettinen 199, to classify the different classes of approaches by methological approach rather than technical techniques.
\subsection{no preference methods}
The task is to find some neutral compromise solution without any additional information.  This means instead of asking the DM for preference information, some assumption are made about what a reasonable compromise could be like.
\subsection{\emph{a priori} methods}
In \emph{a priori}, the DM first articulates preference information and the solution tries to find a Pareto optimal solution satisfying them as well as possible.

\subsection{\emph{a posteriori} methods}
A representation of a set of Pareto optimal solution is first generated and then the DM is supposed to select the most preferred one among them.  This approach gives the DM an overview of different solutions available but if there are more than two objectives in the problem, it may be difficult for the DM to analyze the large amount of information.

\subsection{Interactive methods}
After each iteration, some information is provided to the DM in order to specify preference information.  What is noteworthy is that the DM can specify and adjust one's preferences between each iteration and at the same time learn about interdepencies between each iteration and at the same time learn about interdependencies in the problem as well as one's own preferences.
\section{Solution Methods}
MOO solution methods fall under the category of scalarization or non-scalarization methods.  Scalarization is the primary method for MOO problems [Miettinen 1999].  Scalarization converts the MOO problem into a paramterized single-objective problem which can be solved using using well-established single-objective optimization methods.
\subsection{Scalarization Methods}
\subsubsection{Weighting Method}
\subsection{Cost Function}
\begin{equation}
  C(\bm{\theta})=\sum w_i (\hat{q}_i(\bm{\theta})-q_i)^2
\end{equation}

Gass and Saaty 1955
Zadeh 1963

For a interatomic potential being fit with respect to $k$ quantities of interest,
\begin{equation}
  \begin{aligned}
  & \underset{\bm{\theta}}{\text{minimize}}
        & & \sum_{i=1}^{k}w_{i} \varepsilon_i^2(\bm{\theta}) \\
  & \text{subject to}
        & & \bm{\theta} \in \bm{\Theta}
  \end{aligned}
\end{equation}
where $w_i \geq 0$ for $i=1,...,k$
Weakly Pareto optimal.

  In the development of interatomic potentials, the DM is asked to specify weights in which case the method is used as an \emph{a priori} method.

Algorithms for multiobjective optimization should produce Pareto optimal solutions, and that any Pareto optimal solution can be found.  Censor1977 discusses the conditions which the whole Pareto set can be generated by the weighting metho when positive weights are presented.  In this respect, the weighting method has a serious shortcoming.  Any Pareto optimal solution can be found by altering weights only if the problem is convex.  Some Pareto optimal solutions of nonconvex problems cannot be found regardless of how the weights are selected.

The problems of the weighting schemes have ben explored by the classical potential development community.  The method may jump from one vertex to anohter vertex leaving intermediate solutions undetected with relatively small changes in the weighting schemes.

Scaling of the objective functions.

The weighting method can be used as an \emph{a posteriori} method where different weight can be used to generate different Pareto optimal solutions, and then the DM selects the most satisfactory solution.  Systemic methods of perturbing the weights to obatain different Pareto optimal solutions are suggested (Chankong and Haimes 1983), but Das and Dennis 1997 illustrates that an evenly distributed set of weights does not necessarily produce an evenly distributed representation of the Pareto optimal set, even when the problem is convex.

When the weighting scheme is used as an \emph{a priori} method, the DM is expected to represent his/her preferences in the form of weights.  Roy and Mousseau (1996) suggests that the role of weights in expressing preferences maybe mis leading.  Although the relative importance of weights show the relative importance of the objective functions it is not clear what underlies this notion.  The relative importance of objective functions is usually understood globally, for the entire decision problem, while many practical applications show that the importannce typically varies for different objective function values, that is, the concept is only meaningful locally. (Podinovsky 1994).

Weights that produce a certain Pareto optimal solution are not necessarily unique, and different weights may produce similar solutions.  On the other hand, a small change in weights may cause big differences in the objective function.  It is not easy for the potential developer to control the solution process because weights behave in an indirect way.  The solution process then becomes an interactive one where the DM trues to guess such weights that would produce a satisfactory solution, and this is not desirable because the DM cannot be properly suppported which leads to frustation complications in potential development.  Instead, in such cases it is advisable to use real interactive methods where the DM can better control the solution process with more intuitive preference information.

The weighting method is also difficult
%https://books.google.com/books?id=NHZqCQAAQBAJ&pg=PA1&source=gbs_toc_r&cad=4#v=onepage&q&f=false

\section{Optimization Methods}

\begin{equation}
  \begin{aligned}
  & \underset{\bm{x}}{\text{minimize}}
        & &  f(x)\\
  & \text{subject to}
        & & g_i(x) \leq 0
        & & h_j(x) = 0
        & & x \in X
  \end{aligned}
\end{equation}
here $x$ is the optimization variable, $f$ is the objective function, $g_i$ are inequality constraints, and $h_j$ are equality constraint functions.

\subsection{Convex Optimization}
Numerical algorithms make heavy use of scalarization results, and most papers in the field of MOO and economics deal with non-linear programming problems, corresponding duality theorems, and the repeated application of the simplex method.

However, within the literature of potential development approaches focus upon local minimization techniques and global optimization techniques.

objective function is concave.  constraint set is convex.  KKT requirements for uniqueness.

\subsection{Global Approaches}

Genetic algorithms are a popular meta-heuristic that is particularly well-suited for this class of problems.  Traditional GA are customized to accomodate multi-objective problems by using specialized fitness functions and introducing methods to promote solution diversity.

The second general approach is to determine an entire Pareto optimal solution set or a representative subset. A Pareto optimal set is a set of solutions that are nondominated with respect to each other. While moving from one Pareto solution to another, there is always a certain amount of sacrifice in one objective(s) to achieve a certain amount of gain in the other(s). Pareto optimal solution sets are often preferred to single solutions because they can be practical when considering real-life problems since the final solution of the decision-maker is always a trade-off. Pareto optimal sets can be of varied sizes, but the size of the Pareto set usually increases with the increase in the number of objectives.

The ultimate goal of a multi-objective optimization algorithm is to identify solution in the Pareto optimal set.  However, identifiying the entire Pareto optimal set, for multi-objective problems, is impossible to its size.  Proof of solution optimality is computationally infeasible.  Therefore, a practical approach is achieve successively better approximations of the Pareto surface that represent the Pareto set as well as possible.

A multi-objective optimization approach should achieve the following conflicting goals as described by Zitzler \emph{et al}\cite{zitzler2000_moo_evolve}: (1) the best known Pareto front should be as close as possible to the true Pareto front.  Ideally, the best-known Pareto set should be a subset of the Pareto set, (2) solutions in the best known Pareto set should be uniformly distributed and diverse over the Pareto front in order to provide the decision-maker a true picture of trade-offs, and (3) the best-known Pareto front should capture the whole spectrum of the Pareto front at the extreme ends of the spectrum.  While the first two goals are important for multi-objective optimization, the last goal is erroneous.  In general, when developing potentials, the DM is interested in compromise solutions and a parameterization with high fidelity with respect to one material property at the expense of a loss of fidelity with respect to all other prediction would be a pathological parameterization.

\subsubsection{Genetic Algorithms}
The method which will be proposed in chapter 5 is not a genetic algorithm, but has many similarities as Genetic Algorithms but tailored to create an ensemble of Pareto optimal parameters.  However, it is a genetic solution and the iterative approach of generating new populations is akin to previous solutions.  As a result, the section of review in this section is necessarily incomplete but refer to an introductory review by Konak \emph{et al}\cite{Konak2006_moo_ga} as well as the book by Deb\cite{deb2001_moo_ga}

The concept of genetic algorithms were inspired by evolutionist theories explaining the origin of species\cite{holland1992_ga}.  In nature, weak and unfit speicies within their environment are faced with extinction by natural selection, while strong ones pass on their genes to future generations through reproduction.  In the long run, species carrying the correct combinatioon in their
    genes become dominant in their population.

In GA terminology, a solution vector $\bm{x}\in\bm{X}$ is called an individual or a chromosome.  Chromosomes are made of descrete units called genes.  Each gene controls on or more features of the chromosome.  Normally, a chromosome corresponds to a unique solution $\bm{x}$ in the solution space.  This requires a mapping mechanism between the solution space and chromosome.  GA operates with a collection of chromosomes, called a population.  As the search evolves, the poulation includes fitter and fitter positions, eventually it converges, meaning that it is dominated by a single solution.  Two operators are defined crossover and mutation.  In the crossover operator, two parent solutions are combined togehter to form offspring.  The mutation operator introduces random changes into the population.

The first multi-objective GA, called vector evaluated GA (or VEGA), was proposed by Schaffer [5]. Afterwards, several multi-objective evolutionary algorithms were developed including Multi-objective Genetic Algorithm (MOGA) [6], Niched Pareto Genetic Algorithm (NPGA) [7], Weight-based Genetic Algorithm (WBGA) [8], Random Weighted Genetic Algorithm (RWGA)[9], Nondominated Sorting Genetic Algorithm (NSGA) [10], Strength Pareto Evolutionary Algorithm (SPEA) [11], improved SPEA (SPEA2) [12], Pareto-Archived Evolution Strategy (PAES) [13], Pareto Envelope-based Selection Algorithm (PESA) [14], Region-based Selection in Evolutionary Multiobjective Optimization (PESA-II) [15], Fast Non-dominated Sorting Genetic Algorithm (NSGA-II) [16], Multi-objective Evolutionary Algorithm (MEA) [17], Micro-GA [18], Rank-Density Based Genetic Algorithm (RDGA) [19], and Dynamic Multi-objective Evolutionary Algorithm (DMOEA) [20]. Note that although there are many variations of multi-objective GA in the literature, these cited GA are well-known and credible algorithms that have been used in many applications and their performances were tested in several comparative studies.



\emph{Vector Evaluated Genetic Algorithm (VEGA)}.  Schaffer proposed VEGA for finding multiple solutions to multiobjective problems.  He created VEGA to find and maintain multiple classification rules in a set covering problem.  VEGA tried to achieve this goal by selecting a fraction of the next generation using one of the objective functions.

Fitness Sharing encourage the search in unexplored section of a Pareto front by artificially thinning solutions in densely populated area.  To achieve this goal, densely populated areas are identified and a penalty method is used to penalize the solutions located in such areas.  This approach was recommended by Goldberg and Richardson\cite{goldberg1987genetic} and used by Fonseca and Fleming\cite{fonseca1993multiobjective} to penalize clustered solutions.
\begin{equation}
    dz(\bm{x}_1,\bm{x}_2)
    = \sqrt{\sum_{k=1}^{K}  \left(\frac{z_k(\bm{x_1})-z_k(\bm{x_2})}
                                       {z_{k}^{max}-z_{k}^{min}}
                            \right)^{2}
      }
\end{equation}

based on these distances, calculate a niche count for each solution $\bm{x}\in\bm{X}$ as
\begin{equation}
  nc(\bm{x}_1,t)=\sum_{\bm{x}_2\in\bm{X},r(\bm{x}_2,t)=r(\bm{x}_1,t)}
      \max\left\{ \frac{\sigma_{share}-dz(\bm{x}_1,\bm{x}_2)}
                       {\sigma_{share}},0
          \right\}
\end{equation}
where $\sigma_{share}$ is the niche size by defining a neighborhood of solutions in the objective space.  Solutions in the same neighborhood contribute to each other's nich count.  Therefore, a solution in a crowded neighborhood will have a higher niche count, reducing the probability of selecting that solution from being culled from the survivor set.

\section{Visualization}
This problem is dealt with in discussions about visualization and and analysis of the large amounts of data generated from a posteriori approaches to solving these problems.
Edgeworth 1881
Koopmans 1951
Kuhn Tucker 1951
Pareto 1896, 1906
\section{Treatment}

Our treatment of the mapping of the empirical potential is treated as a bijective mapping into two measure spaces.

Let us define parameter space with the probability measure space $(\Theta,\mathcal{F}(\Theta),\mathbb{P})$.

Then we define the error space of the structure property relationships with the probability measure space $(\mathcal{E},\mathcal{F}(\mathcal{E})),\mathbb{Q})$.
