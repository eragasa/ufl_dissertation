\chapter{DIMENSIONALITY REDUCTION}

In the development of empirical interatomic potentials (EIP), the natural dimensionality of the problem can be daunting.  The representation of the atomic structures, $\bm{R}$, also known as the configuration space.

In this section, we look into exploratory measures to tackling analysis and sampling in this high-dimensional space.

We have established in a previous section that even when we a single objective function,
%\begin{equation}
%	C(\theta)=\Sum_{N} w_i L_i(\theta)
%\end{equation}
that when the problem is convex, the optimization is likely NP-hard.
As a result, we are looking at mechanisms which reduce the size of the effective search space.
We expect our initial algorithm to provide many candidate potentials initially, but since sampling techniques converge slowly, the arrival of new candidate potentials slows down, causing the Kullbach-Liebach divergence indicator to indicate an early end to the iteration loop.

\section{Potential}

The interatomic potential, $V$ can be represented in the Born-Oppenheimer approximation as a mapping between the configurational space of atoms, $\bm{X}$, onto the set of possible energies, $U$.
Since $V:\bm{X} \rightarrow U$, then we can denote this relation as, $V(\bm{x})$, to represent energy for a structure, and $V(\bm{X})$ as potential energy surface.

Since \emph{ab initio} calculations are computational expensive, empirical interatomic potentials, $\hat{V}$ which which are defined by formulae, are often used to simulate larger systems and large time scales.
If $\bm{\Theta}$ represents the feasible parameterizations for $\hat{V}$, then the vector of $P$ optimal parameters, $\bm{\theta}^{*} = (\theta_1^{*},...,\theta_P^{*})$, must be be contained within the feasible parameterization, $\bm{\theta}^{*} \in \bm{\Theta}$.
Since $\hat{V}:\bm{\Theta},\bm{X} \rightarrow U$, then it is clear that
\begin{align}
  V &= \hat{V} + \varepsilon \\
  V(\bm{x}) &= \hat{V}(\bm{x},\bm{\theta}) + \varepsilon(\bm{x},\bm{\theta})
\end{align}
where $\varepsilon$ is the residual difference between $V(\bm{x})$ and $\hat{V}(\bm{x})$
 are defined by formulae which are parameterized for a specific material system.  Using the hat notation to denote an estimate,   The set of optimal parameters, $\bm{\theta}^{*} = (\theta_1^{*},...,\theta_P^{*})$ for $P$ parameters

\section{Principal Components Analysis}

Principal component analysis (PCA) is a statistical procedure that uses an orthogonal transformation to convert a set of observations of possibly correlated variables into a set of values linearly uncorrelated variables called principal compoents.

Pearson[1] and Hotelling[2]

\section{Manifold Learning Learning}

\section{Clustering Algorithms}
